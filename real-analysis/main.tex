% main.tex

\input{preamble}
\usepackage{titlesec}
\usepackage[many]{tcolorbox}

% Adjust spacing after the chapter title
\titlespacing*{\chapter}{0cm}{-2.0cm}{0.50cm}
\titlespacing*{\section}{0cm}{0.50cm}{0.25cm}

\titleformat*{\section}   {\bfseries\sffamily\Large}
\titleformat*{\subsection}{\bfseries\sffamily\large}
\titleformat*{\subsubsection}{\bfseries\sffamily\normalsize}
\titleformat*{\paragraph} {\bfseries\sffamily\normalsize}
\titleformat*{\subparagraph}{\bfseries\sffamily\normalsize}

% Indent 
\setlength{\parindent}{0pt}
\setlength{\parskip}{1ex}

% --- Theorems, lemma, corollary, postulate, definition ---
\usepackage[dvipsnames]{xcolor}
\usepackage{titlesec}
\usepackage[many]{tcolorbox}
\tcbuselibrary{theorems,skins,breakable}

% Custom colors
\definecolor{MyBlue}{HTML}{37CDDE}
\definecolor{MyGreen}{HTML}{50C878}
\definecolor{MyPurple}{HTML}{7851A9}
\definecolor{MyGold}{HTML}{D4AF37}
\definecolor{MyGray}{HTML}{808080}

\newtcbtheorem[]{theorem}{Theorem}%
  {enhanced,
   colback        = red!20,
   colbacktitle   = red!30,
   coltitle       = black,
   boxrule        = 0pt,
   frame hidden,
   arc            = 2mm,
   before skip    = 3ex,
   after skip     = 3ex,
   before title   = {\vspace{2mm}},
   after title    = {\vspace{2mm}},
   fonttitle      = \bfseries\sffamily,
   breakable
  }{theorem}

\newtcbtheorem[]{proposition}{Proposition}%
  {enhanced,
   colback        = MyGold!20,
   colbacktitle   = MyGold!30,
   coltitle       = black,
   boxrule        = 0pt,
   frame hidden,
   arc            = 2mm,
   before skip    = 3ex,
   after skip     = 3ex,
   before title   = {\vspace{2mm}},
   after title    = {\vspace{2mm}},
   fonttitle      = \bfseries\sffamily,
   breakable
  }{proposition}

\newtcbtheorem[]{lemma}{Lemma}%
  {enhanced,
   colback        = MyBlue!20,
   colbacktitle   = MyBlue!30,
   coltitle       = black,
   boxrule        = 0pt,
   frame hidden,
   arc            = 2mm,
   before skip    = 3ex,
   after skip     = 3ex,
   before title   = {\vspace{2mm}},
   after title    = {\vspace{2mm}},
   fonttitle      = \bfseries\sffamily,
   breakable
  }{lemma}

\newtcbtheorem[]{corollary}{Corollary}%
  {enhanced,
   colback        = MyGreen!20,
   colbacktitle   = MyGreen!30,
   coltitle       = black,
   boxrule        = 0pt,
   frame hidden,
   arc            = 2mm,
   before skip    = 3ex,
   after skip     = 3ex,
   before title   = {\vspace{2mm}},
   after title    = {\vspace{2mm}},
   fonttitle      = \bfseries\sffamily,
   breakable
  }{corollary}

\newtcbtheorem[]{definition}{Definition}%
  {enhanced,
   colback        = MyPurple!20,
   colbacktitle   = MyPurple!30,
   coltitle       = black,
   boxrule        = 0pt,
   frame hidden,
   arc            = 2mm,
   before skip    = 3ex,
   after skip     = 3ex,
   before title   = {\vspace{2mm}},
   after title    = {\vspace{2mm}},
   fonttitle      = \bfseries\sffamily,
   breakable
  }{definition}

\newtcbtheorem[]{remark}{Remark}%
  {enhanced,
   colback        = MyGray!20,
   colbacktitle   = MyGray!30,
   coltitle       = black,
   boxrule        = 0pt,
   frame hidden,
   arc            = 2mm,
   before skip    = 3ex,
   after skip     = 3ex,
   before title   = {\vspace{2mm}},
   after title    = {\vspace{2mm}},
   fonttitle      = \bfseries\sffamily,
   breakable
  }{remark}

\newtcbtheorem[]{example}{Example}%
  {enhanced,
   colback        = Violet!20,
   colbacktitle   = Violet!30,
   coltitle       = black,
   boxrule        = 0pt,
   frame hidden,
   arc            = 2mm,
   before skip    = 3ex,
   after skip     = 3ex,
   before title   = {\vspace{2mm}},
   after title    = {\vspace{2mm}},
   fonttitle      = \bfseries\sffamily,
   breakable
  }{example}

\newtcbtheorem[]{problem}{Problem}%
  {enhanced,
   colback        = MyGold!20,
   colbacktitle   = MyGold!30,
   coltitle       = black,
   boxrule        = 0pt,
   frame hidden,
   arc            = 2mm,
   before skip    = 3ex,
   after skip     = 3ex,
   before title   = {\vspace{2mm}},
   after title    = {\vspace{2mm}},
   fonttitle      = \bfseries\sffamily,
   breakable
  }{problem}


% --- Basic commands ---
%   Euler's constant
\newcommand{\eu}{\mathrm{e}}

%   Imaginary unit
\newcommand{\im}{\mathrm{i}}

%   Sexagesimal degree symbol
\newcommand{\grado}{\,^{\circ}}

% --- Comandos para álgebra lineal ---
% Matrix transpose
\newcommand{\transpose}[1]{{#1}^{\mathsf{T}}}

%%% Comandos para cálculo
%   Definite integral from -\infty to +\infty
\newcommand{\Int}{\int\limits_{-\infty}^{\infty}}

%   Indefinite integral
\newcommand{\rint}[2]{\int{#1}\dd{#2}}

%  Definite integral
\newcommand{\Rint}[4]{\int\limits_{#1}^{#2}{#3}\dd{#4}}

%   Dot product symbol (use the command \bigcdot)
\makeatletter
\newcommand*\bigcdot{\mathpalette\bigcdot@{.5}}
\newcommand*\bigcdot@[2]{\mathbin{\vcenter{\hbox{\scalebox{#2}{$\m@th#1\bullet$}}}}}
\makeatother

%   Hamiltonian
\newcommand{\Ham}{\hat{\mathcal{H}}}

%   Trace
\renewcommand{\Tr}{\mathrm{Tr}}

% Christoffel symbol of the second kind
\newcommand{\christoffelsecond}[4]{\dfrac{1}{2}g^{#3 #4}(\partial_{#1} g_{#2 #4} + \partial_{#2} g_{#1 #4} - \partial_{#4} g_{#1 #2})}

% Riemann curvature tensor
\newcommand{\riemanncurvature}[5]{\partial_{#3} \Gamma_{#4 #2}^{#1} - \partial_{#4} \Gamma_{#3 #2}^{#1} + \Gamma_{#3 #5}^{#1} \Gamma_{#4 #2}^{#5} - \Gamma_{#4 #5}^{#1} \Gamma_{#3 #2}^{#5}}

% Covariant Riemann curvature tensor
\newcommand{\covariantriemanncurvature}[5]{g_{#1 #5} R^{#5}{}_{#2 #3 #4}}

% Ricci tensor
\newcommand{\riccitensor}[5]{g_{#1 #5} R^{#5}{}_{#2 #3 #4}}

% Shortcuts
\newcommand{\cl}{\mathcal}
\newcommand{\bb}{\mathbb}
\newcommand{\mb}{\mathbf}
\newcommand{\fr}{\mathfrak}
\newcommand{\oname}{\operatorname}
\newcommand{\ovl}{\overline}
\newcommand{\st}{\;\biggr\rvert\;}
\newcommand{\mr}{\mathrm}
\newcommand{\ds}{\mathds}

% Standard vertical spacing
\newcommand{\stdvspace}{\vspace{0.5ex}}

% Text formatting
\newcommand{\tit}[1]{\textit{#1}}
\newcommand{\tbf}[1]{\textbf{#1}}

% Horizontal line
\newcommand{\horizontal}{\noindent{\rule{\textwidth}{0.4pt}}}



\begin{document}

\begin{Large}
    \textsf{\textbf{Princeton Lectures in Analysis: Real Analysis}}
\end{Large}

\vspace{1ex}

\textsf{\textbf{Student:}} Joshua Lin \\
\textsf{\textbf{Lecturer:}} Assaf Naor, Jacob Shapiro, Allan Sly

\vspace{2ex}
Problems are derived largely from \emph{Integration Theory and Hilbert Spaces} by Stein and Shakarchi and \emph{Real and Complex Analysis} by Rudin. The requisite content for solving the problems comes from the corresponding sections of these \href{https://web.archive.org/web/20250630154205/https://web.math.princeton.edu/~js129/PDFs/teaching/MAT425_spring_2025/MAT425_Lecture_Notes.pdf}{lecture notes}. Some solutions are not my own, but rather thanks to the MAT425 TAs at Princeton (Spring 2025). 
\stdvspace

Convention: if not otherwise specified, \(m\) will denote the Lebesgue measure.

\section{Measures}

\begin{problem}{Topology of the Cantor Set}*
[S\&S E1.1.] Prove that the Cantor set \(\cl{C}\) constructed in the text is nowhere dense but perfect. In other words, given two distinct points \(x, y \in \cl{C}\), there is a point \(z \not\in \cl{C}\) that lies between \(x\) and \(y\), and yet \(\cl{C}\) has no isolated points. 
\end{problem}

\begin{proof}
    We begin by showing that \(\cl{C}\) is nowhere dense. Consider any distinct \(x, y \in \cl{C}\), and let \(d := |x - y|\). Furthermore, select \(k\) such that \(3^{-k} < d\). At each step of Cantor's process \(\cl{C}_k\) (starting with \(\cl{C}_0 := [0, 1]\)), the remaining disjoint intervals are each of length \(3^{-k}\). Hence by construction, \(x\) and \(y\) lie in different intervals of \(\cl{C}_k\), so there is some \(z \not\in \cl{C}_k\) between them. But \(\cl{C}_k \supseteq \cl{C}\), so \(z\) is not in \(\cl{C}\) either. 
    \stdvspace

    Now, we show \(\cl{C}\) has no isolated points. Choose any \(x \in \cl{C}\), and any ball \(B_{\varepsilon}(x)\) around it. Choose some \(k\) such that \(3^{-k} < \varepsilon\). Since \(x \in \cl{C} \implies x \in \cl{C}_k\), it must be that \(x\) and a (distinct) endpoint \(y\) of some interval in \(\cl{C}_k\) satisfy \(|x - y| < \varepsilon\), so \(y \in B_{\varepsilon}(x)\). But since the endpoints of such intervals are never removed, we have \(y \in \cl{C}\). 
\end{proof}

\begin{problem}{Ternary Expansions of the Cantor Set}*
[S\&S E1.2.] The Cantor set \(\cl{C}\) can also be described in terms of ternary expansions.

\begin{enumerate}[(a)]
    \item Every number in \([0,1]\) has a ternary expansion
    \[
        x = \sum_{k=1}^\infty a_k 3^{-k}, \quad \text{where } a_k = 0, 1, \text{ or } 2.
    \]
    Note that this decomposition is not unique since, for example, \(1/3 = \sum_{k=2}^\infty 2/3^k\).

    Prove that \(x \in \cl{C}\) if and only if \(x\) has a representation where every \(a_k\) is either \(0\) or \(2\).
    
    \item The \textbf{Cantor function} is defined on \(\cl{C}\) by
    \[
        F(x) = \sum_{k=1}^\infty \frac{b_k}{2^k} \quad \text{if } x = \sum_{k=1}^\infty a_k 3^{-k}, \text{ where } b_k = a_k / 2.
    \]
    In this definition, we choose the expansion of \(x\) in which \(a_k = 0\) or \(2\).

    Show that \(F\) is well defined and continuous on \(\cl{C}\), and moreover \(F(0) = 0\) as well as \(F(1) = 1\).

    \item Prove that \(F : \cl{C} \to [0,1]\) is surjective, that is, for every \(y \in [0,1]\) there exists \(x \in \cl{C}\) such that \(F(x) = y\).

    \item One can also extend \(F\) to be a continuous function on \([0,1]\) as follows. Note that if \((a,b)\) is an open interval of the complement of \(\cl{C}\), then \(F(a) = F(b)\). Hence we may define \(F\) to have the constant value \(F(a)\) in that interval.
\end{enumerate}
\end{problem}

\begin{proof}
    \emph{Part (a).} Consider \(x \in \cl{C}\). We will construct a sequence of ternary numbers \(\{a_k\}_k\) such that \(x = \sum_k a_k 3^{-k}\) as desired. First, define \(a_1\) to be the smallest ternary number satisfying 
    \[
        \frac{a_1}{3} \leq x \leq \frac{a_1}{3} + \frac{1}{3}
    \]
    Now, suppose the first \(n-1\) ternary numbers \(\{a_k\}_{k=1}^{n-1}\) have been constructed as such, so that 
    \[
    \sum_{k=1}^{n-1} \frac{a_k}{3^k} \leq x \leq \sum_{k=1}^{n-1} \frac{a_k}{3^k} + \frac{1}{3^{n-1}}
    \]
    Then, select \(a_n\) as the smallest ternary number satisfying 
    \[
    \sum_{k=1}^{n} \frac{a_k}{3^k} \leq x \leq \sum_{k=1}^n \frac{a_k}{3^k} + \frac{1}{3^n}
    \]
    Thus we have a ternary sequence \(\{a_n\}\) satisfying the above inequalities. Taking the limit \(n \to \infty\) on both sides, \(x = \sum_k a_k 3^{-k}\) as desired. 
    \stdvspace

    For the reverse direction, take any \(x\) with such a ternary representation. We will show by induction that it is contained in all \(\cl{C}_n\). Let \(x = \sum_k a_k 3^{-k}\) for \(a_k\in \{0, 2\}\). If \(a_1 = 0\), then \(x < 1/3\) so \(x \in \cl{C}_1\); on the other hand if \(a_1 = 2\), then \(x \geq 2/3\) so \(x \in \cl{C}_1\). Now suppose \(x \in \cl{C}_{n-1}\); then by construction of the Cantor set, it must be in some interval \(I_{n-1} = [i_{n-1}, i_{n-1} + 3^{-(n-1)}]\). If \(a_n = 0\), then \(x < i_{n-1} + 3^{-n}\) so \(x \in \cl{C}_n\), and if \(a_n = 2\), then \(x \geq i_{n-1} + 2 \cdot 3^{-n}\) so \(x \in \cl{C}_n\) as well. Hence
    \[
    \forall n \in \bb{N} : x \in \cl{C}_n \implies x \in \bigcap_{n=1}^\infty \cl{C}_n = \cl{C}
    \]
    and that completes the proof. 
\end{proof}

\begin{proof}
    \emph{Part (b).} Clearly \(F : \cl{C} \to [0, 1]\) is well-defined. To show continuity on \(\cl{C}\), for any \(x \in \cl{C}\) and \(\varepsilon > 0\) we must show existence of some \(\delta > 0\) such that \(|x - y| < \delta \implies |f(x) - f(y)| < \varepsilon\). First, select some \(n \in \bb{N}\) such that \(2^{-n} < \varepsilon\), and let \(\delta = 2 \cdot 3^{-n}\). If \(|x - y| < \delta\), then in their ternary \(\{0, 2\}\) representations the first \(n\) digits must agree. Hence \(|F(x) - F(y)| < 2^{-n} < \varepsilon\) as desired. 
    \stdvspace

    Finally, we can write 
    \[
    0 = \sum_{k \in \bb{N}} 0 \cdot 3^{-k}, \quad 1 = \sum_{k \in \bb{N}} 2 \cdot 3^{-k}
    \implies F(0) = 0, F(1) = 1
    \]
    thus completing the proof. 
\end{proof}

\begin{proof}
    \emph{Part (c).} Write \(y = \sum_k b_k 2^{-k}\). Clearly \(x = \sum_k (2a_k) \cdot 3^{-k}\) satisfies \(F(x) = y\). 
\end{proof}

\begin{proof}
    \emph{Part (d).} The precise definition of the extension is \(\tilde{F}(x) = \sup_{y \in \cl{C}, y \leq x} F(y)\) for \(\tilde{F} : [0, 1] \to [0, 1]\). Clearly \(\tilde{F}\) is continuous at each \(x \in \cl{C}\), from part (b). So, take any \(x \not\in \cl{C}\). Since \([0,1] \setminus \cl{C}\) is open, there exists a ball \(B_{\delta}(x) \subseteq [0,1] \setminus \cl{C}\) containing \(x\). For any \(\varepsilon > 0\), clearly \(|x - y| < \delta \implies |\tilde{F}(x) - \tilde{F}(y)| = 0 < \varepsilon\), so \(\tilde{F}\) is continuous at all \(x \not\in \cl{C}\) as well. 
\end{proof}

\begin{problem}{Cardinality of \(\sigma\)-algebras}*
    [RCA E1.1.] Does there exist an infinite \(\sigma\)-algebra with only countably many elements? 
\end{problem}

\begin{proof}
    No. Consider, for contradiction, an infinite \(\sigma\)-algebra \(\fr{M}\) with countably many elements. The key observation is that \(\fr{M}\) must have a countable sequence of non-empty, pairwise disjoint subsets. To see why, suppose that a largest collection of non-empty, pairwise disjoint elements is \(\{D_i\}_{i=1}^N\) for \(N>1\) (since \(A\) and \(X \setminus A\) are disjoint for \(A \neq \varnothing, X\)). We will show that there are at least \(N+1\) non-empty pairwise disjoint elements, contradicting. 
    \stdvspace
    
    Since \(\fr{M}\) has infinitely many elements, there exists a non-empty set \(D\) which cannot be written as a union of some of these \(D_i\), as only finitely many such unions exist. If \(D\) is disjoint to all \(D_i\), then we are done, since \(\{D_i\}_i \cup D\) forms \(N+1\) non-empty pairwise disjoint elements of \(\fr{M}\). So, instead suppose \(D\) intersects some subset \(\{D_{i_j}\}_j\). Because \(D\) is not just a union of the \(D_i\), there exists a set \(D_{k}\) for which \(D' := D \cap D_{k} \subsetneq D_{k}\) is non-empty. Defining \(D'' := D_{k} \setminus D'\), we see \(\{D_i\}_{i \neq k} \cup D' \cup D''\) is a collection of \(N+1\) non-empty pairwise disjoint elements of \(\fr{M}\), a contradiction. 
    \stdvspace

    Thus take a countable sequence of non-empty pairwise disjoint elements \(\{ E_n \}_n \subseteq \fr{M}\). For each binary number \(b \in [0, 1)\), write \(b = 0.\overline{b_1 b_2 b_3 \dots}\) and take \(b \mapsto \bigcup_{n \mid b_n = 1} E_n\). This map is a bijection, since the \(E_n\) are pairwise disjoint and non-empty. Thus \(|\fr{M}| \geq |[0, 1)| = 2^{\aleph_0}\), so \(\fr{M}\) is not countable, contradicting our original assumption. 
\end{proof}

\begin{problem}{Comparison Sets are Measurable}*
    [RCA E1.5(a).] Suppose \(f,g : X \to \overline{\bb{R}}\) are measurable. Then the comparison sets 
    \[
        \{x \mid f(x) < g(x)\}, \quad \{x \mid f(x) = g(x)\}
    \]
    are measurable. [Hint: Look at the rationals between \(f\) and \(g\).]
\end{problem}

\begin{proof}
    If we can show that \(\{x \mid f(x) < g(x)\}\) is measurable, then 
    \[
        \{x \mid f(x) = g(x)\} = X \setminus (\{x \mid f(x) < g(x)\} \cup \{x \mid f(x) > g(x)\})
    \] 
    is also measurable. If \(f(x) < g(x)\), then there exists a rational \(q\) such that \(f(x) < q < g(x)\). The idea is to form preimages of intervals using all such \(q\), since 
    \[
    \{x \mid f(x) < g(x) \}
    = \bigcup_{q \in \bb{Q}} \{x \mid f(x) < q < g(x)\}
    = \bigcup_{q \in \bb{Q}} f^{-1}([-\infty, q)) \cap g^{-1}((q, \infty])
    \in \fr{M}
    \]
    since intervals are measurable and thus their preimages are as well. 
\end{proof}

\begin{problem}{Boundary of Closure with Positive Measure}*
    [S\&S E1.9.] Give an example of an open set \(\cl{O}\) with the following property: the boundary of the closure of \(\cl{O}\) has positive Lebesgue measure; i.e., \(m(\partial \overline{\cl{O}}) > 0\). 
\end{problem}

\begin{proof}
    Let \(\cl{C}\) be the fat Cantor set. Define 
    \[
        V_n := \left\{ x \in \bb{R} \mid \frac{1}{n+1} < d(x, \cl{C}) < \frac{1}{n} \right\},
        \quad
        U := \bigcup_{n=1}^\infty V_{2n}, 
        \quad
        U' := \bigcup_{n=1}^\infty V_{2n-1}
    \]
    TODO. 
\end{proof}


\begin{problem}{The Lebesgue-Stieltjes Measure}*
    [MAT425 P3.5.] Let \(F : \bb{R} \to \bb{R}\) be increasing and right continuous. Let \(\cl{E}\) be Lebesgue elementary family consisting of all intervals of type \((a, b]\) or \((a, \infty)\) including the empty set \(\varnothing\), and let \(\cl{A}\) be the Lebesgue algebra generated by its finite disjoint unions. Define now 
    \[
        \rho_F(S) = \begin{cases}
            \sum_{j=1}^n (F(b_j) - F(a_j)) & S = \bigcup_{j=1}^n (a_j, b_j] \\
            0 & S = \varnothing 
        \end{cases}
        \qquad (S \in \cl{A})
    \]
    Show that \(\rho_F\) is a premeasure on \(\cl{A}\). Apply Caratheodory's construction to obtain a Borel measure \(\mu\), which is known as the \emph{Lebesgue-Stieltjes measure.} (Note that if \(F\) is the identity, then \(\mu = m\), the Lebesgue measure). Show that each of the following equations holds: 
    \begin{align*}
        \mu(\{a\}) &= F(a) - \lim_{\varepsilon \to 0^+} F(a - \varepsilon) \\
        \mu([a, b)) &= \lim_{\varepsilon \to 0^+} [F(b - \varepsilon) - F(a - \varepsilon)] \\
        \mu([a, b]) &= F(b) - \lim_{\varepsilon \to 0^+} F(a - \varepsilon) \\
        \mu((a, b)) &= \lim_{\varepsilon \to 0^+} F(b - \varepsilon) - F(a) 
    \end{align*}
\end{problem}

\begin{proof}
    TODO. 
\end{proof}


\begin{problem}{The Hausdorff Measure}*
    [MAT425 P3.15.] Let \(X\) be a metric space. Define for any \(S \subseteq X, \delta > 0, d \in \bb{R}\) 
    \[
        H_\delta^d(S) = \inf \left( \left\{ \sum_{n=1}^\infty (\oname{diam}(U_n))^d \st \{U_n\}_{n=1}^\infty \subseteq X, \, S \subseteq \bigcup_{n=1}^\infty U_n, \, \oname{diam}(U_n) < \delta \right\} \right)
    \]
    Here \(\oname{diam}(U_n)\) is the diameter of the set defined via the metric on \(X\). 
    \begin{enumerate}[(a)]
        \itemsep0em
        \item Show that \(\delta \mapsto H_\delta^d(S)\) is monotone increasing. 
        \item Define 
        \[
            H^d(S) := \lim_{ \delta \to 0^+} H_\delta^d(S)
        \]
        Show that \(S \mapsto H^d(S)\) is an outer measure. The induced measure is called the \emph{\(d\)-dimensional Hausdorff measure on \(X\)}.  
        \item Show that all Borel subsets \(\fr{B}(X)\) (with respect to the metric topology on \(X\)) are \(H^d\)-measurable, in the sense of Caratheodory. 
        \item Show that for any \(S \subseteq X\),
        \[
            d \mapsto H^d(S) \qquad (0 \leq d < \infty)
        \]
        is monotone decreasing and its image is \(\{H^{d_*(S)}(S), 0, \infty\}\) for some \(0 \leq d_*(S) < \infty\) such that \(0 < H^{d_*(S)}(S) < \infty\). The number 
        \[
            d_*(S) := \inf \{ d \geq 0 \mid H^d(S) = 0 \}
        \]
        is called the \emph{Hausdorff dimension of \(S\)}. 
        \item Let \(d \in \bb{N}\). Show that if \(X = \bb{R}^d\) and \(m : \fr{B}(\bb{R}^d) \to [0, \infty]\) is the Lebesgue measure restricted to Borel sets, then 
        \[
            m(E) = \beta_d H^d(E)
            \qquad 
            (E \in \fr{B}(\bb{R}^d))
        \]
        for some constant \(\beta_d\) which only depends on \(d\). Calculate the constant. 
    \end{enumerate}
\end{problem}

\begin{proof}
    TODO. 
\end{proof}


\begin{problem}{Borel-Cantelli Lemma}*
    [S\&S E1.16.] Suppose \(\{E_k\}_k\) is a countable family of measurable subsets of \(\bb{R}^d\) and that \(\sum_k m(E_k) < \infty\). Let 
    \[
    E := \{x \in \bb{R}^d : x \in E_k, \text{ for infinitely many } k\}
    \equiv \limsup_{k \to \infty} (E_k)
    \]
    \begin{enumerate}[(a)]
        \itemsep0em
        \item Show that \(E\) is measurable. 
        \item Prove \(m(E) = 0\). 
    \end{enumerate}
    In particular, if the sum of the individual probabilities of infinitely many events is finite, then the probability that infinitely many of them occur is zero. [Hint: Write \(E = \bigcap_n \bigcup_{k \geq n} E_k\).]
\end{problem}

\begin{proof}
    \emph{Part (a).} Observe \(x \in E_k\) for infinitely many \(k\) if and only if for each \(n \in \bb{N}\), \(x\) resides in at least one \(E_k\) for \(k \geq n\). So, \(E = \bigcap_n \bigcup_{k \geq n} E_k\). Then measurability follows since 
    \[
    \{E_k\}_k \subseteq \fr{L}(\bb{R}^d)
    \implies \bigcup_{k = n}^\infty E_k \in \fr{L}(\bb{R}^d) \quad (n \in \bb{N})
    \implies E = \bigcap_{n=1}^\infty \bigcup_{k = n}^\infty E_k \in \fr{L}(\bb{R}^d)
    \]
    and that completes the proof.
\end{proof}

\begin{proof}
    \emph{Part (b).} Observe that for all \(n \in \bb{N}\), \(E \subseteq \bigcup_{k \geq n} E_k\). By monotonicity, thus 
    \[
        m(E) 
        \leq m \left( \bigcup_{k = n}^\infty E_k \right)
        \implies 
        m(E) 
        \leq \lim_{n \to \infty} m\left(\bigcup_{k = n}^\infty E_k \right) 
        = m\left(\lim_{n \to \infty} \bigcup_{k = n}^\infty E_k \right) 
        = m(\varnothing) 
        = 0
    \]
    and so \(E\) has Lebesgue-measure zero. 
\end{proof}

\begin{problem}{Dominating Almost-Finite Functions with Constants}*
    [S\&S E1.17.] Let \(\{f_n : [0, 1] \to \widehat{\bb{C}}\}_n\) be a sequence of measurable functions with \(|f_n(x)|<\infty\) for almost every \(x\). Show that there exists a sequence \(c_n\) of positive real numbers such that \(f_n(x) / c_n \to 0\) for almost every \(x\). [Hint: Pick \(c_n\) such that \(m(\{x : |f_n(x) / c_n| > 1/n\}) < 2^{-n}\), and apply the Borel-Cantelli Lemma.]
\end{problem}

\begin{proof}
    By replacing \(f_n\) with \(|f_n|\), without loss of generality let \(f_n : [0,1] \to [0, \infty]\). Since \(f_n < \infty\) almost everywhere, 
    \[
    m\left( \bigcap_{M = 1}^\infty \left\{ x \in [0, 1] \st f_n(x) > M \right\} \right) = 0
    \qquad (n \in \bb{N})
    \]
    Therefore, fix any \(n\), and choose \(M_n\) large enough such that 
    \[
    m\left( \left\{ x \in [0, 1] \st f_n(x) > M_n \right\} \right) < \frac{1}{2^n}
    \implies 
    m\left( \left\{ x \in [0, 1] \st \frac{f_n(x)}{c_n} > \frac{1}{n} \right\} \right) < \frac{1}{2^n}
    \]
    by writing \(c_n := nM_n\). Let \(E_n\) denote the set in the last step. Since \(\sum_{n} m(E_n) = 1 < \infty\), by the Borel-Cantelli Lemma it must be that \(m(E) = 0\) where \(E := \limsup_n E_n\). Thus every \(x \in [0, 1] \setminus E\) is in at most finitely many \(E_n\), so for \(n\) large enough (depending on \(x\)), 
    \[
    \frac{f_n(x)}{c_n} \leq \frac{1}{n} \implies \lim_{n \to \infty} \frac{f_n(x)}{c_n} = 0
    \]
    and thus this property holds for almost every \(x\). 
\end{proof}

\begin{problem}{Measurable Functions as Continuous Limits}*
    [S\&S E1.18.] Suppose \(f : \bb{R} \to \bb{C}\) is a measurable function. Then there exist \(\{f_n : \bb{R} \to \bb{C}\}\) continuous and a set \(E\) of Lebesgue-measure zero such that 
    \[
    \lim_{n \to \infty} f_n(x) = f(x) \qquad (x \in \bb{R} \setminus E)
    \]
    In other words, every measurable function is the limit pointwise almost everywhere of a sequence of continuous functions. 
\end{problem}

\begin{proof}
    Let the ``good'' functions \(\cl{G}\) be the functions which satisfy the desired property. Let the ``good'' subsets of \(\bb{R}\) be those whose characteristic functions satisfy the property. The proof will be deconstructed into eight steps. 

    \textbf{Step 1.} \emph{The intervals \((a, \infty)\) are good.}

    Explicitly construct the continuous functions
    \[
    f_n : \bb{R} \to \bb{C}, \quad 
    f_n(x) = \begin{cases}
    0 & x < a \\
    n(x - a) & a \leq x < a + \frac{1}{n} \\
    1 & a + \frac{1}{n} \leq x
    \end{cases}
    \]
    Observe that \(f_n \to \cl{X}_{(a, \infty)}\) everywhere, so the intervals \((a, \infty)\) are good. 

    \textbf{Step 2.} \emph{The good functions \(\cl{G}\) form a commutative ring.}

    Suppose \(f, f' \in \cl{G}\). Then, respectively, there exists \(\{f_n\}_n\) and \(\{f_n'\}_n\) continuous such that \(f_n \to f\) and \(f_n' \to f'\) except on null sets \(E\) and \(E'\). Thus \(\{f_n + f_n'\}_n\) tend to \(f+f'\) except on the null set \(E \cup E'\), so \(f + f' \in \cl{G}\). Similarly, \(f_n f_n' \to ff'\) on the same set, so \(f f' \in \cl{G}\). Thus \(\cl{G}\) is closed under addition and multiplication.

    Then, suppose \(f \in \cl{G}\) and \(\alpha \in \bb{C}\). Clearly \(\alpha f_n \to \alpha f\) except on a null set \(E\), so \(\alpha f \in \cl{G}\). Hence \(\cl{G}\) is closed under scalar multiplication. It follows that \(\cl{G}\) is a commutative ring. 

    \textbf{Step 3.} \emph{Every set in the Lebesgue algebra \(\cl{A}\) is good.}

    Recall that the Lebesgue elementary family (as defined in Folland) is the collection of sets \(\cl{E} = \{ \varnothing \} \cup \{ (a, b] \subseteq \bb{R} \} \cup \{ (a, \infty) \subseteq \bb{R} \}\), and that the Lebesgue algebra \(\cl{A}\) is constructed via finite disjoint unions of elements of \(\cl{E}\). 
    
    It suffices to show that every set in \(\cl{E}\) is good: by Step 2, if \(S_1, \dots, S_n\) are disjoint and good, then \(\cl{X}_{\bigcup_{i=1}^n S_i} = \sum_{i=1}^n \cl{X}_{S_i}\) is good as well. Clearly \(\varnothing\) is good, and Step 1 shows that \((a, \infty)\) is as well. But also \(\cl{X}_{(a, b]} = \cl{X}_{(a, \infty)} + (-1) \cdot \cl{X}_{(b, \infty)}\) is good by Step 2, so actually every set in \(\cl{E}\) is good. 

    \textbf{Step 4.} \emph{The pointwise limit of a sequence of good functions, if it exists, is good.}

    Suppose \(f_n \to f\) for \(\{f_n\}\) good. For each \(n\), there exists \(\{f_{nm}\}_m\) continuous which tend to \(f_n\) pointwise almost everywhere. We wil show that \(f_k' \to f\) where \(k\) is indexed according to the triangular numbers as in Cantor's diagonalization argument, 
    \[
    k := T_{n+m-1} + n = \frac{(n+m-1)(n+m)}{2} + n, 
    \quad 
    f_k' := f_{nm}
    \]
    Observe that \(k\) is increasing in both \(n\) and \(m\), so for any \(\varepsilon > 0\) and \(k\) large enough, 
    \[
    |f_k' - f| 
    \leq |f_k' - f_n| + |f_n - f|
    = |f_{nm} - f_n| + |f_n - f|
    \leq \frac{\varepsilon}{2} + \frac{\varepsilon}{2}
    = \varepsilon
    \]
    outside at most \(E := \bigcup_{n,m} E_{nm}\). Thus \(f_k' \to f\) except on \(E\), but by Cantor's diagonalization theorem, this union is countable and thus has measure zero. 

    \textbf{Step 5.} \emph{The good sets form a \(\sigma\)-algebra.}

    Let \(\fr{M}\) be the collection of all good sets. It's clear that \(\bb{R} \in \fr{M}\) since \(\cl{X}_{\bb{R}}\) is continuous. Moreover, if \(E \in \fr{M}\), then \(\bb{R} \setminus E \in \fr{M}\) since \(\cl{X}_{\bb{R} \setminus E} = \cl{X}_{\bb{R}} + (-1) \cdot \cl{X}_E \in \fr{M}\) from Step 2. Finally, take \(\{E_n\}_n \subseteq \fr{M}\), and without loss of generality suppose they are pairwise disjoint (otherwise, write \(E_n' := E_n \setminus \bigcup_{k<n}E_k\)). Then, 
    \[
    \cl{X}_{\bigcup_{n=1}^\infty E_n} 
    = \sum_{n=1}^\infty \cl{X}_{E_n} 
    = \lim_{N \to \infty} \sum_{n=1}^N \cl{X}_{E_n}
    \in \cl{G}
    \]
    since each partial sum is good by Step 2, and the resulting limit is good by Step 4. Hence \(\fr{M}\) is a \(\sigma\)-algebra. 

    \textbf{Step 6.} \emph{The Lebesgue-measurable sets are good.}

    Denote the Lebesgue-measurable sets by \(\fr{L}(\bb{R})\). Since \(\cl{A} \subseteq \fr{M}\) and \(\fr{M}\) is a \(\sigma\)-algebra by Step 5, we have \(\sigma(\cl{A}) = \fr{B}(\bb{R}) \subseteq \fr{M}\) as well. Since \(\fr{L}(\bb{R})\) is the completion of \(\fr{B}(\bb{R})\), it suffices that the sets with Lebesgue-measure zero are good. But since \(f_n = 0\) converges pointwise a.e. to \(\cl{X}_N\) for any such null set \(N\), this fact is immediate. 

    \textbf{Step 7.} \emph{The measurable functions \(f : \bb{R} \to [0, \infty)\) are good.}

    There exists a sequence of simple functions \(\{s_n\}_n\) such that \(s_n \to f\) pointwise. By Step 6, every simple function is good since it is the sum of characteristic functions of Lebesgue-measurable sets, and thus by Step 4, \(f\) is good. 

    \textbf{Step 8.} \emph{The measurable functions \(f : \bb{R} \to \bb{C}\) are good.}

    Since \(\Re(f)^+, \Re(f)^-, \Im(f)^+, \Im(f)^- : \bb{R} \to [0, \infty)\) are measurable and thus good, so is \(f\). 
\end{proof}

\begin{problem}{Measure of the Minkowski Sum}*
    [S\&S E1.20.] Show that there exist closed sets \(A,B \subseteq \bb{R}^d\) with \(m(A) = m(B) = 0\), but \(m(A+B) > 0\):
    \begin{enumerate}[(a)]
        \itemsep0em
        \item In \(\bb{R}\), let \(A = \cl{C}\) and \(B = \cl{C}/2\). Note that \(A+B \supseteq [0, 1]\).
        \item In \(\bb{R}^2\), observe that if \(A = I \times \{0,1\}\) and \(B = \{0\} \times I\) for \(I = [0,1]\), then \(A+B = I \times I\). 
    \end{enumerate}
\end{problem}

\begin{proof}
    The first observation is that if \(A,B\) are closed, then \(A+B\) is measurable. Since \(\bb{R}^d\) is \(\sigma\)-compact, we can write 
    \[
    A = \bigcup_{n=1}^\infty \, (A \cap K_n), \quad B = \bigcup_{n=1}^\infty \, (B \cap K_n) \qquad (K_n := [-n, n]^d)
    \]
    If we label \(A_n := A \cap K_n\) and \(B_n := B \cap K_n\), then each \(A_n\) and \(B_n\) is closed. Therefore, 
    \[
    A + B
    = \left( \bigcup_{n=1}^\infty A_n \right) + \left( \bigcup_{n=1}^\infty B_n \right)
    = \bigcup_{n, m=1}^\infty \, (A_n + B_m)
    \]
    which is a countable union thanks to Cantor's diagonalization argument. Thus \(A+B\) is \(F_\sigma\) and so it is measurable. 

    \emph{Part (a).} Let \(A = \cl{C}\) and \(B = \cl{C} / 2\), where \(\cl{C}\) is the Cantor set. Clearly \(A\) is closed, since its complement is a union of intervals and thus open. But \(B\) is also closed, since \(f : \bb{R} \to \bb{R}\) with \(x \mapsto 2x\) is continuous and \(f^{-1}(A) = B\). We can characterize \(A\) as the ternary numbers in \([0,1]\) which consist of digits \(0\) or \(2\), so \(B\) is the ternary numbers in \([0,1]\) which consist of digits \(0\) or \(1\). Since \(A+B \supseteq A \cup B \supseteq [0,1]\), and \(A+B\) is measurable from above, we have \(m(A+B) > 0\). 

    \emph{Part (b).} Let \(A = I \times \{0\}\) and \(B = \{0\} \times I\) for \(I = [0,1]\). Both \(m(A) = m(B) = 0\). But since \(A+B = I \times I\), \(m(A+B) = 1 > 0\). 
\end{proof}

\begin{problem}{Continuous Transformations Do Not Preserve Measurability}*
    [S\&S E1.21.] Prove that there is a continuous function that maps a Lebesgue measurable set to a non-measurable set. [Hint: Consider a non-measurable subset of \([0,1]\) and its preimage in \(\cl{C}\) by the Cantor function.]
\end{problem}

\begin{proof}
    Take any non-measurable subset of \([0,1]\), such as the Vitali set \(\cl{V}\). Let \(f : [0,1] \to [0,1]\) be the Cantor function, which is continuous on this domain. Observe that \(\cl{C} \cap f^{-1}(\cl{V})\) is measurable since the Lebesgue measure is complete. But since \(f \vert_{\cl{C}}\) is surjective, 
    \[
    f(\cl{C} \cap f^{-1}(\cl{V}))
    = f(f\vert_{\cl{C}}^{-1}(\cl{V}))
    = f\vert_{\cl{C}}(f\vert_{\cl{C}}^{-1}(\cl{V})) \cup f\vert_{[0,1] \setminus \cl{C}} (f\vert_{\cl{C}}^{-1}(\cl{V}))
    = (\cl{V} \cap f\vert_{\cl{C}}(\cl{C})) \cup \varnothing
     = \cl{V}
    \]
    using the fact that \(h(h^{-1}(E)) = E \cap h(X)\) for any map \(h : X \to Y\), on \(h = f\vert_{\cl{C}}\). 
\end{proof}

\begin{problem}{Characteristic Functions Without Continuous Representatives}*
    [S\&S E1.22.] Show that there is no continuous function \(f : \bb{R} \to \bb{R}\) such that \(f = \cl{X}_{[0,1]}\) almost everywhere.
\end{problem}

\begin{proof}
    Let \(\Delta(f, \cl{X}_{[0,1]}) = \{x \in \bb{R} \mid f(x) \neq \cl{X}_{[0,1]}(x)\}\). We will show that \(\Delta(f, \cl{X}_{[0,1]})\) has non-empty interior whenever \(f\) is continuous. Choose \(a \in (-1,0)\) and \(b \in [0,1]\) such that \(f(a) = \cl{X}_{[0,1]}(a)\) and \(f(b) = \cl{X}_{[0,1]}(b)\). If such \(a\) and \(b\) do not exist, we are done. By the Intermediate Value Theorem, there exists \(c \in (a, b)\) such that \(f(c) = 1/2\). But because \(f\) is continuous, for any \(0 < \varepsilon < 1/2\), there exists some \(\delta > 0\) such that \(|x - c| < \delta \implies |f(x) - f(c)| < \varepsilon \implies f(x) \in (0, 1)\). Thus, \((c - \delta, c + \delta) \subseteq \Delta(f, \cl{X}_{[0,1]})\), so the set of disagreements has nonzero Lebesgue measure. 
\end{proof}

\begin{problem}{Composition Fails to Preserve Lebesgue Measurability}*
    [S\&S E1.35.] Prove that if \(\varphi\) is continuous and \(f\) is Lebesgue-measurable, then \(\varphi \circ f\) is Lebesgue-measurable. However, show that the converse need not be true: find an example of where \(f \circ \varphi\) is non-measurable.\footnote{The nuance here is that the composition of \emph{Borel}-measurable functions is Borel!} Use the construction in the hint to show that there exists a Lebesgue-measurable set that is not a Borel set. 
    [Hint: Let \(\varphi : \hat{\cl{C}} \to \cl{C}\) where \(\hat{\cl{C}}\) is the fat Cantor set and \(\cl{C}\) is the Cantor set. Let \(N \subseteq \hat{\cl{C}}\) be non-measurable, and take \(f = \cl{X}_{\varphi(N)}\).]
\end{problem}

\begin{proof}
    TODO. 
\end{proof}

\begin{problem}{Lebesgue's Theorem}*
    [S\&S P1.4.] Let \(f\) be a \emph{bounded} function on a compact interval \(K\), and let \(B_r(c) \subseteq \bb{R}\) denote the open interval of radius \(r > 0\) centered at \(c \in \bb{R}\). Let 
    \[
        \oname{osc}_f(c) := \lim_{r \to 0} \left( \sup_{x, y \in K \cap B_r(c)}|f(x) - f(y)| \right)
    \]
    Note that \(f\) is continuous at \(c \in K\) if and only if \(\oname{osc}_f(c) = 0\). Prove the following assertions:
    \begin{enumerate}[(a)]
        \itemsep0em
        \item For every \(\varepsilon > 0\), the set of points \(c \in K\) such that \(\oname{osc}_f(c) \geq \varepsilon\) is compact. 
        \item If the set of discontinuities of \(f\) has measure zero, then \(f\) is Riemann integrable. [Hint: given \(\varepsilon > 0\) let \(A_\varepsilon = \{c \in K \mid \oname{osc}_f(c) \geq \varepsilon \}\). Select an appropriate partition of \(K\) and estimate the difference between the upper and lower sums of \(f\) over this partition.]
        \item Conversely, if \(f\) is Riemann integrable on \(K\), then its set of discontinuities has measure zero. [Hint: The set of discontinuities is contained in \(\bigcup_n A_{1/n}\). Choose a partition \(P\) such that \(U_f(P) - L_f(P) < \varepsilon / n\). Show that the total length of the intervals in \(P\) whose interior intersect \(A_{1/n}\) is at most \(\varepsilon\). 
    \end{enumerate}

    Conclude that a bounded function \(f : K \to \bb{R}\) on a compact interval is Riemann integrable if and only if its set of discontinuities has Lebesgue-measure zero. 
\end{problem}

\begin{proof}
    \emph{Part (a).} Since every closed subset of a compact space is compact, it suffices to prove that \(U_\varepsilon := \{c \in K \mid \oname{osc}_f(c) < \varepsilon\}\) is open. Take any \(c_0 \in U_\varepsilon\), and note there exists \(r > 0\) for which 
    \[
        \sup_{x, y \in K \cap B_r(c_0)} |f(x) - f(y)| < \varepsilon 
    \]
    Fix any \(c \in B_{r/2}(c_0)\), and observe that 
    \[
        \sup_{x, y \in K \cap B_{r/2}(c)} |f(x) - f(y)| 
        \leq \sup_{x, y \in K \cap B_r(c_0)} |f(x) - f(y)|
        < \varepsilon
    \]
    so that actually \(\oname{osc}_f(c) < \varepsilon\) as well. 
\end{proof}

\begin{proof}
    \emph{Part (b).} TODO. 
\end{proof}


\begin{proof}
    \emph{Part (c).} TODO. 
\end{proof}


\begin{problem}{Equivalence of the Riemann and Lebesgue Integrals}*
    [No Source.] Let \(f : [a, b] \to \bb{R}\) be a Riemann integrable function, i.e. bounded and continuous almost everywhere. Prove that: 
    \begin{enumerate}[(a)]
        \itemsep0em
        \item \(f\) is measurable with respect to \(\fr{L}([a, b])\) on its domain and \(\fr{B}(\bb{R})\) on its codomain. 
        \item \(f \in L^1([a, b])\), so \(f\) is Lebesgue integrable. 
        \item The (proper) Riemann and Lebesgue integrals agree, i.e. 
        \[
            \int_a^b f(x) \, \mr{d}x = \int_{[a, b]} f \, \mr{d}m
        \]
        \item If \(f : \bb{R} \to \bb{R} \in L^1(m)\) and \(f\) is Riemann integrable on any interval \([a, b]\), then 
        \[
            \int_{-\infty}^{\infty} f(x) \, \mr{d}x = \int_{\bb{R}} f \, \mr{d}m
        \]
    \end{enumerate}
\end{problem}

\begin{proof}
    TODO. 
\end{proof}


\begin{problem}{Axiom of Choice is Equivalent to Well-Ordering Principle}*
    [S\&S P1.6.] TODO. 
\end{problem}


\begin{problem}{The Vitali Set}*
    [MAT425 P4.8.] In this problem you will construct a bounded, non Lebesgue-measurable set.

    \begin{enumerate}[(a)]
        \itemsep0em 
        \item \textbf{The Equivalence Relation.}
        \begin{itemize}
            \item[(i)] Let \(\sim\) be the relation on \([0,1]\) defined by
            \[
                x \sim y \quad \text{if and only if} \quad x - y \in \bb{Q}.
            \]
            Prove that \(\sim\) is an equivalence relation on \([0,1]\).
            \item[(ii)] Show that each equivalence class is countable. [Hint: For any fixed \(x \in [0,1]\), the equivalence class of \(x\) is given by \(\{x + q : q \in \bb{Q} \} \cap [0,1]\), and \(\bb{Q}\) is countable.]
        \end{itemize}
        \item \textbf{Existence of a Vitali Set.} Using the Axiom of Choice, show that there exists a subset \(V \subset [0,1]\) (the Vitali set) such that \(V\) contains exactly one element from each equivalence class defined by \(\sim\). 
        \item \textbf{Translates of the Vitali Set}. For each rational number \(r \in \bb{Q} \cap [-1,1]\), define the translated set
        \[
            V_r = \{ v + r : v \in V \}.
        \]
        Prove that the sets \(\{V_r : r \in \bb{Q} \cap [-1,1]\}\) are pairwise disjoint. [Hint: Suppose that for \(r_1 \ne r_2\), there exist \(v_1, v_2 \in V\) such that \(v_1 + r_1 = v_2 + r_2\). Use the definition of the equivalence classes to reach a contradiction.] 
        \item \textbf{Covered by a Finite Interval.} Show that
        \[
            U := \bigcup_{r \in \bb{Q} \cap [-1,1]} V_r \subset [-1,2].
        \]
        [Hint: If \(v \in V \subset [0,1]\) and \(r \in [-1,1]\), then \(v + r \in [-1,2]\).] 
        \item \textbf{Non-measurability of the Vitali Set.} Assume, for the sake of contradiction, that \(V\) is Lebesgue measurable with measure \(m(V)\). Using translation invariance and countable additivity of the Lebesgue measure, show this assumption leads to contradiction.
        \begin{itemize}
            \item[(i)] Express \(m(U)\) in terms of \(m(V)\).
            \item[(ii)] Explain why this leads to a contradiction given that \(U\) is contained in the finite interval \([-1,2]\). [Hint: Consider the cases \(m(V) = 0\) and \(m(V) > 0\), and show that each case contradicts the finiteness of the measure of \([-1,2]\).]
        \end{itemize}
    \end{enumerate}
\end{problem}

\begin{proof}
    Note that \(\sim\) is an equivalence relation, since it is clearly reflexive, symmetric, and transitive. Let \(\{C_\alpha\}_\alpha\) be the equivalence classes, and note that each class can be written as \(C_\alpha = (x + \bb{Q}) \cap [0,1]\) for some \(x \in \bb{R}\), and hence each \(C_\alpha\) is countable. Using Axiom of Choice, choose an arbtirary \(x_\alpha \in C_\alpha\) and let \(V = \{x_\alpha\}_\alpha \subseteq [0,1]\). Suppose for contradiction that \(V\) is Lebesgue-measurable. 
    \stdvspace

    Observe that by construction, \(V+r\) is disjoint for each \(r \in \bb{Q} \cap [-1, 1]\). Define 
    \[
        U := \bigcup_{r \in \bb{Q} \cap [-1, 1]} (V + r) \subseteq V + [-1, 1] \subseteq [0,1] + [-1,1] \subseteq [-1, 2]
    \]
    and note that \(U \supseteq [0,1]\). Hence by countable additivity and translation-invariance, 
    \[
        m(U) = \sum_{r \in \bb{Q} \cap [-1,1]} m(V+r) = \sum_{r \in \bb{Q} \cap [-1,1]} m(V)
    \]
    Since \(m(U) \leq m([-1,2]) = 3 < \infty\), we must have that \(m(V) = 0\). But then \(m(U) = 0\), contradicting the fact that \(m(U) \geq m([0,1]) = 1\). 
\end{proof}

\begin{problem}{Banach-Tarski Paradox}*
    [MAT425 P4.9.] TODO. 
\end{problem}

\newpage
\section{Integration}

\begin{problem}{Absolute Continuity of the Integral}*
    [MAT425 P2.4.] Suppose that \(f \in L^1(X \to \bb{C}, \mu)\). For any \(\varepsilon > 0\), there exists some \(\delta > 0\) such that if \(\mu(E) \leq \delta\) then \(\int_E |f| \, \mr{d}\mu \leq \varepsilon\). 
\end{problem}

\begin{proof}
    Fix any \(\varepsilon > 0\). Let \(E_n := \{x \mid f(x) \leq n\}\) and \(f_n := \cl{X}_{E_n}f\). Then by the Monotone Convergence Theorem, for any set \(E \in \fr{M}\), we can \emph{fix} some \(N\) large enough such that
    \[
    \int_E |f_N - f| \, \mr{d}\mu \leq \frac{\varepsilon}{2}
    \]
    Define \(\delta := \varepsilon / 2N\). If \(\mu(E) \leq \delta\), then by the triangle inequality, 
    \[
    \int_E |f| \, \mr{d}\mu 
    \leq \int_E |f - f_N| \, \mr{d}\mu + \int_E |f_N| \, \mr{d}\mu
    \leq \frac{\varepsilon}{2} + N \cdot \frac{\varepsilon}{2N}
    \leq \varepsilon
    \]
    as desired. 
\end{proof}

\begin{problem}{Uniform Convergence for Limits of Integrals}*
    [RCA E1.10.] Suppose \(\mu(X) < \infty\) and \(\{f_n : X \to \bb{C}\}_n\) are bounded, and \(f_n \to f\) uniformly on \(X\). Then 
    \[
    \lim_{n \to \infty} \int_X f_n \, \mr{d}\mu = \int_X f \, \mr{d}\mu
    \]
    Furthermore, show that the result may fail if \(\mu(X) < \infty\) is not true. 
\end{problem}

\begin{proof}
    If \(f_n \to f\) uniformly, then for all \(\varepsilon > 0\) there exists \(N\) large enough such that \(\sup_{x \in X} |f_n(x) - f(x)| \leq \varepsilon / \mu(X)\) whenever \(n \geq N\), since \(X\) has finite measure. Therefore, for such \(n \geq N\), 
    \[
    \left| \int_X f_n \, \mr{d}\mu - \int_X f \, \mr{d}\mu \right| 
    \leq \int_X |f_n - f| \, \mr{d}\mu 
    \leq \mu(x) \sup_{x \in X} |f_n(x) - f(x)|
    \leq \mu(X) \cdot \frac{\varepsilon}{\mu(X)}
    = \varepsilon
    \]
    and thus we conclude. For a counterexample when \(\mu(X) = \infty\), take \(f = 0\) and \(f_n = \frac{1}{n} \cl{X}_{[0, n)}\) with the Lebesgue measure. Then \(\int_{\bb{R}} f_n \, \mr{d}m = 1 \neq 0 = \int_{\bb{R}} f \, \mr{d}m \) for all \(n\), even though \(f_n \to f\) uniformly. 
\end{proof}

\begin{problem}{Convergence in \(L^1\)}*
    [S\&S E2.2.] If \(f \in L^1(\bb{R}^d \to \bb{R}, m)\) and \(\delta > 0\), then \(f_\delta(x) := f(\delta x)\) converges to \(f\) in the \(L^1\) norm. In other words, 
    \[
    \lim_{\delta \to 1} \norm{ f - f_\delta }_{L^1} 
    \equiv \lim_{\delta \to 1} \int_{\bb{R}^d} |f - f_\delta| \, \mr{d}m = 0
    \]
\end{problem}

\begin{proof}
    Since the continuous functions of compact support are dense in \(L^1(\mu)\) [S\&S Thm 2.2.4], there exists a function \(g \in L^1(m)\) such that \(\norm{f - g}_{L^1} \leq \varepsilon / 4\). Write \(f - f_{\delta} = (f-g) + (g-g_{\delta}) + (g_{\delta} - f_{\delta})\). Then by the triangle inequality, 
    \[
    \norm{f - f_{\delta}}_{L^1} \leq \norm{f - g}_{L^1} + \norm{g - g_{\delta}}_{L^1} + \norm{g_{\delta} - f_{\delta}}_{L^1}
    \]
    Since \(g\) is continuous with compact support, it is uniformly continuous.\footnote{See this \href{https://math.stackexchange.com/questions/2086124/continuous-with-compact-support-implies-uniform-continuity}{Mathematics Stack Exchange} post.} Consider \(\delta > 0\) such that \(|x - \delta x| = |(1 - \delta)x| \leq \varepsilon / 4 m(\oname{supp}(g))\) for all \(x \in \oname{supp}(g)\), which exists by compactness. Then 
    \[
    \norm{g - g_{\delta}}_{L^1} 
    = \int_{\bb{R}^d} |g - g_{\delta}| \, \mr{d}m
    \leq \int_{\oname{supp}(g)} \frac{\varepsilon}{4 m(\oname{supp}(g))} \, \mr{d}m
    = \frac{\varepsilon}{4}
    \]
    since compact sets have finite measure thanks to Heine-Borel. Next, apply change of variables on \(g_{\delta} - f_{\delta}\) so that  
    \[
    \norm{g_{\delta} - f_{\delta}}_{L^1}
    = \int_{\bb{R}^d} |g_{\delta} - f_{\delta}| \, \mr{d}m
    = \delta^{-d} \int_{\bb{R}^d} |g - f| \, \mr{d}m
    = \delta^{-d} \norm{g - f}_{L^1}
    \leq \frac{\delta^{-d} \varepsilon}{4}
    \leq \frac{2\varepsilon}{4}
    < \frac{\varepsilon}{2}
    \]
    where \(\delta^{-d} \leq 2\) for \(\delta\) close enough to 1. Altogether, we have that 
    \[
        \norm{f - f_{\delta}}_{L^1} \leq \frac{\varepsilon}{4} + \frac{\varepsilon}{4} + \frac{\varepsilon}{2}
        = \varepsilon
    \]
    and thus \(f_\delta \stackrel{\delta \to 1}{\longrightarrow} f\) in the \(L^1\)-norm. 
\end{proof}

\begin{problem}{\(L^1\) Does Not Imply Tail Convergence}*
    [S\&S E2.6.] Integrability of \(f\) on \(\bb{R}\) does not necessarily imply \(f(x) \to 0\) as \(x \to \infty\). 
    \begin{enumerate}[(a)]
        \item There exists a positive continuous function \(f \in L^1(\bb{R} \to [0, \infty), m)\), and yet it tends to infinity for large \(x\): \(\limsup_{x \to \infty} f(x) = \infty\). 
        \item However, if we additionally assume \(f \in L^1(\bb{R} \to \bb{R}, m)\) is uniformly continuous, then actually \(\lim_{|x| \to \infty} f(x) = 0\). 
    \end{enumerate}
\end{problem}

\begin{proof}
    \emph{Part (a).} Look at the function \(f = \sum_{n} n \cl{X}_{[n, n+1/n^3)}\). Clearly it is positive, and it is \(L^1(m)\) since by the Monotone Convergence Theorem, 
    \[
        \int_{\bb{R}} f \, \mr{d}m
        = \int_{\bb{R}} \sum_{n=1}^\infty n \cl{X}_{[n, n + 1/n^3)} \, \mr{d}m
        = \sum_{n=1}^\infty \int_{\bb{R}} n \cl{X}_{[n, n + 1/n^3)} \, \mr{d}m
        = \sum_{n=1}^\infty \frac{1}{n^2}
        < \infty
    \]
    and yet \(\limsup_{x \to \infty} f(x) = \infty\). Note that while \(f\) itself is not continuous, we can replace the discontinuous points with steep line segments, of which the \(k^{\text{th}}\) segment will increase the area by \(2^{-k}\). This new function \(\tilde{f} \in L^1(m)\) still tends to infinity, but it is now continuous. 
\end{proof}

\begin{proof}
    \emph{Part (b).} Suppose \(f \in L^1(\bb{R} \to \bb{R}, m)\) but that \(\lim_{|x| \to \infty} f(x) \neq 0\). The latter means that either \(\lim_{x \to \infty} f(x) \neq 0\) or \(\lim_{x \to -\infty} f(x) \neq 0\), so without loss of generality suppose \(\lim_{x \to \infty} \neq 0\). Then fix any \(\varepsilon > 0\); there exists a sequence \(\{x_n\}_n\) such that \(|f(x_n)| \geq \varepsilon\) for all \(n\). But by uniform continuity, there exists a \(\delta > 0\) such that 
    \[
        |x - x_n| \leq \delta \implies |f(x) - f(x_n)| \leq \frac{\varepsilon}{2} \implies |f(x)| \geq \frac{\varepsilon}{2}
        \qquad (n \in \bb{N})
    \]
    It thus follows that 
    \[
    \int_{B_{\delta}(x_n)} |f| \, \mr{d}m \geq 2\delta \cdot \frac{\varepsilon}{2} = \delta \varepsilon
    \implies \int_{\bb{R}} |f| \, \mr{d}m \geq \sum_{n \geq 1} \delta \varepsilon = \infty
    \]
    a contradiction, and that completes the proof. 
\end{proof}

\begin{problem}{Function Graphs Are Measure Zero}*
    [S\&S E2.7.] Let \(\Gamma = \{(x, y) \in \bb{R}^d \times \bb{R} \mid y = f(x) \}\) for \(f : \bb{R}^d \to \bb{R}\) measurable. Then \(\Gamma\) is measurable in \(\bb{R}^{d+1}\) and \(m(\Gamma) = 0\).
\end{problem}

\begin{proof}
    To show that \(\Gamma\) is measurable, we exploit finiteness of \(f\). Denote \(g(x, y) = |y - f(x)|\) for \(g : \bb{R}^{d+1} \to \bb{R}\) which is measurable, so that
    \[
        \Gamma 
        = \bigcap_{n=1}^\infty \left\{ (x, y) \in \bb{R}^d \times \bb{R} \st |y - f(x)| < \frac{1}{n} \right\}
        = \bigcap_{n=1}^\infty g^{-1}\left(\left[0, \frac{1}{n}\right)\right)
    \]
    is measurable in \(\bb{R}^{d+1}\). Now, let \(m_d\) denote the Lebesgue measure on \(\bb{R}^d\). Integrating over sections, 
    \[
        (m_d \times m)(\Gamma)
        = \int_{\bb{R}^d} m(\Gamma_2(x)) \, \mr{d}m_d(x)
        = \int_{\bb{R}^d} m(\{f(x)\}) \, \mr{d}m_d(x)
        = 0
    \]
    as desired. 
\end{proof}

\begin{problem}{Space-Filling Curves}*
    [S\&S E7.8\footnote{Slightly modified, since Stein and Shakarchi's problem asks for a \emph{simple} (i.e. injective) curve, which relies on fractal patterns to construct.}.] There exists a continuous surjection from the unit interval to the unit square.
\end{problem}

\begin{proof}
    Take the Cantor function \(h : \cl{C} \to [0, 1]\), which is bicontinuous. Thus its product map \(H := h \times h : \cl{C} \times \cl{C} \to [0,1] \times [0,1]\) is bicontinuous as well, where \(H(x) = (h(x), h(x))\). Now we show that there exists a bicontinuous map \(g : \cl{C} \to \cl{C} \times \cl{C}\). Define
    \[
        g \left( \sum_{n=1}^\infty \frac{a_n}{3^n} \right) 
        = \left( \sum_{n=1}^\infty \frac{a_{2n-1}}{3^n}, \sum_{n=1}^\infty \frac{a_{2n}}{3^n} \right)
    \]
    where the representation is the standard \(\{a_n\}_n \subseteq \{0, 2\}\) form. Clearly \(g\) is a bijection. To show it is continuous, fix any \(\varepsilon > 0\). Choose \(k\) such that \(\sqrt{2} \cdot 3^{-k} < \varepsilon\). Let \(\delta = 2 \cdot 3^{-k}\) and note that if \(|x - y| < \delta\) for \(x,y \in \cl{C}\) then \(x\) and \(y\) have the same first \(k\) digits, and 
    \[
        |f(x) - f(y)|
        \leq \left| \left( \sum_{n > k} \frac{|a_{2n-1} - b_{2n-1}|}{3^n}, \sum_{n > k} \frac{|a_{2n} - b_{2n}|}{3^n} \right) \right| 
        \leq \sqrt{\left(\frac{1}{3^k}\right)^2 + \left(\frac{1}{3^k} \right)^2}
        = \sqrt{2} \cdot \frac{1}{3^k} 
        < \varepsilon
    \]
    Finally, since \(g\) is bicontinuous, \(f := H \circ g : \cl{C} \to [0, 1]^2\) is a bicontinuous map as well. But similarly to the Cantor function, we can extend \(f\) to a function \(F : [0, 1] \to [0, 1]^2\) as follows: for each deleted interval \((a, b)\) in a step of the Cantor construction, let \(F(\theta a + (1 - \theta)b) = \theta F(a) + (1 - \theta) F(b)\), where \(\theta \in (0, 1)\). First we show continuity at each \(x \not\in \cl{C}\). Take \(x, y \in (a, b)\) parametrized by \(\theta, \varphi\) respectively and note that if
    \[
        |x - y| = |(\theta - \varphi)a + (\varphi - \theta)b| 
        < \frac{|a - b|}{|f(a) - f(b)|} \varepsilon
        \implies |\theta - \varphi| < \frac{1}{|f(a) - f(b)|} \varepsilon
    \]
    then 
    \[
        |F(x) - F(y)|
        = |(\theta - \varphi) F(a) + (\varphi - \theta) F(b)|
        = |\theta - \varphi| |F(a) - F(b)|
        < \varepsilon
    \]
    so \(F\) is continuous on \([0,1] \setminus \cl{C}\). To show \(F\) is continuous on \(\cl{C}\), fix any \(x \in \cl{C}\) and \(\varepsilon > 0\). There exists \(\delta > 0\) such that whenever \(|c - x| < \delta\) for \(c \in \cl{C}\), then \(|f(x) - f(c)| < \varepsilon / 2\). Now, fix \(w \not\in \cl{C}\) such that \(|x - w| < \min(\delta, (|y - z| / 2|F(y) - F(z)|) \cdot \varepsilon)\), where \(y\) and \(z\) are the endpoints of the deleted interval containing \(w\), and \(y\) is the endpoint between \(x\) and \(w\). Then, 
    \begin{align*}
        |F(x) - F(w)|
        &< |F(x) - F(y)| + |F(y) - F(w)|
        < \frac{\varepsilon}{2} + \frac{|y - w|}{|y - z|} |F(y) - F(z)| \\
        &\leq \frac{\varepsilon}{2} + \frac{|x -w|}{|y - z|} |F(y) - F(z)|
        < \frac{\varepsilon}{2} + \frac{\varepsilon}{2} 
        = \varepsilon
    \end{align*}
    Therefore \(F\) is surjective and continuous (though not injective since \(f\) is already surjective), and that completes the proof. 
\end{proof}

\begin{problem}{Uniform Continuity of the Riemann Integral}*
    [S\&S E2.8.] If \(f \in L^1(\bb{R} \to \bb{R}, m)\), then \(F(x) = \int_{-\infty}^x f(t) \, \mr{d}t\) is uniformly continuous.
\end{problem}

\begin{proof}
    Fix any \(\varepsilon > 0\). Without loss of generality, let \(x \leq y\). Then
    \begin{align*}
        |F(x) - F(y)| 
        &= \left| \int_{-\infty}^x f(t) \, \mr{d}t - \int_{-\infty}^y f(t) \, \mr{d}t \right| 
        = \left| \lim_{N \to \infty} \int_{-N}^x f(t) \,\mr{d}t - \lim_{N \to \infty} \int_{-N}^y f(t) \, \mr{d}t \right|  \\
        &= \left| \int_x^y f(t) \, \mr{d}t \right| 
        = \left| \int_{[x, y]} f \, \mr{d} m \right| 
    \end{align*}
    thanks to the equality of the Riemann and Lebesgue integrals. However, by absolute continuity of the Lebesgue integral, there exists \(\delta\) such that if \(|x - y| \leq \delta\), then the integral is at most \(\varepsilon\).
\end{proof}

\begin{problem}{Markov's and Chebyshev's Inequalities}*
    [S\&S E2.9.\footnote{Generalized to arbitrary \(p\), rather than just \(p=1\).}] Suppose \(f : \Omega \to \ovl{\bb{R}}\) is measurable with respect to measure \(\lambda \geq 0\). Show that 
    \[
        \lambda(\{|f| \geq \alpha\}) \leq \frac{1}{\alpha^p} \int_{\Omega} |f|^p \, \mr{d}\lambda \qquad (0 < \alpha, p < \infty)
    \]
    Now suppose \(\Omega\) is a probability space, and let \(X\) be a positive random variable.
    \begin{enumerate}[(a)]
        \itemsep0em
        \item Prove \emph{Markov's Inequality:} if \(X\) also has finite expectation \(\mu\), 
        \[
            \bb{P}(X \geq k) \leq \frac{\mu}{k}
            \qquad (k > 0)
        \]
        \item Prove \emph{Chebyshev's Inequality:} if \(X\) also has finite expectation \(\mu\) and finite variance \(\sigma^2\), 
        \[
            \bb{P}(|X - \mu| \geq k \sigma) \leq \frac{1}{k^2}
            \qquad (k > 0)
        \]
    \end{enumerate}
\end{problem}

\begin{proof}
    Without loss of generality, allow \(f \geq 0\). Suppose first that \(p = 1\). The key observation is that \(f \geq \alpha \cl{X}_{\{f \geq \alpha\}}\), since then
    \[
        \lambda(\{f \geq \alpha\})
        = \int_{\Omega} \cl{X}_{\{f \geq \alpha\}} \, \mr{d}\lambda
        \leq \int_{\Omega} \frac{1}{\alpha} f \, \mr{d} \lambda
        = \frac{1}{\alpha} \int_{\Omega} f \, \mr{d} \lambda
    \]
    as desired. Then letting \(g = (f / \alpha)^p\) yields 
    \[
        \lambda(\{g \geq 1\}) \leq \int_{\Omega} g^p \, \mr{d}\lambda
        \implies 
        \lambda(\{f \geq \alpha\}) \leq \frac{1}{\alpha^p} \int_{\Omega} f^p \, \mr{d}\lambda
    \]
    Markov's inequality then follows by letting \(f = X\), \(\alpha = k\), and \(p = 1\). Chebyshev's inequality follows from letting \(f = X - \mu\), \(\alpha = k \sigma\), and \(p = 2\). 
\end{proof}

\begin{problem}{Layer-Cake Representation}*
    [MAT425 P4.6.] Let \(f \in L^1(\bb{R}^d)\). Show that 
    \[
        \int_{\bb{R}^d} |f| \, \mr{d}m = \int_0^\infty m(\{|f| > \alpha\}) \, \mr{d}\alpha
    \]
    where the expression on the right is an \emph{improper} Riemann integral. 
\end{problem}

\begin{proof}
    Without loss of generality, suppose \(f \geq 0\). Then, 
    \[
        \int_0^\infty \cl{X}_{\{f > \alpha\}}(x) \, \mr{d}\alpha
        = \int_0^\infty \cl{X}_{(0, f(x))}(\alpha) \, \mr{d}\alpha
        = \int_0^{f(x)} \, \mr{d}\alpha
        = f(x) 
    \]
    Since the improper Riemann integral converges almost everywhere (i.e. whenever \(f < \infty\)), it agrees with the Lebesgue integral, so by Fubini's Theorem we have 
    \begin{align*}
        \int_{\bb{R}^d} f(x) \, \mr{d}m(x)
        &= \int_{\bb{R}^d} \left(\int_0^\infty \cl{X}_{\{f > \alpha\}}(x) \, \mr{d}\alpha\right) \mr{d}m(x)
        = \int_{\bb{R}^d} \int_{(0, \infty)} \cl{X}_{\{f > \alpha\}} \, \mr{d}m(\alpha) \mr{d}m(x) \\
        &= \int_{(0, \infty)} \int_{\bb{R}^d} \cl{X}_{\{f > \alpha\}} \, \mr{d}m(\alpha) \mr{d}m(x)
        = \int_{(0, \infty)} m(\{f > \alpha\}) \, \mr{d}m(\alpha) \\
        &= \lim_{n \to \infty} \int_{(0, n)} m(\{f > \alpha\}) \, \mr{d}m(\alpha) 
        = \int_0^\infty m(\{f > \alpha\}) \, \mr{d}\alpha
    \end{align*}
    where the penultimate step follows from approximation on \(\mr{d}\mu = m(\{f > \alpha\}) \mr{d}m(\alpha)\), and the final step from equality of the Lebesgue and Riemann integrals on bounded intervals. 
\end{proof}


\begin{problem}{Convergence in \(L^1\) but Pointwise Nowhere}*
    [S\&S E2.12.] Show that there exists \(f, f_n \in L^1(\bb{R}^d \to \bb{R}, m)\) such that \(\norm{f - f_n}_{L^1} \to 0\) as \(n \to \infty\), but \(f_n(x) \to f(x)\) for no \(x \in \bb{R}^d\) at all. [Hint: in \(\bb{R}\), let \(f_n = \cl{X}_{I_n}\), where \(I_n\) is an appropriately chosen sequence of intervals with \(m(I_n) \to 0\).]
\end{problem}

\begin{proof}
    TODO. 
\end{proof}


\begin{problem}{High-Dimensional Volumes}*
    [S\&S E2.14.] Prove that \(m(B) = v_d r^d\), whenever \(B\) is a ball of radius \(r\) in \(\bb{R}^d\) and \(v_d = m(B_1)\), with \(B_1\) the unit ball. Then, evaluate the constant \(v_d\): 
    \begin{enumerate}[(a)]
        \itemsep0em
        \item For \(d = 2\), prove that 
        \[
            v_2 = 2 \int_{-1}^1 (1 - x^2)^{1/2} \, \mr{d}x
        \]
        and hence that \(v_2 = \pi\). 
        \item By similar methods, show that 
        \[
            v_d = 2v_{d-1} \int_0^1 (1 - x^2)^{(d-1)/2} \, \mr{d}x
        \]
        \item The result is 
        \[
            v_d = \frac{\pi^{d/2}}{\Gamma(d/2 + 1)}
        \]
    \end{enumerate}
\end{problem}

\begin{proof}
    TODO. 
\end{proof}


\begin{problem}{Convolutions}*
    [S\&S E2.21.] Suppose that \( f \) and \( g \) are measurable functions on \( \mathbb{R}^d \).
    \begin{enumerate}[(a)]
        \itemsep0em
        \item Prove that \( f(x - y)g(y) \) is measurable on \( \mathbb{R}^{2d} \).
        \item Show that if \( f \) and \( g \) are integrable on \( \mathbb{R}^d \), then \( f(x - y)g(y) \) is integrable on \( \mathbb{R}^{2d} \).
        \item Recall the definition of the convolution of \( f \) and \( g \) given by
        \[
            (f * g)(x) = \int_{\mathbb{R}^d} f(x - y)g(y) \, \mr{d}y
        \]
        Show that \( f * g \) is well defined for a.e. \( x \) (that is, \( f(x - y)g(y) \) is integrable on \( \mathbb{R}^d \) for a.e. \( x \)).
        \item Show that \( f * g \) is integrable whenever \( f \) and \( g \) are integrable, and that
        \[
            \norm{f*g}_1 \leq \norm{f}_1 \norm{g}_1,
        \]
        with equality if \( f \) and \( g \) are non-negative.
        \item The Fourier transform of an integrable function \( f \) is defined by
        \[
            \hat{f}(\xi) = \int_{\mathbb{R}^d} f(x)e^{-2\pi i x \cdot \xi} \, \mr{d}x
        \]
        Check that \( \hat{f} \) is bounded and is a continuous function of \( \xi \). Prove that for each \( \xi \),
        \[
            \widehat{(f * g)}(\xi) = \hat{f}(\xi)\hat{g}(\xi)
        \]
    \end{enumerate}
\end{problem}

\begin{proof}
    TODO. 
\end{proof}


\begin{problem}{The Riemann-Lebesgue Lemma}*
    [S\&S E2.22.] Prove that if \(f \in L^1(\mathbb{R}^d)\) and 
    \[
        \hat{f}(\xi) = \int_{\bb{R}^d} f(x) e^{-2\pi i x \xi} \, \mr{d}x
    \]
    then \(\hat{f}(\xi) \to 0\) as \(|\xi| \to 0\). [Hint: Write \(\hat{f}(\xi) = \frac{1}{2} \int_{\bb{R}^d} [f(x) - f(x -\xi')] e^{-2\pi ix \xi} \, \mr{d}x\), where \(\xi' = \frac{1}{2} \frac{\xi}{|\xi|^2}\), and use the fact that \(f_h := f \circ (x \mapsto x -h)\) satisfies \(\norm{f_h - f}_1 \to 0\) as \(h \to 0\).]
\end{problem}

\begin{proof}
    TODO. 
\end{proof}



\begin{problem}{Uniform Continuity of the Convolution}*
    [S\&S E2.24.] Consider the convolution 
    \[
        (f*g)(x) = \int_{\bb{R}^d} f(x - y) g(y) \, \mr{d}y
    \]
    \begin{enumerate}[(a)]
        \itemsep0em
        \item Show that \(f*g\) is uniformly continuous when \(f\) is integrable and \(g\) is bounded. 
        \item If in addition \(g\) is integrable, prove that \((f*g)(x) \to 0\) as \(|x| \to \infty\). 
    \end{enumerate}
\end{problem}


\newpage
\section{\(L^p\) Spaces}


\begin{problem}{Preserving Convexity in Composition}*
    [RCA E2.2] If \(\varphi : (a, b) \to \bb{R}\) is convex and if \(\psi\) is convex and nondecreasing on the range of \(\varphi\), prove that \(\psi \circ \varphi : (a, b) \to \bb{R}\) is convex. For \(\varphi > 0\), show that the convexity of \(\log\varphi\) implies the convexity of \(\varphi\), but not vice versa. 
\end{problem}

\begin{proof}
    That \(\psi \circ \varphi\) is convex follows immediately from the definitions. If \(\varphi > 0\), convexity of \(\log\varphi\) means that \(\varphi = \exp \log \varphi\) is convex as well, since \(\exp\) is convex and increasing. Of course, the converse doesn't hold since \(x \mapsto x\) is convex but \(x \mapsto \log x\) is not.
\end{proof}


\begin{problem}{Midpoint Convexity}*
    [RCA E2.3.] Assume that \(\varphi: (a, b) \to \bb{R}\) is continuous and 
    \[
        \varphi \left(\frac{1}{2}x + \frac{1}{2}y\right) \leq \frac{1}{2} \varphi(x) + \frac{1}{2}\varphi(y)
        \qquad (x, y \in (a, b))
    \]
    Prove that \(\varphi\) is convex and that continuity is a required assumption. 
\end{problem}

\begin{proof}
    Let a function \(f\) be \(\Theta\)-\emph{convex} if 
    \[
        f(\theta x + (1 - \theta) y) \leq \theta f(x) + (1 - \theta) f(y) 
        \qquad (\theta \in \Theta)
    \]
    Let \(B_n = \{k / 2^n \mid k \in \bb{N}, 0 < k < 2^n\}\) for each \(n \in \bb{N}\). We will prove by induction that \(\varphi\) is \(B_n\)-convex. For \(n=1\), the hypothesis is true by assumption. So suppose \(\varphi\) is \(B_n\) convex; we will show it is \(B_{n+1}\)-convex. Fix any integer \(k\) such that \(0 < k < 2^n\). If \(k\) is even, then \(k/2^{n+1} \in B_n\), so suppose \(k\) is odd. Then, 
    \begin{align*}
        \varphi \left( \frac{k}{2^{n+1}} x + \left(1 - \frac{k}{2^{n+1}}\right) y \right)
        &= \varphi \left( \frac{1}{2} \left( \frac{k}{2^n} x + \left(1 - \frac{k}{2^n}\right) y \right) + \frac{1}{2} y \right) \\
        &\leq \frac{1}{2} \left( \frac{k}{2^n} x + \left( 1 - \frac{k}{2^n} \right) y \right) + \frac{1}{2} \varphi(y) \\
        &\leq \frac{k}{2^{n+1}} \varphi(x) + \left(1 - \frac{k}{2^{n+1}} \right) \varphi(y)
    \end{align*}
    so that \(\varphi\) is \(B_{n+1}\)-convex, completing the induction. Now, let \(B\) be the set of all \(b \in (0, 1)\) which have a binary representation with at most finitely many nonzero digits. Note that every \(b \in B\) satisfies \(b \in B_n\) for some \(n\), so \(\varphi\) is \(B\)-convex. So, take any \(\theta \in (0, 1)\), and consider a binary representation \(\theta = \sum_i b_i / 2^i\). If \(\theta_N = \sum_{i=1}^N b_i / 2^i\), then since \(\theta_N \in B\), 
    \[
        \varphi(\theta_N x + (1 - \theta_N) y) \leq \theta_N \varphi(x) + (1 - \theta_N) \varphi(y)
        \qquad (N \in \bb{N})
    \]
    Since \(\varphi\) is continuous, taking \(N \to \infty\) on both sides shows that \(\varphi\) is convex. Finally, continuity is a required assumption: if \(\varphi\) were not continuous, then it could not be convex, since every convex function is continuous. 
\end{proof}


\begin{problem}{Finite \(L^p\)-Norm Implies \(\sigma\)-Finite Support}*
    [\href{https://terrytao.wordpress.com/2009/01/09/245b-notes-3-lp-spaces/}{Tao} E3.3] Suppose \((X, \fr{M}, \mu)\) is a positive measure space and let \(f\) be a complex measurable function. If \(\norm{f}_p < \infty\) for some \(0 < p < \infty\), then \(f\) has \(\sigma\)-finite support. 
\end{problem}

\begin{proof}
    Suppose for contradiction that \(f\) did not have \(\sigma\)-finite support. Note that 
    \[
        \oname{supp}(f) = \{x : f(x) > 0 \} = \bigcup_{n=1}^\infty \left\{ x : f(x) > \frac{1}{n} \right\}
    \]
    So, there exists some \(m \in \bb{N}\) for which \(\mu(\{x : f(x) > 1/m\}) = \infty\). But then
    \[
        \norm{f}_p = \left( \int_X |f|^p \, \mr{d}\mu \right)^{\frac{1}{p}}
        > \left( \int_{\{x : f(x) > 1/m\}} \left(\frac{1}{m}\right)^p \, \mr{d}\mu \right)^{\frac{1}{p}}
        = \infty
    \]
    a contradiction, so \(f\) has \(\sigma\)-finite support. 
\end{proof}


\begin{problem}{Monotonicity of the \(L^p\)-norm}*
    [RCA E2.4.] Suppose \((X, \fr{M}, \mu)\) is a positive measure space and let \(f\) be a complex measurable function. Define 
    \[
        \varphi(p) := \int_X |f|^p \, \mr{d}\mu = \norm{f}_p^p
        \qquad (0 < p < \infty) 
    \]
    and \(E := \{ p \mid \varphi(p) < \infty\}\). Then, prove the following: 
    \begin{enumerate}[(a)]
        \itemsep0em
        \item \(E\) is connected, in the sense that if \(r, s \in E\) then \((r, s) \subseteq E\). 
        \item \(\log \varphi\) is convex on \(E^\circ\) and \(\varphi\) is continuous on \(E\). 
        \item \(E\) could be any singleton or interval subset of \((0, \infty)\). 
        \item \(\norm{f}_p\) is monotonic in \(p\) for fixed \(f\), but the monotonicity may be increasing or decreasing in \(p\). [Hint: show that if \(r < p < s\), then \(\norm{f}_p \leq \max(\norm{f}_r, \norm{f}_s)\).]
        \item If \(\norm{f}_r < \infty\) for some \(0 < r < \infty\), then \(\norm{f}_p \to \norm{f}_\infty\) as \(p \to \infty\). 
    \end{enumerate}
\end{problem}

\begin{proof}
    \emph{Part (a).} Let \(r, s \in E\) and take \(p \in (r, s)\). Define \(A := \{ x \mid |f(x)| \leq 1\}\). Then, 
    \[
        \int_X |f|^p \, \mr{d} \mu 
        = \int_{A} |f|^p \, \mr{d}\mu + \int_{A^c} |f|^p \, \mr{d}\mu
        \leq \int_A |f|^r \, \mr{d}\mu + \int_{A^c} |f|^s \, \mr{d}\mu
        \leq \varphi(r) + \varphi(s) < \infty
    \]
    so that \(p \in E\) as well. 
\end{proof}

\begin{proof}
    \emph{Part (b).} For any \(r, s \in E^\circ\), applying H\"older's inequality on \(|f|\) using the conjugate exponents \((\theta^{-1}, (1 - \theta)^{-1})\) yields
    \begin{align*}
        \log \varphi(\theta r + (1 - \theta) x) 
        &= \log \left( \int_X |f|^{\theta r} |f|^{(1 - \theta) s} \, \mr{d}\mu \right)
        \leq \log \left[ \left( \int_X |f|^r \, \mr{d}\mu \right)^\theta \left( \int_X |f|^s \, \mr{d}\mu \right)^{1 - \theta} \right] \\
        &= \theta \log \varphi(r) + (1 - \theta) \log \varphi(s)
    \end{align*}
    Note that we must look at \(E^\circ\) rather than \(E\) since convexity (according to RCA) is defined on open intervals. Now, to show that \(\varphi\) is continuous on \(E\), first note that \(\varphi\) is convex on \(E^\circ\) since \(\varphi = \exp \log \varphi\). Then consider any \(p \in \partial E\). Any sequence \(\{p_n\}_n\) which approaches \(p\) must do so either from above or below, since \(E\) is an interval. Define \(A := \{ x \mid |f(x)| \leq 1 \}\) so that 
    \begin{align*}
        \lim_{n \to \infty} \varphi(p_n) 
        &= \lim_{n \to \infty} |f|^{p_n} \, \mr{d}\mu 
        = \lim_{n \to \infty} \int_A |f|^{p_n} \, \mr{d}\mu + \lim_{n \to \infty} \int_{A^c} |f|^{p_n} \, \mr{d}\mu \\
        &\stackrel{(*)}{=} \int_A |f|^p \, \mr{d}\mu + \int_{A^c} |f|^p \, \mr{d}\mu
        = \int_X |f|^p \, \mr{d}\mu
        = \varphi(p)
    \end{align*}
    The justification for \((*)\) follows from applications of the Dominated Convergence Theorem. If \(p_n\) approaches \(p\) from below, then the first term in \((*)\) uses domination by \(|f|^{p_0}\), where \(p_0 := \inf p_n \in E\), and the second uses domination by \(|f|^p\). On the other hand, if \(p_n\) approaches \(p\) from above, then the first term uses domination by \(|f|^p\), and the second term by \(|f|^{p_0}\) where \(p_0 := \sup p_n\) this time. Hence \(\varphi\) is continuous at both the interior and boundary points of \(E\), so it is continuous on \(E\). 
\end{proof}

\begin{proof}
    \emph{Part (c).} First, we show that \(E\) could be any singleton \(0 < p_0 < \infty\). Define 
    \begin{align*}
        & a_n = 2^{-n}, \quad \mu(\{A_n\}) = n^{-2} 2^{np_0}  \\
        & b_n = 2^{-n}, \quad \mu(\{B_n\}) = n^{-2} 2^{-np_0}
    \end{align*}
    where the points \(\{A_n\}_n, \{B_n\}_n\) form a measure space \(X = \left(\bigcup_n A_n\right) \cup \left(\bigcup_n B_n\right)\) whose \(\sigma\)-algebra \(\fr{M}\) is the power set \(\cl{P}(X)\). Define \(f(A_n) = a_n\) and \(f(B_n) = b_n\) so that \(f\) is a limit of simple functions and therefore is measurable. Observe, 
    \[
        \varphi(p) = \int_X f^p \, \mr{d}\mu = \sum_{n=1}^\infty f(A_n) \mu(\{A_n\}) + \sum_{n=1}^\infty f(B_n) \mu(\{B_n\})
        = \sum_{n=1}^\infty \frac{1}{n^2} \left( 2^{n(p_0 - p)} + 2^{n(p - p_0)} \right)
    \]
    Of course, this series converges only for \(p = p_0\), so \(E = \{p_0\}\) as desired. 
    \stdvspace

    Now we show that \(E\) can be any open interval \((\alpha, \beta)\) where \(0 \leq \alpha < \beta \leq \infty\). Define
    \[
        f_\beta = \left(x \mapsto x^{-\frac{1}{\beta}}\right) : (0,1) \to \bb{R}, 
        \quad 
        f_\alpha = \left(x \mapsto x^{-\frac{1}{\alpha}}\right) : (1,\infty) \to \bb{R}
    \]
    where \(f_\beta = 1\) whenever \(\beta = \infty\) and \(f_\alpha = 0\) whenever \(\alpha = 0\). Observe that \(\norm{f_\beta}_p\) is finite on \((0,1)\) precisely when \(p < \beta\), and \(\norm{f_\alpha}_p\) is finite on \((1, \infty)\) precisely when \(p > \alpha\). Hence define 
   \[
        f = f_\beta \cl{X}_{(0,1)} + f_\alpha \cl{X}_{(1, \infty)} : \bb{R} \to \bb{R}
    \]
    so that \(\norm{f}_p\) is finite on \(\bb{R}\) precisely when \(\alpha < p < \beta\). So, \(E = (\alpha, \beta)\) for \(f\). 
    \stdvspace

    Next, we show that \(E = [\alpha, \beta)\), \(E = (\alpha, \beta]\), or \(E = [\alpha, \beta]\) can be constructed as well. TODO. See the linked \href{https://math.stackexchange.com/questions/3186292/rudins-real-and-complex-analysis-exercise-3-4-c}{Math Stack Exchange post.} 
\end{proof}

\begin{proof}
    \emph{Part (d).} Without loss of generality, assume \(\norm{f}_r\) and \(\norm{f}_s\) are finite. First, consider the case where \(r, s \in E^\circ\). Let \(p = \theta r + (1 - \theta) s\) for some \(\theta \in (0, 1)\). Since \(\log\varphi\) is convex, we have 
    \begin{align*}
        \log\varphi(p) \leq \theta \log\varphi(r) + (1 - \theta) \log\varphi(s) 
        \implies & \norm{f}_p^p \leq (\norm{f}_r^r)^\theta (\norm{f}_s^s)^{1 - \theta} 
        \leq \max\{\norm{f}_r, \norm{f}_s\}^p 
    \end{align*}
    from which it follows that \(\norm{f}_p \leq \max\{ \norm{f}_r, \norm{f}_s \}\). Now suppose either \(r\) or \(s\) lies on the boundary \(\partial E\). Then observe that since \(\psi(p) := \norm{f}_p = \varphi(p)^p = \exp \left( \frac{1}{p} \log \varphi(p) \right)\) is the composition of continuous maps (whenever \(\varphi > 0\), which occurs insofar as \(\mu(X) > 0\) and \(f\) is not zero a.e., which are both cases where the desired result is trivially true) and therefore is continuous, we have that
    \begin{align*}
        & \norm{f}_p \leq \lim_{\varepsilon \to 0} \max( \norm{f}_{r + \varepsilon}, \norm{f}_s ) = \max( \norm{f}_r, \norm{f}_s) 
        \quad \text{if } r \in \partial E, s \in E^\circ \\
        & \norm{f}_p \leq \lim_{\varepsilon \to 0} \max( \norm{f}_r, \norm{f}_{s + \varepsilon} ) = \max( \norm{f}_r, \norm{f}_s) 
        \quad \text{if } s \in \partial E, r \in E^\circ
    \end{align*}
    or we can even take both limits if \(r, s \in \partial E\). Hence \(\norm{f}_p\) is monotonic (either weakly increasing or decreasing) in \(p\) if we fix a function \(f\). However, either direction is possible, even within the same measure space. For instance, when \((X, \fr{M}, \mu) = (\bb{R}, \fr{L}(\bb{R}), m)\) where \(\fr{L}(\bb{R})\) is the collection of Lebesgue-measurable sets on \(\bb{R}\), the function \(f=\cl{X}_{[0, 1/4)}\) satisfies
    \[
        \norm{f}_1 = \int_\bb{R} \cl{X}_{\left[0, \frac{1}{4}\right)} \, \mr{d}m = \frac{1}{4},
        \quad 
        \norm{f}_2 = \sqrt{\int_\bb{R} \cl{X}_{\left[0, \frac{1}{4}\right)} \, \mr{d}m} = \frac{1}{2}
    \]
    and so \(\norm{f}_1 < \norm{f}_2\). Yet if we choose instead \(f = \cl{X}_{[0, 4)}\), 
    \[
        \norm{f}_1 = \int_{\bb{R}} \cl{X}_{[0, 4)} \, \mr{d}m = 4,
        \quad
        \norm{f}_2 = \sqrt{\int_{\bb{R}} \cl{X}_{[0, 4)} \, \mr{d}m} = 2
    \]
    so that \(\norm{f}_1 > \norm{f}_2\). 
\end{proof}

\begin{proof}
    \emph{Part (e).} We consider two cases. First, suppose \(\norm{f}_\infty = \infty\). Fix \(n \in \bb{N}\) greater than \(1\), and observe that we can find a set \(K\) with measure \(\varepsilon\) on which \(f > n\). Hence
    \[
        \varphi(p) = \int_X |f|^p \, \mr{d}\mu > \int_K |f|^p \, \mr{d}\mu > n^p \mu(K) = n^p \varepsilon
    \]
    Of course, as \(p \to \infty\) we thus have \(\varphi(p) \to \infty\) and so \(\norm{f}_p \to \infty\) as well. 
    \stdvspace

    Second, suppose \(\norm{f}_\infty < \infty\). We will show inequality in both directions. To show \(\norm{f}_\infty\) is an upper bound, 
    \[
        \lim_{p \to \infty} \norm{f}_p = \lim_{p \to \infty} \left( \int_X |f|^r |f|^{p-r} \, \mr{d}\mu \right)^{\frac{1}{p}} 
        \leq \lim_{p \to \infty} (\norm{f}_\infty)^{1 - \frac{r}{p}} (\norm{f}_r)^{\frac{r}{p}}
        = \norm{f}_\infty
    \]
    Now to show \(\norm{f}_\infty\) is a lower bound. Since \(\norm{f}_r < \infty\), we know that \(\oname{supp}(f)\) is \(\sigma\)-finite. Let \(\{K_n\}_n\) denote a partition of \(\oname{supp}(f)\) with \(\mu(K_n) < \infty\), and furthermore write 
    \[
        A_{\varepsilon, n} = \{x \in K_n \mid f(x) > \norm{f}_\infty - \varepsilon\}
        \qquad (\varepsilon > 0)
    \]
    Note that \(0 < \mu(A_{\varepsilon, n}) < \infty\), so 
    \[
        \lim_{p \to \infty} \norm{f}_p 
        \geq \lim_{p \to \infty} \left( \sum_{n=1}^\infty \int_{K_n} |f|^p \, \mr{d}\mu \right)^{\frac{1}{p}} 
        \geq \lim_{p \to \infty} \left( (\norm{f}_\infty - \varepsilon) \left( \sum_{n=1}^\infty \mu(A_{\varepsilon, n}) \right)^{\frac{1}{p}} \right)
        = \norm{f}_\infty - \varepsilon
    \]
    The last step follows since \(\sum_n \mu(A_{\varepsilon, n})\) must converge since otherwise \(\lim \norm{f}_p\) would be infinite, contradicting \(\lim \norm{f}_p \leq \norm{f}_\infty < \infty\). Finally, taking \(\varepsilon \to 0\) yields \(\lim \norm{f}_p \geq \norm{f}_\infty\), as desired. 
\end{proof}

\begin{problem}{Montonicity of \(L^p\)}*
    [RCA E3.7.] For some measures, the relation \(r < s\) implies \(L^r(\mu) \subseteq L^s(\mu)\); for others, the inclusion is reversed; and there are some for which \(L^r(\mu)\) does not contain \(L^s(\mu)\) if \(r \neq s\). Give examples of these situations, and find conditions on \(\mu\) under which each will occur. 
\end{problem}

\begin{proof}
    TODO. 
\end{proof}


\begin{problem}{Hardy's Inequality}*
    [RCA E3.14, E3.15.] Suppose \(1 < p < \infty\), \(f \in L^p((0, \infty) \to \bb{C}, m)\), and 
    \[
        F(x) = \frac{1}{x} \int_0^x f(t) \, \mr{d}t
        \qquad
        (0 < x < \infty)
    \]
    \begin{enumerate}[(a)]
        \itemsep0em
        \item Prove Hardy's inequality 
        \[
            \norm{F}_p \leq \frac{p}{p-1} \norm{f}_p
        \]
        which shows that the mapping \(f \mapsto F\) carries \(L^p\) into \(L^p\). [Hint: Assume first that \(f \geq 0, f \in C_c((0,\infty))\). Integration by parts gives \(\int_0^\infty F^p(x) \, \mr{d}x = -p \int_0^\infty F^{p-1}(x) x F'(x) \, \mr{d}x\). Note that \(xF' = f - F\), and apply H\"older's inequality to \(\int F^{p-1} f\). Then derive the general case.]
        \item Prove that equality holds if and only if \(f = 0\) a.e. 
        \item Prove that the constant \(p/(p-1)\) cannot be replaced by a smaller one. [Hint: Take \(f(x) = x^{-1/p} \cl{X}_{[1, A]}\) for large \(A\). Use the fact that if each section \(f_x\) is Borel and each \(f_y\) is continuous, then \(f\) is Borel on \(\bb{R}^2\).]
        \item If \(f > 0\) and \(f \in L^1((0, \infty) \to \bb{R}, m)\), prove that \(F \not\in L^1\). 
        \item Suppose \(\{a_n\}_n\) is a sequence of positive numbers. prove that 
        \[
            \sum_{N=1}^\infty \left( \frac{1}{N} \sum_{n=1}^N a_n \right)^p
            \leq \left( \frac{p}{p-1} \right)^p \sum_{n=1}^\infty a_n^p
        \]
        if \(1 < p < \infty\). [Hint: If \(a_n \geq a_{n+1}\), thte result can be made to follow from Hardy's inequality. This special case implies the general one.]
    \end{enumerate}
\end{problem}


\begin{proof}
    TODO. 
\end{proof}


\begin{problem}{Egorov's Theorem}*
    [RCA E3.16.] Prove \emph{Egorov's Theorem}, that if \(\mu(X) < \infty\), if \(\{f_n : X \to \bb{C}\}_n\) converges pointwise everywhere, and if \(\varepsilon > 0\), then there exists a measurable \(E \subseteq X\) such that \(\mu(X \setminus E) < \varepsilon\) and \(\{f_n\}_n\) converges uniformly on \(E\), using the following outline: 
    \begin{enumerate}[(a)]
        \itemsep0em
        \item Put \(S(n, k) = \bigcap_{i,j > n} \{|f_i - f_j| < 1/k\}\). Prove \(\mu(S(n, i)) \to \mu(X)\) as \(n \to \infty\) for each \(k\).
        \item Hence there is a suitably increasing sequence \(n_k\) such that \(E = \bigcap_k S(n_k, k)\) has the desired property.
    \end{enumerate}
    The conclusion is that by redefining the \(f_n\) on a set of arbitrarily small measure we can convert a pointwise convergent sequence to a uniformly convergent one. Furthermore, 
    \begin{enumerate}[(a)]
        \itemsep0em
        \item Show that the theorem does not extend to \(\sigma\)-finite spaces. 
        \item Show that the theorem does extend, with essentially the same proof, to thte situation in which the sequence \(\{f_n\}_n\) is replaced by a family \(\{f_t\}_{t \in \bb{R}}\). The assumptions now are that for each \(x \in X\), \(\lim_{t \to \infty} f_t(x) = f(x)\) and \(t \mapsto f_t(x)\) is continuous. 
    \end{enumerate}
\end{problem}

\begin{proof}
    TODO. 
\end{proof}


\begin{problem}{Convergence in Measure}*
    [RCA E3.18.] Let \(\mu\) be a positive measure on \(X\). A sequence \(\{f_n : X \to \bb{C}\}_n\) of measurable functions is said to \emph{converge in measure} to the measurable function \(f\) if to every \(\varepsilon > 0\)
    \[
        \lim_{n \to \infty} \mu( \{ x \mid |f_n(x) - f(x)| > \varepsilon \} ) = 0
    \]
    Assume \(\mu(X) < \infty\) and prove the following: 
    \begin{enumerate}[(a)]
        \itemsep0em
        \item If \(f_n(x) \to f(x)\) a.e., then \(f_n \to f\) in measure. 
        \item If \(f_n \in L^p(\mu)\) and \(\norm{f_n - f}_p \to 0\), then \(f_n \to f\) in measure; here \(1 \leq p \leq \infty\). 
        \item If \(f_n \to f\) in measure, then \(\{f_n\}_n\) has a subsequence which converges to \(f\) a.e. 
    \end{enumerate}

    Investigate the converses of (a) and (b). What happens to (a), (b), and (c) if \(\mu(X) = \infty\), for instance if \(\mu\) is the Lebesgue measure on \(\bb{R}\)? 
\end{problem}

\begin{proof}
    TODO. 
\end{proof}


\newpage
\section{Differentiation}

\begin{problem}{Chain Rule for the Radon-Nikodym Derivative}*
    [MAT425 P6.2.] Let \((X, \fr{M})\) be a measurable space. Let \(\mu : \fr{M} \to [0, \infty]\) be \(\sigma\)-finite, \(\nu : \fr{M} \to [0, \infty)\) and \(\eta : \fr{M} \to \bb{C}\) be measures such that \(\eta \ll \nu \ll \mu\). Prove that \(\eta \ll \mu\) and that 
    \[
        \frac{\mr{d}\eta}{\mr{d}\mu} = \frac{\mr{d}\eta}{\mr{d}\nu} \frac{\mr{d}\nu}{\mr{d}\mu}
    \]
\end{problem}

\begin{proof}
    To show \(\eta \ll \mu\), note that \(\mu(E) = 0 \implies \nu(E) = 0 \implies \eta(E) = 0\) for any \(E \in \fr{M}\). The key observation is that for any positive \(g \in L^1(\nu)\), 
    \[
        \int_X g \, \mr{d}\nu = \int_X g \frac{\mr{d}\nu}{\mr{d}\mu} \, \mr{d}\mu
    \]
    Indeed, the equality is certainly true for characteristics and thus simple functions, so taking an increasing sequence of simple functions \(s_n \to g\) for any positive \(g \in L^1(\nu)\) we find it must hold for such \(g\) as well. But since \(\mr{d}\eta / \mr{d}\nu \in L^1(\nu)\) is positive, we have 
    \[
        \eta(E) = \int_E \frac{\mr{d}\eta}{\mr{d}\nu} \, \mr{d}\nu = \int_E \frac{\mr{d}\eta}{\mr{d}\nu} \frac{\mr{d}\nu}{\mr{d}\mu} \, \mr{d}\mu
    \]
    as well, and so uniqueness of the Radon-Nikodym derivative concludes the proof. 
\end{proof}

\begin{problem}{Inverse of the Radon-Nikodym Derivative}*
    [MAT425 P6.3.] If \(\mu \ll \nu\) and \(\nu \ll \mu\) are positive finite measures then 
    \[
        \frac{\mr{d}\mu}{\mr{d}\nu} = \left( \frac{\mr{d}\nu}{\mr{d}\mu} \right)^{-1}
    \]
\end{problem}

\begin{proof}
    From the Chain Rule, we have that since \(\mu \ll \nu \ll \mu\), 
    \[
        \frac{\mr{d}\mu}{\mr{d}\nu}\frac{\mr{d}\nu}{\mr{d}\mu} = \frac{\mr{d}\mu}{\mr{d}\mu} = 1
        \implies 
        \frac{\mr{d}\mu}{\mr{d}\nu} = \left( \frac{\mr{d}\nu}{\mr{d}\mu} \right)^{-1} 
    \]
    again by uniqueness of the Radon-Nikodym derivative. 
\end{proof}

\begin{problem}{Linearity of the Radon-Nikodym Derivative}*
    [MAT425 P6.4.] Let \((X, \fr{M})\) be a measurable space. Let \(\mu : \fr{M} \to [0, \infty]\) be \(\sigma\)-finite, and let \(\{\nu_i\}_{i=1}^n : \fr{M} \to \bb{C}\) be measures such that \(\nu_i \ll \mu\) for all \(i\). Then 
    \[
        \sum_{i=1}^n \nu_i \ll \mu
        \quad \text{and} \quad 
        \frac{\mr{d} \sum_{i=1}^n \nu_i}{\mr{d} \mu} = \sum_{i=1}^n \frac{\mr{d} \nu_i}{\mr{d}\mu}
    \]
\end{problem}

\begin{proof}
    For any \(E \in \fr{M}\), it's clear that \(\mu(E) = 0 \implies \forall i: \nu_i(E) = 0 \implies \sum_{i=1}^n \nu_i(E) = 0\). Thus \(\sum_{i=1}^n \nu_i \ll \mu\). Following that, it's also immediate that 
    \[
        \int_E \sum_{i=1}^n \frac{\mr{d}\nu_i}{\mr{d}\mu} \, \mr{d}\mu
        = \sum_{i=1}^n \int_E \frac{\mr{d}\nu_i}{\mr{d}\mu} \, \mr{d}\mu
        = \sum_{i=1}^n \nu_i(E)
        = \left( \sum_{i=1}^n \nu_i \right) (E)
    \]
    so we conclude once more by uniqueness of the Radon-Nikodym derivative. 
\end{proof}

\begin{problem}{Multiplicativity of the Radon-Nikodym Derivative}*
    [MAT425 P6.5.] Let \((X_i, \fr{M}_i)_{i=1}^n\) be measurable spaces, each with \(\mu_i : \fr{M}_i \to [0, \infty]\) \(\sigma\)-finite and \(\nu_i : \fr{M}_i \to \bb{C}\). If \(\nu_i \ll \mu_i\) for all \(i\), then
    \[
        \prod_{i=1}^n \nu_i \ll \prod_{i=1}^n \mu_i 
        \quad \text{and} \quad 
        \frac{ \mr{d} \prod_{i=1}^n \nu_i }{ \mr{d} \prod_{i=1}^n \mu_i } (x_1, \dots, x_n) = \prod_{i=1}^n \frac{\mr{d}\nu_i}{\mr{d}\mu_i} (x_i) 
    \]
\end{problem}

\begin{proof}
    Without loss of generality we consider \(n=2\). Suppose \((\mu_1 \times \mu_2)(E) = 0\). Then, 
    \[
        0 
        = (\mu_1 \times \mu_2)(E) 
        = \int_{X_1} \mu_2(E_2) \, \mr{d}\mu_1
        = \int_{X_2} \mu_1(E_1) \, \mr{d}\mu_2
    \]
    and therefore \(\mu_2(E_2) = 0\) \(\mu_1\)-a.e., while \(\mu_1(E_1) = 0\) \(\mu_2\)-a.e. It follows that \(\nu_2(E_2) = 0\) \(\mu_1\)-a.e. and \(\nu_1(E_1) = 0\) \(\nu_2\)-a.e., so 
    \[
        (\nu_1 \times \nu_2)(E) = \int_{X_1} \nu_2(E_2) \, \mr{d}\nu_1 = \int_{X_2} \nu_1(E_1) \, \mr{d}\nu_2 = 0
    \]
    thus showing \(\nu_1 \times \nu_2 \ll \mu_1 \times \mu_2\). Finally, by Tonelli's Theorem, we have 
    \[
        \int_{X_1 \times X_2} \cl{X}_E \frac{\mr{d}\nu_1}{\mr{d}\mu_1} \frac{\mr{d}\nu_2}{\mr{d}\mu_2} \, \mr{d} (\mu_1 \times \mu_2)
        = \int_{X_1} \int_{X_2} \cl{X}_{E_2} \frac{\mr{d} \nu_2}{\mr{d} \mu_2} \, \mr{d} \mu_2 \mr{d}\mu_1 
        = \int_{X_1} \nu_2(E_2) \, \mr{d}\mu_1
        = (\nu_1 \times \nu_2)(E)
    \]
    and so the derivative of the products is the product of the derivatives. 
\end{proof}


\begin{problem}{Maximality of the Maximal Function}*
    [RCA E7.1.] Show that \(|f(x)| \leq (Mf)(x)\) at every Lebesgue point of \(f\) if \(f \in L^1(\bb{R}^d)\). 
\end{problem}

\begin{proof}
    TODO. 
\end{proof}



\begin{problem}{Sufficient Condition for \(L^\infty\) Derivative}*
    [RCA E7.10.] If \(f \in \oname{Lip}{1}\) on \([a, b]\), prove that \(f\) is absolutely continuous and that \(f' \in L^\infty\). 
\end{problem}

\begin{proof}
    To show that \(f'\) is absolutely continuous, choose \(\delta = \varepsilon\), so
    \[
        \sum_{i=1}^N |f(\beta_i) - f(\alpha_i)| \leq \sum_{i=1}^N |\beta_i - \alpha_i| < \delta = \varepsilon
    \]
    for disjoint intervals \(I_i = (\alpha_i, \beta_i) \subseteq [a, b]\) whose lengths sum to at most \(\delta\). Therefore, \(f\) is differentiable almost everywhere, and hence \(f'\) exists on \([a, b]\) outside a set \(E\) of measure zero. Then note that by the Lipschitz condition on any distinct \(x, y \in [a, b]\), 
    \[
        \frac{|f(y) - f(x)|}{|y - x|} \leq 1 
        \implies |f'(x)| \leq 1
    \]
    for almost every \(x \in [a, b]\). Hence \(\|f'\|_\infty \leq 1 < \infty\), and \(f' \in L^\infty\). 
\end{proof}


\begin{problem}{Integration by Parts}*
    [RCA E7.14.] Show that the product of two absolutely continuous functions on \([a, b]\) is absolutely continuous. Use this to derive a theorem about integration by parts, i.e. that 
    \[
        \int_a^x f'(t) g(t) \, \mr{d}t = f(t) g(t) \Big\rvert_a^x - \int_a^x f(t) g'(t) \, \mr{d}t
    \]
\end{problem}

\begin{proof}
    Let \(f\) and \(g\) be absolutely continuous on \([a, b]\). Note that since absolute continuity implies continuity, we have that their images are bounded: \(|f| \leq A\) and \(|g| \leq B\), for \(0 < A, B < \infty\). Then, 
    \begin{align*}
        \sum_{i=1}^N |f(\beta_i) g(\beta_i) - f(\alpha_i) g(\alpha_i)|
        &= \sum_{i=1}^N |g(\beta_i) [f(\beta_i) - f(\alpha_i)] + f(\alpha_i) [g(\beta_i) - g(\alpha_i)]| \\
        &\leq B \sum_{i=1}^N |f(\beta_i) - f(\alpha_i)| + A \sum_{i=1}^N |g(\beta_i) - g(\alpha_i)| \\
        &< B((2B)^{-1} \varepsilon) + A ((2A)^{-1} \varepsilon) 
        = \varepsilon
    \end{align*}
    whenever we choose appropriately small \(\delta_f\) and \(\delta_g\). Take \(\delta = \min\{\delta_f, \delta_g\}\) to conclude. Hence \(fg\) is absolutely continuous on \([a, b]\). Then, since
    \[
        \int_a^x \frac{d}{dt} (f(t) g(t)) \, \mr{d}t = f(t) g(t) \Big\rvert_a^x
    \]
    by the Fundamental Theorem of Calculus, applying the product rule to this expression yields the integration by parts formula. 
\end{proof}


\begin{problem}{Gap Between \(C^0\) and \(C^1\)}*
    [RCA E7.15.] Construct a monotonic function \(f\) on \(\bb{R}\) so that \(f'(x)\) exists and is finite for any \(x \in \bb{R}\), but \(f'\) is not a continuous function. 
\end{problem}

\begin{proof}
    TODO. 
\end{proof}


\begin{problem}{Symmetric Derivate of Series}*
    [TODO: find source.] Suppose \(\{\mu_n\}\) is a sequence of positive Borel measures on \(\bb{R}^d\) and \(\mu = \sum_n \mu_n\). Assume that \(\mu(\bb{R}^d) < \infty\). 
    \begin{enumerate}[(a)]
        \itemsep0em
        \item Show that \(\mu\) is a Borel measure.
        \item What is the relation between the Lebesgue decompositions of the \(\mu_n\) and that of \(\mu\)? 
        \item Prove that 
        \[
            (D\mu)(x) = \sum_{n=1}^\infty (D\mu_n)(x)
            \quad \text{a.e.}\, [m]
        \]
        \item Derive corresponding theorems for sequences \(\{f_n\}_n\) of positive nondecreasing functions on \(\bb{R}\) and their sums \(f = \sum_n f_n\). 
    \end{enumerate}
\end{problem}

\begin{proof}
    TODO. 
\end{proof}



\end{document}

