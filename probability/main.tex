\input{preamble}
\usepackage{titlesec}
\usepackage[many]{tcolorbox}

% Adjust spacing after the chapter title
\titlespacing*{\chapter}{0cm}{-2.0cm}{0.50cm}
\titlespacing*{\section}{0cm}{0.50cm}{0.25cm}

\titleformat*{\section}   {\bfseries\sffamily\Large}
\titleformat*{\subsection}{\bfseries\sffamily\large}
\titleformat*{\subsubsection}{\bfseries\sffamily\normalsize}
\titleformat*{\paragraph} {\bfseries\sffamily\normalsize}
\titleformat*{\subparagraph}{\bfseries\sffamily\normalsize}

% Indent 
\setlength{\parindent}{0pt}
\setlength{\parskip}{1ex}

% --- Theorems, lemma, corollary, postulate, definition ---
\usepackage[dvipsnames]{xcolor}
\usepackage{titlesec}
\usepackage[many]{tcolorbox}
\tcbuselibrary{theorems,skins,breakable}

% Custom colors
\definecolor{MyBlue}{HTML}{37CDDE}
\definecolor{MyGreen}{HTML}{50C878}
\definecolor{MyPurple}{HTML}{7851A9}
\definecolor{MyGold}{HTML}{D4AF37}
\definecolor{MyGray}{HTML}{808080}

\newtcbtheorem[]{theorem}{Theorem}%
  {enhanced,
   colback        = red!20,
   colbacktitle   = red!30,
   coltitle       = black,
   boxrule        = 0pt,
   frame hidden,
   arc            = 2mm,
   before skip    = 3ex,
   after skip     = 3ex,
   before title   = {\vspace{2mm}},
   after title    = {\vspace{2mm}},
   fonttitle      = \bfseries\sffamily,
   breakable
  }{theorem}

\newtcbtheorem[]{proposition}{Proposition}%
  {enhanced,
   colback        = MyGold!20,
   colbacktitle   = MyGold!30,
   coltitle       = black,
   boxrule        = 0pt,
   frame hidden,
   arc            = 2mm,
   before skip    = 3ex,
   after skip     = 3ex,
   before title   = {\vspace{2mm}},
   after title    = {\vspace{2mm}},
   fonttitle      = \bfseries\sffamily,
   breakable
  }{proposition}

\newtcbtheorem[]{lemma}{Lemma}%
  {enhanced,
   colback        = MyBlue!20,
   colbacktitle   = MyBlue!30,
   coltitle       = black,
   boxrule        = 0pt,
   frame hidden,
   arc            = 2mm,
   before skip    = 3ex,
   after skip     = 3ex,
   before title   = {\vspace{2mm}},
   after title    = {\vspace{2mm}},
   fonttitle      = \bfseries\sffamily,
   breakable
  }{lemma}

\newtcbtheorem[]{corollary}{Corollary}%
  {enhanced,
   colback        = MyGreen!20,
   colbacktitle   = MyGreen!30,
   coltitle       = black,
   boxrule        = 0pt,
   frame hidden,
   arc            = 2mm,
   before skip    = 3ex,
   after skip     = 3ex,
   before title   = {\vspace{2mm}},
   after title    = {\vspace{2mm}},
   fonttitle      = \bfseries\sffamily,
   breakable
  }{corollary}

\newtcbtheorem[]{definition}{Definition}%
  {enhanced,
   colback        = MyPurple!20,
   colbacktitle   = MyPurple!30,
   coltitle       = black,
   boxrule        = 0pt,
   frame hidden,
   arc            = 2mm,
   before skip    = 3ex,
   after skip     = 3ex,
   before title   = {\vspace{2mm}},
   after title    = {\vspace{2mm}},
   fonttitle      = \bfseries\sffamily,
   breakable
  }{definition}

\newtcbtheorem[]{remark}{Remark}%
  {enhanced,
   colback        = MyGray!20,
   colbacktitle   = MyGray!30,
   coltitle       = black,
   boxrule        = 0pt,
   frame hidden,
   arc            = 2mm,
   before skip    = 3ex,
   after skip     = 3ex,
   before title   = {\vspace{2mm}},
   after title    = {\vspace{2mm}},
   fonttitle      = \bfseries\sffamily,
   breakable
  }{remark}

\newtcbtheorem[]{example}{Example}%
  {enhanced,
   colback        = Violet!20,
   colbacktitle   = Violet!30,
   coltitle       = black,
   boxrule        = 0pt,
   frame hidden,
   arc            = 2mm,
   before skip    = 3ex,
   after skip     = 3ex,
   before title   = {\vspace{2mm}},
   after title    = {\vspace{2mm}},
   fonttitle      = \bfseries\sffamily,
   breakable
  }{example}

\newtcbtheorem[]{problem}{Problem}%
  {enhanced,
   colback        = MyGold!20,
   colbacktitle   = MyGold!30,
   coltitle       = black,
   boxrule        = 0pt,
   frame hidden,
   arc            = 2mm,
   before skip    = 3ex,
   after skip     = 3ex,
   before title   = {\vspace{2mm}},
   after title    = {\vspace{2mm}},
   fonttitle      = \bfseries\sffamily,
   breakable
  }{problem}


% --- Basic commands ---
%   Euler's constant
\newcommand{\eu}{\mathrm{e}}

%   Imaginary unit
\newcommand{\im}{\mathrm{i}}

%   Sexagesimal degree symbol
\newcommand{\grado}{\,^{\circ}}

% --- Comandos para álgebra lineal ---
% Matrix transpose
\newcommand{\transpose}[1]{{#1}^{\mathsf{T}}}

%%% Comandos para cálculo
%   Definite integral from -\infty to +\infty
\newcommand{\Int}{\int\limits_{-\infty}^{\infty}}

%   Indefinite integral
\newcommand{\rint}[2]{\int{#1}\dd{#2}}

%  Definite integral
\newcommand{\Rint}[4]{\int\limits_{#1}^{#2}{#3}\dd{#4}}

%   Dot product symbol (use the command \bigcdot)
\makeatletter
\newcommand*\bigcdot{\mathpalette\bigcdot@{.5}}
\newcommand*\bigcdot@[2]{\mathbin{\vcenter{\hbox{\scalebox{#2}{$\m@th#1\bullet$}}}}}
\makeatother

%   Hamiltonian
\newcommand{\Ham}{\hat{\mathcal{H}}}

%   Trace
\renewcommand{\Tr}{\mathrm{Tr}}

% Christoffel symbol of the second kind
\newcommand{\christoffelsecond}[4]{\dfrac{1}{2}g^{#3 #4}(\partial_{#1} g_{#2 #4} + \partial_{#2} g_{#1 #4} - \partial_{#4} g_{#1 #2})}

% Riemann curvature tensor
\newcommand{\riemanncurvature}[5]{\partial_{#3} \Gamma_{#4 #2}^{#1} - \partial_{#4} \Gamma_{#3 #2}^{#1} + \Gamma_{#3 #5}^{#1} \Gamma_{#4 #2}^{#5} - \Gamma_{#4 #5}^{#1} \Gamma_{#3 #2}^{#5}}

% Covariant Riemann curvature tensor
\newcommand{\covariantriemanncurvature}[5]{g_{#1 #5} R^{#5}{}_{#2 #3 #4}}

% Ricci tensor
\newcommand{\riccitensor}[5]{g_{#1 #5} R^{#5}{}_{#2 #3 #4}}

% Shortcuts
\newcommand{\cl}{\mathcal}
\newcommand{\bb}{\mathbb}
\newcommand{\mb}{\mathbf}
\newcommand{\fr}{\mathfrak}
\newcommand{\oname}{\operatorname}
\newcommand{\ovl}{\overline}
\newcommand{\st}{\;\biggr\rvert\;}
\newcommand{\mr}{\mathrm}
\newcommand{\ds}{\mathds}

% Standard vertical spacing
\newcommand{\stdvspace}{\vspace{0.5ex}}

% Text formatting
\newcommand{\tit}[1]{\textit{#1}}
\newcommand{\tbf}[1]{\textbf{#1}}

% Horizontal line
\newcommand{\horizontal}{\noindent{\rule{\textwidth}{0.4pt}}}



\begin{document}

\begin{Large}
    \textsf{\textbf{Probability Theory (Princeton)}}
\end{Large}

\vspace{1ex}

\textsf{\textbf{Student:}} Joshua Lin \\
\textsf{\textbf{Lecturer:}} Allan Sly

\vspace{2ex}

Problems are largely derived from \emph{Probability and Measure} by Billingsley (BLN).

\section{Probability}

\begin{problem}{Vi\`ete's Formula}*
    [BLN E1.8.] Show that the Rademacher functions satisfy 
    \[
        \int_0^1 \exp \left[ i \sum_{k=1}^n a_k r_k(\omega) \right] \, \mr{d} \omega
        = \prod_{k=1}^n \frac{e^{ia_k} + e^{-ia_k}}{2}
        = \prod_{k=1}^n \cos{a_k}
    \]
    Take \(a_k = t2^{-k}\) and from \(\sum_{k=1}^\infty r_k(\omega) 2^{-k} = 2\omega - 1\) deduce 
    \[
        \frac{\sin{t}}{t} = \prod_{k=1}^\infty \cos \left( \frac{t}{2^k} \right) 
    \]
    by letting \(n \to \infty\) inside the integral above. Derive Vi\`ete's remarkable formula
    \[
        \frac{2}{\pi} = \frac{\sqrt{2}}{2} \frac{\sqrt{2 + \sqrt{2}}}{2} \frac{\sqrt{2 + \sqrt{2 + \sqrt{2}}}}{2} \cdots
    \]
\end{problem}


\begin{proof}
    TODO. 
\end{proof}


\begin{problem}{Borel-Cantelli Lemmas}*
    [BLN Thm 4.3/4.4] Prove the following:
    \begin{enumerate}[(a)]
        \itemsep0em
        \item \emph{First Borel-Cantelli Lemma:} If \(\sum_n \bb{P}(A_n)\) converges, then \(\bb{P}(\limsup_n A_n) = 0\).
        \item \emph{Second Borel-Cantelli Lemma:} If \(\{A_n\}_n\) are independent and \(\sum_n \bb{P}(A_n)\) diverges, then \(\bb{P}(\limsup_n A_n) = 1\).
    \end{enumerate}
\end{problem}


\begin{proof}
    The first Borel-Cantelli lemma is proved (for general measure spaces, even) earlier in the section on measures. For the second Borel-Cantelli lemma, we can show that \((\limsup_n A_n)^c\) occurs with probability zero. Effectively,
    \[
        \bb{P}\left[\limsup_{n \to \infty} A_n\right] = 1
        \iff \bb{P}\left[ \left(\bigcap_{n=1}^\infty \bigcup_{k=n}^\infty A_k\right)^c \right] = 0
        \iff \bb{P}\left[ \bigcap_{k=n}^\infty A_k^c \right] = 0
        \quad (n \in \bb{N})
    \]
    By independence,
    \[
        \bb{P}\left[ \bigcap_{k =n}^\infty A_k^c \right]
        = \prod_{k=n}^\infty \bb{P}[A_k^c]
        = \prod_{k=n}^\infty (1 - \bb{P}(A_k))
        \leq \prod_{k=n}^\infty \exp(-\bb{P}[A_k])
        = \exp\left( -\sum_{k=n}^\infty \bb{P}[A_k] \right)
    \]
    Since \(\sum_k \bb{P}[A_k] = +\infty\), the final expression vanishes, as desired. Note that there is no measure-theoretic generalization of the second lemma, due to the requirement of independence.
\end{proof}


\begin{problem}{Kolgomorov's Zero-One Law}*
    [BLN Thm 4.5.] If \(\{A_n\}_n\) is an independent sequence of events, and if \(A\) is an event in the tail \(\sigma\)-field defined by
    \[
        \cl{T} = \bigcap_{n=1}^\infty \sigma \left( \{A_k\}_{k=n}^\infty \right)
    \]
    then \(\bb{P}(A)\) is either \(0\) or \(1\).
\end{problem}


\begin{problem}{Monotonicity of \(L^p\) for Simple Random Variables}*
    [BLN Eq 5.33.] Prove that if the \(L^p\) norm of a simple random variable \(X\) is \(\|X\|_p := \sqrt[p]{\bb{E}[|X|^p]}\), then for any \(0 < \alpha \leq \beta\), \(\|X\|_\alpha \leq \|X\|_\beta\).
\end{problem}


\begin{problem}{Cantelli's Inequality}*
    [BLN P5.5.] Suppose \(X\) is a random variable with mean \(\mu\) and variance \(\sigma^2\).
    \begin{enumerate}[(a)]
        \itemsep0em
        \item Prove \emph{Cantelli's inequality},
        \[
            \bb{P}[X - \mu \geq \alpha] \leq \frac{\sigma^2}{\sigma^2 + \alpha^2}
            \qquad (\alpha \geq 0)
        \]
        \item Prove \(\bb{P}[|X - \mu| \geq \alpha] \leq 2\sigma^2 / (\sigma^2 + \alpha^2)\). When is this tighter than Chebyshev's inequality?
        \item By considering a particular simple random variable, show Cantelli's inequality is sharp.
    \end{enumerate}
\end{problem}

\begin{proof}
    Without loss of generality, let \(\mu = 0\). Then by Chebyshev's inequality,
    \begin{align*}
        \bb{P}[X \geq \alpha]
        &= \bb{P}[X + c \geq \alpha + c]
        \leq \bb{P}[|X + c| \geq \alpha + c]
        \leq \frac{1}{(\alpha + c)^2} \int_{\Omega} (X + c)^2 \, \mr{d}\bb{P}
        = \frac{\sigma^2 + c^2}{(\alpha + c)^2}
    \end{align*}
    Optimizing the bound with respect to \(c\), we find the minimizer to be \(c = \alpha^{-1} \sigma^2\). Substituting yields Cantelli's inequality. From then it follows that
    \[
        \bb{P}[|X - \mu| \geq \alpha]
        = \bb{P}[X - \mu \geq \alpha] + \bb{P}[-(X - \mu) \geq \alpha]
        \leq \frac{2\sigma^2}{\sigma^2 + \alpha^2}
    \]
    Finally, to show Cantelli's inequality is sharp, let \(X\) have the Rademacher distribution. If \(\alpha = 1\),
    \[
        \bb{P}[X \geq \alpha] = \frac{1}{2} = \frac{\sigma^2}{\sigma^2 + \alpha^2}
    \]
    so equality is attained.
\end{proof}


\begin{problem}{The Law of Large Numbers}*
    [BLN Thm 6.1, 6.2.] Prove the following:
    \begin{enumerate}[(a)]
        \itemsep0em
        \item \emph{The Strong Law:} If \(\{X_n\}_n\) are simple i.i.d. with expectation \(\mu < \infty\), then
        \[
            \bb{P}\left[\ev{X} \to \mu\right]
            \equiv \bb{P}\left[ \lim_{n \to \infty} \frac{1}{n} \sum_{k=1}^n X_k = \mu \right]
            = 1
        \]
        \item \emph{The Weak Law:} Let \((\Omega_n, \fr{M}_n, \bb{P}_n)\) be a sequence of real probability spaces. If \(\{X_{nk}\}_{k=1}^{r_n}\) are simple and independent in \(k\) for each \(n\), and \(\{v_n\}_n\) satisfies \(\sigma_n / v_n \to 0\), then
        \[
            \lim_{n \to \infty} \bb{P}_n \left[ \left| \frac{\ev{X_n} - \mu_n}{v_n} \right| \geq \varepsilon \right] = 0
        \]
    \end{enumerate}
\end{problem}


\end{document}

