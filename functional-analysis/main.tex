% preamble.tex

\documentclass[letterpaper, 11pt]{extarticle}
% \usepackage{fontspec}

% ==================================================

% document parameters
% \usepackage[spanish, mexico, es-lcroman]{babel}
\usepackage[english]{babel}
\usepackage[margin = 1in]{geometry}

% ==================================================

% Packages for math
\usepackage{mathrsfs}
\usepackage{amsfonts}
\usepackage{amsmath}
\usepackage{amsthm}
\usepackage{amssymb}
\usepackage{physics}
\usepackage{dsfont}
\usepackage{esint}

% ==================================================

% Packages for writing
\usepackage{enumerate}
\usepackage[shortlabels]{enumitem}
\usepackage{framed}
\usepackage{csquotes}

% ==================================================

% Miscellaneous packages
\usepackage[toc,page]{appendix}
\usepackage{csquotes}
\usepackage{float}
\usepackage{tabularx}
\usepackage{xcolor}
\usepackage{multicol}
\usepackage{subcaption}
\usepackage{caption}
\captionsetup{format = hang, margin = 10pt, font = small, labelfont = bf}

% Citation
\usepackage[round, authoryear]{natbib}

% Widehat + other math commands
\usepackage{yhmath}

% Hyperlinks setup
\usepackage{hyperref}
\definecolor{links}{rgb}{0.36,0.54,0.66}
\hypersetup{
   colorlinks = true,
    linkcolor = black,
     urlcolor = blue,
    citecolor = blue,
    filecolor = blue,
    pdfauthor = {Author},
     pdftitle = {Title},
   pdfsubject = {subject},
  pdfkeywords = {one, two},
  pdfproducer = {LaTeX},
   pdfcreator = {pdfLaTeX},
   }

\usepackage{titlesec}
\usepackage[many]{tcolorbox}

% Adjust spacing after the chapter title
\titlespacing*{\chapter}{0cm}{-2.0cm}{0.50cm}
\titlespacing*{\section}{0cm}{0.50cm}{0.25cm}

\titleformat*{\section}   {\bfseries\sffamily\Large}
\titleformat*{\subsection}{\bfseries\sffamily\large}
\titleformat*{\subsubsection}{\bfseries\sffamily\normalsize}
\titleformat*{\paragraph} {\bfseries\sffamily\normalsize}
\titleformat*{\subparagraph}{\bfseries\sffamily\normalsize}

% Indent 
\setlength{\parindent}{0pt}
\setlength{\parskip}{1ex}

% --- Theorems, lemma, corollary, postulate, definition ---
\usepackage[dvipsnames]{xcolor}
\usepackage{titlesec}
\usepackage[many]{tcolorbox}
\tcbuselibrary{theorems,skins,breakable}

% Custom colors
\definecolor{MyBlue}{HTML}{37CDDE}
\definecolor{MyGreen}{HTML}{50C878}
\definecolor{MyPurple}{HTML}{7851A9}
\definecolor{MyGold}{HTML}{D4AF37}
\definecolor{MyGray}{HTML}{808080}

\newtcbtheorem[]{theorem}{Theorem}%
  {enhanced,
   colback        = red!20,
   colbacktitle   = red!30,
   coltitle       = black,
   boxrule        = 0pt,
   frame hidden,
   arc            = 2mm,
   before skip    = 3ex,
   after skip     = 3ex,
   before title   = {\vspace{2mm}},
   after title    = {\vspace{2mm}},
   fonttitle      = \bfseries\sffamily,
   breakable
  }{theorem}

\newtcbtheorem[]{proposition}{Proposition}%
  {enhanced,
   colback        = MyGold!20,
   colbacktitle   = MyGold!30,
   coltitle       = black,
   boxrule        = 0pt,
   frame hidden,
   arc            = 2mm,
   before skip    = 3ex,
   after skip     = 3ex,
   before title   = {\vspace{2mm}},
   after title    = {\vspace{2mm}},
   fonttitle      = \bfseries\sffamily,
   breakable
  }{proposition}

\newtcbtheorem[]{lemma}{Lemma}%
  {enhanced,
   colback        = MyBlue!20,
   colbacktitle   = MyBlue!30,
   coltitle       = black,
   boxrule        = 0pt,
   frame hidden,
   arc            = 2mm,
   before skip    = 3ex,
   after skip     = 3ex,
   before title   = {\vspace{2mm}},
   after title    = {\vspace{2mm}},
   fonttitle      = \bfseries\sffamily,
   breakable
  }{lemma}

\newtcbtheorem[]{corollary}{Corollary}%
  {enhanced,
   colback        = MyGreen!20,
   colbacktitle   = MyGreen!30,
   coltitle       = black,
   boxrule        = 0pt,
   frame hidden,
   arc            = 2mm,
   before skip    = 3ex,
   after skip     = 3ex,
   before title   = {\vspace{2mm}},
   after title    = {\vspace{2mm}},
   fonttitle      = \bfseries\sffamily,
   breakable
  }{corollary}

\newtcbtheorem[]{definition}{Definition}%
  {enhanced,
   colback        = MyPurple!20,
   colbacktitle   = MyPurple!30,
   coltitle       = black,
   boxrule        = 0pt,
   frame hidden,
   arc            = 2mm,
   before skip    = 3ex,
   after skip     = 3ex,
   before title   = {\vspace{2mm}},
   after title    = {\vspace{2mm}},
   fonttitle      = \bfseries\sffamily,
   breakable
  }{definition}

\newtcbtheorem[]{remark}{Remark}%
  {enhanced,
   colback        = MyGray!20,
   colbacktitle   = MyGray!30,
   coltitle       = black,
   boxrule        = 0pt,
   frame hidden,
   arc            = 2mm,
   before skip    = 3ex,
   after skip     = 3ex,
   before title   = {\vspace{2mm}},
   after title    = {\vspace{2mm}},
   fonttitle      = \bfseries\sffamily,
   breakable
  }{remark}

\newtcbtheorem[]{example}{Example}%
  {enhanced,
   colback        = Violet!20,
   colbacktitle   = Violet!30,
   coltitle       = black,
   boxrule        = 0pt,
   frame hidden,
   arc            = 2mm,
   before skip    = 3ex,
   after skip     = 3ex,
   before title   = {\vspace{2mm}},
   after title    = {\vspace{2mm}},
   fonttitle      = \bfseries\sffamily,
   breakable
  }{example}

\newtcbtheorem[]{problem}{Problem}%
  {enhanced,
   colback        = MyGold!20,
   colbacktitle   = MyGold!30,
   coltitle       = black,
   boxrule        = 0pt,
   frame hidden,
   arc            = 2mm,
   before skip    = 3ex,
   after skip     = 3ex,
   before title   = {\vspace{2mm}},
   after title    = {\vspace{2mm}},
   fonttitle      = \bfseries\sffamily,
   breakable
  }{problem}


% --- Basic commands ---
%   Euler's constant
\newcommand{\eu}{\mathrm{e}}

%   Imaginary unit
\newcommand{\im}{\mathrm{i}}

%   Sexagesimal degree symbol
\newcommand{\grado}{\,^{\circ}}

% --- Comandos para álgebra lineal ---
% Matrix transpose
\newcommand{\transpose}[1]{{#1}^{\mathsf{T}}}

%%% Comandos para cálculo
%   Definite integral from -\infty to +\infty
\newcommand{\Int}{\int\limits_{-\infty}^{\infty}}

%   Indefinite integral
\newcommand{\rint}[2]{\int{#1}\dd{#2}}

%  Definite integral
\newcommand{\Rint}[4]{\int\limits_{#1}^{#2}{#3}\dd{#4}}

%   Dot product symbol (use the command \bigcdot)
\makeatletter
\newcommand*\bigcdot{\mathpalette\bigcdot@{.5}}
\newcommand*\bigcdot@[2]{\mathbin{\vcenter{\hbox{\scalebox{#2}{$\m@th#1\bullet$}}}}}
\makeatother

%   Hamiltonian
\newcommand{\Ham}{\hat{\mathcal{H}}}

%   Trace
\renewcommand{\Tr}{\mathrm{Tr}}

% Christoffel symbol of the second kind
\newcommand{\christoffelsecond}[4]{\dfrac{1}{2}g^{#3 #4}(\partial_{#1} g_{#2 #4} + \partial_{#2} g_{#1 #4} - \partial_{#4} g_{#1 #2})}

% Riemann curvature tensor
\newcommand{\riemanncurvature}[5]{\partial_{#3} \Gamma_{#4 #2}^{#1} - \partial_{#4} \Gamma_{#3 #2}^{#1} + \Gamma_{#3 #5}^{#1} \Gamma_{#4 #2}^{#5} - \Gamma_{#4 #5}^{#1} \Gamma_{#3 #2}^{#5}}

% Covariant Riemann curvature tensor
\newcommand{\covariantriemanncurvature}[5]{g_{#1 #5} R^{#5}{}_{#2 #3 #4}}

% Ricci tensor
\newcommand{\riccitensor}[5]{g_{#1 #5} R^{#5}{}_{#2 #3 #4}}

% Shortcuts
\newcommand{\cl}{\mathcal}
\newcommand{\bb}{\mathbb}
\newcommand{\mb}{\mathbf}
\newcommand{\fr}{\mathfrak}
\newcommand{\oname}{\operatorname}
\newcommand{\ovl}{\overline}
\newcommand{\st}{\;\biggr\rvert\;}
\newcommand{\mr}{\mathrm}

% Standard vertical spacing
\newcommand{\stdvspace}{\vspace{0.5ex}}

% Text formatting
\newcommand{\tit}[1]{\textit{#1}}
\newcommand{\tbf}[1]{\textbf{#1}}


\begin{document}

\begin{Large}
    \textsf{\textbf{Princeton Lectures in Analysis: Functional Analysis}}
\end{Large}

\vspace{1ex}

\textsf{\textbf{Student:}} Joshua Lin \\
\textsf{\textbf{Lecturer:}} Jacob Shapiro

\vspace{2ex}

\section{Topological Vector Spaces}

\begin{problem}{Complex Euclidean Space is a Topological Vector Space}*
    [MAT520 P1.1.] Prove that \(\bb{C}^n\) with its Euclidean topology is a topological vector space, i.e., show that vector addition and scalar multiplication are continuous with respect to the Euclidean topology. 
\end{problem}

\begin{proof}
    To show that vector addition is continuous, take any \(z, w \in \bb{C}^n\), and fix some \(\varepsilon > 0\). Take \(\delta = \varepsilon / 2\), and choose \(x \in B_{\delta}(z), y \in B_{\delta}(y)\). Then, 
    \[
        |(x + y) - (z + w)|
        \leq |x - z| + |y - w|
        < 2 \delta 
        = \varepsilon
    \]
    Thus \(B_\delta(z) + B_\delta(w) \subseteq B_\varepsilon(z+w)\), and vector addition is continuous. 
    \stdvspace

    To show that scalar multiplication is continuous, fix any \(\alpha \in \bb{C} \setminus \{0\}, x \in \bb{C}\). By definition, we must show that for any \(U \in \oname{Nbhd}(\alpha x)\), there exists some \(r > 0\) and \(V \in \oname{Nbhd}(x)\), such that \(\beta V \subseteq U\) whenever \(|\beta - \alpha| < r\). Since the case of \(x = 0\) is trivial, suppose \(x \neq 0\). It suffices to show that if \(|\beta - \alpha| < r\) and \(|z - x| < \delta\) for some \(r, \delta > 0\) yet to be chosen, then \(|\beta z - \alpha x| < \varepsilon\). But, 
    \[
        |\beta z - \alpha x| = |\beta(x + z - x) - \alpha x|
        \leq |\beta - \alpha| |x| + |\beta| |z - x|
        \leq r |x| + (|\alpha| + r) \delta
    \]
    Hence if \(r = \varepsilon / 2|x|\) and \(\delta = \varepsilon / 2 (|\alpha| + r)\), we obtain the desired bound. 
\end{proof}

\begin{problem}{Counterexample: TVS But Not \(T_1\)}*
    [MAT520 P1.2.] Provide an example for a topological vector space that is not \(T_1\). 
\end{problem}

\begin{proof}
    Equip \(\bb{C}\) with the trivial topology, which is not \(T_1\). Observe that the only non-empty open set is \(\bb{C}\) itself, and so it can be easily shown that addition and scalar multiplication are continuous. 
\end{proof}

\begin{problem}{French Metro Metric is not a Topological Vector Space}*
    [MAT520 P1.3.] Prove that \(\bb{C}\) with the French metro metric is \textit{not} homeomorphic to \(\bb{C}\) with the Euclidean metric. Conclude (why?) that \(\bb{C}\) with the French metro metric is not a TVS. 
\end{problem}

\begin{proof}
    Let \(\bb{C}_E\) and \(\bb{C}_F\) be the complex plane with topologies induced by the Euclidean and French metro metrics, respectively. Suppose for contradiction that \(\bb{C}_E \cong \bb{C}_F\). Fix \(z \in \bb{C}_F\), and let \(w = f^{-1}(z) \in \bb{C}_E\) where \(f\) is a homeomorphism. For some \(\delta > 0\), note \(f(B_\delta(w) \setminus \{w\})\) is a punctured neighborhood of \(z\) in \(\bb{C}_F\), which cannot be path connected. But \(B_\delta(w) \setminus \{w\} \subseteq \bb{C}_E\) is path connected, and so since path connectedness is a topological property, this is a contradiction. 
    \stdvspace 

    As a result, \(\bb{C}_F\) cannot be a TVS: two vector spaces of identical dimensions over the same field are isomorphic, and every isomorphism of \(\bb{C}_E\) onto an \(n\)-dimensional TVS over \(\bb{C}\) is a homeomorphism [RCA Thm 1.21], and thus any complex one-dimensional TVS over the field \(\bb{C}\) would be homeomorphic to \(\bb{C}_E\). 
\end{proof}

\begin{problem}{TVS Induced by Function Metric}*
    [MAT520 P1.4.] Let \(C := \{f : [0,1] \to \mathbb{C} \text{ continuous}\} \) and define
    \[
        d(f,g) := \int_{0}^{1} \frac{|f(x)-g(x)|}{1 + |f(x)-g(x)|}\, \mr{d}x
    \]
    Show that \(d\) is a metric on \(C\), show that \(C\) is a vector space (with pointwise addition and scalar multiplication), and show that the topology which \(d\) induces on \(C\) makes it into a TVS. Show that this TVS has a countable local base.
\end{problem}

\begin{proof}
    First we show that \(d\) is a metric on \(C\). 
    \begin{itemize}
        \itemsep0em
        \item \emph{Positive-Definite.} It's clear that \(d(f, g) \geq 0\) always. Note that \(d(f,g) = 0\) when \(f = g\). Now suppose \(f \neq g\), and let \(h := |f - g| \in C\). By assumption there exists some \(x_0 \in [0,1]\) such that \(y_0 := h(x_0) > 0\). By continuity, for any \(\varepsilon > 0\) there exists some \(\delta' > 0\) such that whenever \(|x - x_0| < \delta'\), we have that \(|h(x) - y_0| < \varepsilon\). Equivalently, \(y_0 - \varepsilon < h(x) < y_0 + \varepsilon\). So, if we choose \(\varepsilon < y_0\) and define \(\delta := \min\{\delta', \min\{x_0, 1 - x_0\}\}\), we have that 
        \[
            d(f, g)
            \geq \int_{x_0 - \delta}^{x_0 + \delta} \frac{h(x)}{1 + h(x)} \, \mr{d}x 
            \geq \int_{x_0 - \delta}^{x_0 + \delta} \frac{y_0 - \varepsilon}{1 + y_0 + \varepsilon} \, \mr{d}x
            = \frac{2\delta (y_0 - \varepsilon)}{1 + y_0 + \varepsilon}
            > 0
        \]
        so that \(d(f, g) > 0\) whenever \(f \neq g\). 
        \item \emph{Symmetric.} It's clear that \(d(f, g) = d(g, f)\). 
        \item \emph{Triangle-Inequality.} It suffices to show that 
        \[
            \frac{|x - y|}{1 + |x - y|} \leq \frac{|x - z|}{1 + |x - z|} + \frac{|z - y|}{1 + |z - y|}
        \]
        which would follow immediately if \(\phi(t) := t(1+t)^{-1}\) were sub-additive, since \(\phi\) is monotonically increasing in \(t\). As such, observe that
        \[
            \phi(a) + \phi(b) - \phi(a + b)
            = 1 - \frac{1}{1+a} - \frac{1}{1+b} + \frac{1}{1+a+b}
            = \frac{ab(a+b+2)}{(1+a)(1+b)(1+a+b)}
            \geq 0
        \]
        so \(\phi\) is indeed sub-additive, so the triangle inequality holds. 
    \end{itemize}

    Next, it's clear that \(C\) is a vector space over \(\bb{C}\), so let us show it is a TVS. 
    \begin{itemize}
        \itemsep0em
        \item \emph{Addition.} Fix any \(\varepsilon > 0\) and \(f_0, g_0 \in C\). Let \(U := \{h \in C \mid d(h, f_0 + g_0) < \varepsilon\} \in \oname{Nbhd}(f_0 + g_0)\). We must show that there exists neighborhoods \(V \in \oname{Nbhd}(f_0), W \in \oname{Nbhd}(g_0)\) such that \(V + W \subseteq U\). Choose \(V = \{f \in C \mid d(f, f_0) < \delta\}\) and \(W = \{g \in C \mid d(g, g_0) < \delta\}\) for \(\delta = \varepsilon / 2\). Using monotonicity and sub-additivity of \(\phi\) once again, 
        \begin{align*}
            d(f + g, f_0 + g_0) 
            &= \int_0^1 \phi(|f+g - f_0 - g_0|) \, \mr{d}x
            \leq \int_0^1 \phi(|f-f_0| + |g - g_0|) \, \mr{d}x \\
            &\leq \int_0^1 \phi(|f - f_0|) \, \mr{d}x + \int_0^1 \phi(|g - g_0|) \, \mr{d}x
            = d(f, f_0) + d(g, g_0)
            < \varepsilon
        \end{align*}
        so that \(f+g \in U\). Hence \(V + W \subseteq U\). 
        \item \emph{Scalar Multiplication.} Fix \(f_0 \in C, \alpha \in \bb{C} \setminus \{0\}\). Let \(U := \{h \in C \mid d(h, \alpha f_0) < \varepsilon\}\) and \(V := \{f \in C \mid d(f, f_0) < \delta\}\) for some \(\delta > 0\) yet to be chosen. We must show that if \(|\beta - \alpha| < r\) for some \(r > 0\) yet to be chosen, then \(\beta V \subseteq U\). TODO. 
    \end{itemize}
\end{proof}

\newpage
\section{Banach Spaces}

\begin{problem}{Equivalence of Complex Norms}*
    [MAT520 P1.12.] On a normed space \(X\), two norms are \(\|\cdot\|_1 \|\cdot\|_2\) are \emph{equivalent} if and only if there exists \(a, b \in (0, \infty)\) such that 
    \[
        a\|x\|_1 \leq \|x\|_2 \leq b\|x\|_1 \quad (x \in X)
    \]
    Prove all norms on \(\bb{C}^n\) are equivalent. Conclude all norm topologies on \(\bb{C}^n\) are homeomorphic. 
\end{problem}

\begin{proof}
    It suffices to show that all norms are equivalent to the Euclidean norm, \(\|\cdot\|_E\). Choose some other norm \(\|\cdot\|_O\) on \(\bb{C}^n\), and define 
    \[
        K := \{w \in \bb{C}^n \mid \|w\|_E = 1\}, \quad 
        a := \inf_{w \in K} \|w\|_O, \quad 
        b := \sup_{w \in K} \|w\|_O
    \]
    Note that the desired bounds are satisfied:
    \[
        a \|z\|_E 
        \leq \left\|\frac{z}{\|z\|_E}\right\|_O \cdot \|z\|_E
        = \|z\|_O
        = \left\|\frac{z}{\|z\|_E}\right\|_O \cdot \|z\|_E
        \leq b \|z\|_E
    \]
    so it remains to be shown that \(a, b \in (0, \infty)\). Observe that \(K\) is compact by the Heine-Borel theorem since it is closed and bounded. We claim that the map \(\phi : K \to \bb{R}\) given by \(z \mapsto \|z\|_O\) is continuous, from which it would follow that it attains a minimum and maximum. By positive-definiteness and finiteness of norms, we would conclude \(a, b \in (0, \infty)\) and hence equivalence of all norms on \(\bb{C}^n\). Begin by observing that 
    \[
        \|x\|_O = \left\| \sum_{i=1}^n x_i \hat{e}_i \right\|_O 
        \leq \sum_{i=1}^n |x_i| \|\hat{e}_i\|_O 
        \leq \|x\|_\infty \sum_{i=1}^n \|\hat{e}_i\|_O 
        =: M\|x\|_\infty
    \]
    where \(0 < M < \infty\). Now fix any \(\varepsilon > 0\), and suppose \(\|z - w\|_E < \delta\) for \(w \in \bb{C}^n\) and some \(\delta > 0\) yet to be chosen. Then, by the (reverse) triangle inequality, 
    \[
        |\phi(z) - \phi(w)|
        \leq \left| \|z\|_O - \|w\|_O \right|
        \leq \|z - w\|_O 
        \leq M \|z - w\|_\infty
        \leq M \|z - w\|_E 
        < M \delta 
    \]
    so that if \(\delta := M^{-1} \varepsilon\), we have our \(\delta\) of continuity. Hence all norms on \(\bb{C}^n\) are equivalent. Note that the identity map \(\oname{id} : \bb{C}^n \to \bb{C}^n\) is a homeomorphism between any two normed spaces with equivalent norms, so actually all normed spaces are homeomorphic on \(\bb{C}^n\). 
\end{proof}

\begin{problem}{Boundedness Equals Continuity for Linear Maps}*
    [MAT520 P1.15.] Let \(A : X \to Y\) be a linear map between Banach spaces \(X\) and \(Y\). Prove that the following statements are equivalent: 
    \begin{enumerate}
        \itemsep0em
        \item \(A\) is bounded, in the sense that \(\|Ax\| \leq M\|x\|\) for some \(M \in \bb{R}\) and all \(x \in X\). 
        \item \(A\) is continuous according to the Banach norms. 
        \item \(A\) is continuous at zero according to the Banach norms. 
    \end{enumerate}
\end{problem}

\begin{proof}
    \((1) \implies (2)\). Suppose \(A\) is bounded with bound \(M\), and fix \(\varepsilon > 0\). Then if \(\|x - x_0\| \leq \varepsilon\), 
    \[
        \|Ax - Ax_0\|
        \leq \|A(x - x_0)\|
        \leq M \|x - x_0\|
        = M \delta < \varepsilon
    \]
    whenever \(\delta < M^{-1} \varepsilon\), so \(A\) is continuous. 
    \stdvspace

    \((2) \implies (3)\). If \(A\) is continuous, then it is certainly continuous at \(0 \in X\). 
    \stdvspace

    \((3) \implies (1)\). Suppose \(A\) is continuous at \(0 \in X\). For \(\|x\| \leq \delta\), we have that \(\|Ax\| \leq 1\). Thus, 
    \[
        \left\| \frac{A(\delta x)}{\|x\|} \right\| \leq 1
        \implies 
        \|Ax\| \leq \frac{1}{\delta} \|x\|
        \quad (x \in X \setminus \{0\})
    \]
    so we have a \(\delta^{-1}\) bound on \(A\).
\end{proof}

\begin{problem}{Equivalent Condition to Completeness}*
    [MAT520 P1.19.] Let \(X\) be a normed space. Show that the following conditions are equivalent: 
    \begin{enumerate}
        \itemsep0em
        \item \(X\) is complete. 
        \item For any \(\{x_n\}_n \subseteq X\), 
        \[
            \sum_{n=1}^\infty \|x_n\| < \infty
            \implies 
            \sum_{n=1}^\infty x_n = x \in X
        \]
    \end{enumerate}
\end{problem}

\begin{proof}
    For the forward direction, suppose \(X\) is complete. Let \(\{x_n\}_n \subseteq X\) be a sequence satisfying \(\sum_n \|x_n\| < \infty\). Note that the tail sequences \(s_N := \sum_{n=1}^N x_n\) are Cauchy, since for \(n \leq m\), 
    \[
        \|s_n - s_m\|
        = \left\| \sum_{k=n+1}^m x_k \right\|
        \leq \sum_{k=n+1}^\infty \|x_k\|
        \leq \varepsilon
    \]
    whenever \(n\) is large enough, as it is a tail sum. Hence by completeness, \(s_N \to x \in X\), as desired. 
    \stdvspace

    For the reverse direction, suppose that \(\sum_n \|x_n\| < \infty \implies \sum_n x_n \in X\). Let \(\{x_n\}_n\) be a Cauchy sequence. For each \(k \in \bb{N}\), choose \(n_k\) such that \(\|x_{n_k} - x_n\| \leq 2^{-k}\) whenever \(n \geq n_k\), and without loss of generality let \(n_k\) be increasing. Then,
    \[
        \sum_{k=1}^\infty \|x_{n_{k+1}} - x_{n_k}\|
        \leq \sum_{k=1}^\infty 2^{-k} 
        = 1 
        < \infty
        \implies 
        \lim_{k \to \infty} x_{n_k} = \sum_{k=1}^\infty (x_{n_{k+1}} - x_{n_k}) 
        \equiv x \in X
    \]
    and so \(\{x_{n_k}\}_k\) is actually a convergent subsequence of \(\{x_n\}_n\). But for \(n, n_k\) large enough, we have 
    \[
        \|x_n - x\| \leq \|x_{n_k} - x_n\| + \|x_{n_k} - x\| \leq \frac{\varepsilon}{2} + \frac{\varepsilon}{2} = \varepsilon
    \]
    which is a simple restatement of the fact that a Cauchy sequence always converges to an (existing) limit of a subsequence. Hence \(x_n \to x\) and so \(X\) is complete. 
\end{proof}

\end{document}
