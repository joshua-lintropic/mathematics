% preamble.tex

\documentclass[letterpaper, 11pt]{extarticle}
% \usepackage{fontspec}

% ==================================================

% document parameters
% \usepackage[spanish, mexico, es-lcroman]{babel}
\usepackage[english]{babel}
\usepackage[margin = 1in]{geometry}

% ==================================================

% Packages for math
\usepackage{mathrsfs}
\usepackage{amsfonts}
\usepackage{amsmath}
\usepackage{amsthm}
\usepackage{amssymb}
\usepackage{physics}
\usepackage{dsfont}
\usepackage{esint}

% ==================================================

% Packages for writing
\usepackage{enumerate}
\usepackage[shortlabels]{enumitem}
\usepackage{framed}
\usepackage{csquotes}

% ==================================================

% Miscellaneous packages
\usepackage[toc,page]{appendix}
\usepackage{csquotes}
\usepackage{float}
\usepackage{tabularx}
\usepackage{xcolor}
\usepackage{multicol}
\usepackage{subcaption}
\usepackage{caption}
\captionsetup{format = hang, margin = 10pt, font = small, labelfont = bf}

% Citation
\usepackage[round, authoryear]{natbib}

% Widehat + other math commands
\usepackage{yhmath}

% Hyperlinks setup
\usepackage{hyperref}
\definecolor{links}{rgb}{0.36,0.54,0.66}
\hypersetup{
   colorlinks = true,
    linkcolor = black,
     urlcolor = blue,
    citecolor = blue,
    filecolor = blue,
    pdfauthor = {Author},
     pdftitle = {Title},
   pdfsubject = {subject},
  pdfkeywords = {one, two},
  pdfproducer = {LaTeX},
   pdfcreator = {pdfLaTeX},
   }

\usepackage{titlesec}
\usepackage[many]{tcolorbox}

% Adjust spacing after the chapter title
\titlespacing*{\chapter}{0cm}{-2.0cm}{0.50cm}
\titlespacing*{\section}{0cm}{0.50cm}{0.25cm}

\titleformat*{\section}   {\bfseries\sffamily\Large}
\titleformat*{\subsection}{\bfseries\sffamily\large}
\titleformat*{\subsubsection}{\bfseries\sffamily\normalsize}
\titleformat*{\paragraph} {\bfseries\sffamily\normalsize}
\titleformat*{\subparagraph}{\bfseries\sffamily\normalsize}

% Indent 
\setlength{\parindent}{0pt}
\setlength{\parskip}{1ex}

% --- Theorems, lemma, corollary, postulate, definition ---
\usepackage[dvipsnames]{xcolor}
\usepackage{titlesec}
\usepackage[many]{tcolorbox}
\tcbuselibrary{theorems,skins,breakable}

% Custom colors
\definecolor{MyBlue}{HTML}{37CDDE}
\definecolor{MyGreen}{HTML}{50C878}
\definecolor{MyPurple}{HTML}{7851A9}
\definecolor{MyGold}{HTML}{D4AF37}
\definecolor{MyGray}{HTML}{808080}

\newtcbtheorem[]{theorem}{Theorem}%
  {enhanced,
   colback        = red!20,
   colbacktitle   = red!30,
   coltitle       = black,
   boxrule        = 0pt,
   frame hidden,
   arc            = 2mm,
   before skip    = 3ex,
   after skip     = 3ex,
   before title   = {\vspace{2mm}},
   after title    = {\vspace{2mm}},
   fonttitle      = \bfseries\sffamily,
   breakable
  }{theorem}

\newtcbtheorem[]{proposition}{Proposition}%
  {enhanced,
   colback        = MyGold!20,
   colbacktitle   = MyGold!30,
   coltitle       = black,
   boxrule        = 0pt,
   frame hidden,
   arc            = 2mm,
   before skip    = 3ex,
   after skip     = 3ex,
   before title   = {\vspace{2mm}},
   after title    = {\vspace{2mm}},
   fonttitle      = \bfseries\sffamily,
   breakable
  }{proposition}

\newtcbtheorem[]{lemma}{Lemma}%
  {enhanced,
   colback        = MyBlue!20,
   colbacktitle   = MyBlue!30,
   coltitle       = black,
   boxrule        = 0pt,
   frame hidden,
   arc            = 2mm,
   before skip    = 3ex,
   after skip     = 3ex,
   before title   = {\vspace{2mm}},
   after title    = {\vspace{2mm}},
   fonttitle      = \bfseries\sffamily,
   breakable
  }{lemma}

\newtcbtheorem[]{corollary}{Corollary}%
  {enhanced,
   colback        = MyGreen!20,
   colbacktitle   = MyGreen!30,
   coltitle       = black,
   boxrule        = 0pt,
   frame hidden,
   arc            = 2mm,
   before skip    = 3ex,
   after skip     = 3ex,
   before title   = {\vspace{2mm}},
   after title    = {\vspace{2mm}},
   fonttitle      = \bfseries\sffamily,
   breakable
  }{corollary}

\newtcbtheorem[]{definition}{Definition}%
  {enhanced,
   colback        = MyPurple!20,
   colbacktitle   = MyPurple!30,
   coltitle       = black,
   boxrule        = 0pt,
   frame hidden,
   arc            = 2mm,
   before skip    = 3ex,
   after skip     = 3ex,
   before title   = {\vspace{2mm}},
   after title    = {\vspace{2mm}},
   fonttitle      = \bfseries\sffamily,
   breakable
  }{definition}

\newtcbtheorem[]{remark}{Remark}%
  {enhanced,
   colback        = MyGray!20,
   colbacktitle   = MyGray!30,
   coltitle       = black,
   boxrule        = 0pt,
   frame hidden,
   arc            = 2mm,
   before skip    = 3ex,
   after skip     = 3ex,
   before title   = {\vspace{2mm}},
   after title    = {\vspace{2mm}},
   fonttitle      = \bfseries\sffamily,
   breakable
  }{remark}

\newtcbtheorem[]{example}{Example}%
  {enhanced,
   colback        = Violet!20,
   colbacktitle   = Violet!30,
   coltitle       = black,
   boxrule        = 0pt,
   frame hidden,
   arc            = 2mm,
   before skip    = 3ex,
   after skip     = 3ex,
   before title   = {\vspace{2mm}},
   after title    = {\vspace{2mm}},
   fonttitle      = \bfseries\sffamily,
   breakable
  }{example}

\newtcbtheorem[]{problem}{Problem}%
  {enhanced,
   colback        = MyGold!20,
   colbacktitle   = MyGold!30,
   coltitle       = black,
   boxrule        = 0pt,
   frame hidden,
   arc            = 2mm,
   before skip    = 3ex,
   after skip     = 3ex,
   before title   = {\vspace{2mm}},
   after title    = {\vspace{2mm}},
   fonttitle      = \bfseries\sffamily,
   breakable
  }{problem}


% --- Basic commands ---
%   Euler's constant
\newcommand{\eu}{\mathrm{e}}

%   Imaginary unit
\newcommand{\im}{\mathrm{i}}

%   Sexagesimal degree symbol
\newcommand{\grado}{\,^{\circ}}

% --- Comandos para álgebra lineal ---
% Matrix transpose
\newcommand{\transpose}[1]{{#1}^{\mathsf{T}}}

%%% Comandos para cálculo
%   Definite integral from -\infty to +\infty
\newcommand{\Int}{\int\limits_{-\infty}^{\infty}}

%   Indefinite integral
\newcommand{\rint}[2]{\int{#1}\dd{#2}}

%  Definite integral
\newcommand{\Rint}[4]{\int\limits_{#1}^{#2}{#3}\dd{#4}}

%   Dot product symbol (use the command \bigcdot)
\makeatletter
\newcommand*\bigcdot{\mathpalette\bigcdot@{.5}}
\newcommand*\bigcdot@[2]{\mathbin{\vcenter{\hbox{\scalebox{#2}{$\m@th#1\bullet$}}}}}
\makeatother

%   Hamiltonian
\newcommand{\Ham}{\hat{\mathcal{H}}}

%   Trace
\renewcommand{\Tr}{\mathrm{Tr}}

% Christoffel symbol of the second kind
\newcommand{\christoffelsecond}[4]{\dfrac{1}{2}g^{#3 #4}(\partial_{#1} g_{#2 #4} + \partial_{#2} g_{#1 #4} - \partial_{#4} g_{#1 #2})}

% Riemann curvature tensor
\newcommand{\riemanncurvature}[5]{\partial_{#3} \Gamma_{#4 #2}^{#1} - \partial_{#4} \Gamma_{#3 #2}^{#1} + \Gamma_{#3 #5}^{#1} \Gamma_{#4 #2}^{#5} - \Gamma_{#4 #5}^{#1} \Gamma_{#3 #2}^{#5}}

% Covariant Riemann curvature tensor
\newcommand{\covariantriemanncurvature}[5]{g_{#1 #5} R^{#5}{}_{#2 #3 #4}}

% Ricci tensor
\newcommand{\riccitensor}[5]{g_{#1 #5} R^{#5}{}_{#2 #3 #4}}

% Shortcuts
\newcommand{\cl}{\mathcal}
\newcommand{\bb}{\mathbb}
\newcommand{\mb}{\mathbf}
\newcommand{\fr}{\mathfrak}
\newcommand{\oname}{\operatorname}
\newcommand{\ovl}{\overline}
\newcommand{\st}{\;\biggr\rvert\;}
\newcommand{\mr}{\mathrm}

% Standard vertical spacing
\newcommand{\stdvspace}{\vspace{0.5ex}}

% Text formatting
\newcommand{\tit}[1]{\textit{#1}}
\newcommand{\tbf}[1]{\textbf{#1}}

\begin{document}

\begin{Large}
    \textsf{\textbf{Real and Complex Analysis}}
\end{Large}

\vspace{1ex}

\textsf{\textbf{Author:}} Joshua Lin \\
\textsf{\textbf{Reference:}} Rudin, \tit{Real and Complex Analysis}

\vspace{2ex}

\section{Complex Measures}

References: Rudin \tit{Real and Complex Analysis}, Chapter 6. 
\stdvspace

\subsection{Total Variation}

\begin{definition}{Complex Measure}*
    Let \((X, \fr{M})\) be a measurable space. A complex measure \(\mu : \fr{M} \to \bb{C}\) satisfies \tit{countable additivity}, in the sense that if \(E \in \fr{M}\) is partitioned into \(\{E_n\}_{n} \subseteq \fr{M}\), then 
    \[
        \mu(E) = \sum_{n=1}^\infty \mu(E_n)
        \qquad (E \in \fr{M})
    \]
\end{definition}

The series must converge \tit{absolutely}, since the partition does not depend on the order of the sum. Neither does it depend on the particular choice of partition. This motivates the following definition. 

\begin{definition}{Total Variation Measure}*
    Let \((X, \fr{M}, \mu)\) be a complex measure space. For \(E \in \fr{M}\), let \(\fr{P}(E)\) denote the collection of (measurable) partitions of \(E\). The \tit{total variation measure} is
    \[
        |\mu|(E) := \sup_{\{E_n\}_n \in \fr{P}(E)} \, \sum_{n=1}^\infty |\mu(E_n)|
        \equiv \sup \sum_{n=1}^\infty |\mu(E_n)|
    \]

    and the \tit{total variation} of \(\mu\) is \(|\mu|(X)\). 
\end{definition}

The total variation measure \(|\mu|\) can be thought of as the smallest positive measure which dominates a (complex) measure \(\mu\). If \(\mu\) is a positive measure, though, of course \(|\mu| = \mu\). 
\stdvspace

\begin{theorem}{Total Variation Measure}*
    The total variation \(|\mu|\) of a complex measure \(\mu : \fr{M} \to \bb{C}\) is a positive measure. 
\end{theorem}

\begin{proof}
    To show countable additivity, first consider the forward direction \(\sum_n |\mu|(E_n) \leq |\mu|(E)\). The idea is to partition each of the \(E_n\), which collectively forms a new partition of \(E\). For each \(E_n\), there exists \(t_n \in \bb{R}\) such that \(|\mu|(E_n) > t_n\). By the approximation property of the supremum, there exists a partition \(\{E_{nm}\}_m\) such that 
    \[
        t_n < \sum_{m=1}^\infty |\mu(E_{nm})|
        \implies
        \sum_{n=1}^\infty t_n \leq \sum_{n,m=1}^\infty |\mu(E_{nm})|
        \leq |\mu|(E) 
    \]

    since \(\{E_{nm}\}_{n,m}\) partitions \(E\). However, taking the supremum over \(t_n\) yields \(\sum_n |\mu|(E_n) \leq |\mu|(E)\). For the reverse direction, we want to show that \(|\mu|(E) \leq \sum_n |\mu|(E_n)\). If we can show that
    \[
        \sum_{m=1}^\infty |\mu(A_m)| \leq \sum_{n=1}^\infty |\mu|(E_n)
    \]

    for any partition \(\{A_m\}_m\) of \(E\), then the claim will follow. The only way to make these two distinct partitions comparable is to look at their pairwise intersections \(\{E_n \cap A_m\}_{n,m}\), e.g. as follows: 
    \begin{align*}
        \sum_{m=1}^\infty |\mu(A_m)|
        &= \sum_{m=1}^\infty \left| \sum_{n=1}^\infty \mu(E_n \cap A_m) \right| 
        \leq \sum_{m=1}^\infty \sum_{n=1}^\infty |\mu(E_n \cap A_m)| \\
        &= \sum_{n=1}^\infty \sum_{m=1}^\infty |\mu(E_n \cap A_m)|
        \leq \sum_{n=1}^\infty |\mu|(E_n)
    \end{align*}

    so \(|\mu|(E) \leq \sum_n |\mu|(E_n)\). Hence actually \(|\mu|(E) = \sum_n |\mu|(E_n)\), proving countable additivity. Finally, to show that there exists a set \(F \in \fr{M}\) for which \(|\mu|(F) < \infty\), we can take \(F = \varnothing\). Since the only partition of \(\varnothing\) is \(\{\varnothing\}_n\) and \(\mu(\varnothing) = 0\), we conclude that \(|\mu|(\varnothing) = 0\). 
\end{proof}

\subsection{Bounded Variation}

\begin{lemma}{Average Modulus}*
    If \(\{z_i\}_{i=1}^N \subseteq \bb{C}\) then there is \(S \subseteq \{1, \dots, N\}\) such that 
    \[
        \left| \sum_{k \in S} z_k \right| \geq \frac{1}{\pi} \sum_{k=1}^N |z_k|
    \]
\end{lemma}

\begin{proof}
    Represent each \(z_k\) as \(|z_k| e^{i\alpha_k}\) for \(\alpha_k \in [-\pi, \pi]\). For each \(\theta \in [-\pi, \pi]\), we define \(S(\theta)\) to be the indices of the \(z_k\) which are “aligned” with \(\theta\), in the sense that 
    \[
        S(\theta) = \{k \mid \cos(\alpha_k - \theta) > 0\}
    \]

    By taking the real part, we can bound
    \begin{align*}
        \left| \sum_{k \in S(\theta)} z_k \right| 
        &\geq \Re \left( \sum_{k \in S(\theta)} |z_k| e^{i \alpha_k} \right)
        \geq \sum_{k \in S(\theta)} |z_k| \cos(\alpha_k - \theta)
        = \sum_{k=1}^N |z_k| \cos^+(\alpha_k - \theta)
        =: \varphi(\theta)
    \end{align*}

    Let \(\theta_0\) be the value of \(\theta\) which maximizes \(\varphi(\theta)\) on the compact interval \([- \pi, \pi]\). This value is at least as large as the average, \(\varphi(\theta_0) \geq \overline{\varphi(\theta)}\). But the average with respect to \(\theta\) is 
    \begin{align*}
        \overline{\varphi(\theta)} 
        &= \sum_{k=1}^N |z_k| \overline{\cos^+(\alpha_k - \theta)}
        = \sum_{k=1}^N \Bigl( |z_k| \cdot \tfrac{1}{2\pi} \int_{-\pi}^\pi \cos^+(\alpha_k - \theta) \, d\theta \Bigr) \\
        &= \sum_{k=1}^N |z_k| \cdot \tfrac{1}{2\pi} \int_{-\pi}^\pi \cos^+ \theta \, d\theta
        = \sum_{k=1}^N |z_k| \cdot \int_{-\pi/2}^{\pi/2} \cos \theta \, d\theta
        = \frac{1}{\pi} \sum_{k=1}^N |z_k| 
    \end{align*}

    Assigning \(S := S(\theta_0)\) therefore completes the proof. 
\end{proof}

\begin{theorem}{Bounded Variation}*
    If \(\mu\) is a complex measure on \(X\), then \(|\mu|(X) < \infty\).
\end{theorem}

\begin{proof}
    TODO: The intuition for why \(|\mu|\) must be finite is that 
\end{proof}

Now we can give some structure to complex measures on a measurable space. 

\begin{definition}{Complex Measures Form a Normed Space}*
    Let \((X, \fr{M})\) be a measurable space. The set of complex measures \(\{\mu : \fr{M} \to \bb{C}\}\) forms a normed vector space, with norm \(\|\mu\| := |\mu|(X)\). 
\end{definition}

\begin{definition}{Jordan Decomposition of Signed Measures}*
    Let \((X, \fr{M})\) be a measurable space, and suppose \(\mu : \fr{M} \to \bb{R}\) is a \tit{signed measure}. Then the \tit{positive and negative variations} of \(\mu\) are 
    \[
        \mu^+ = \tfrac{1}{2}(|\mu| + \mu), 
        \quad 
        \mu^- = \tfrac{1}{2}(|\mu| - \mu)
    \]

    which are both bounded positive measures, since \(\mu^+, \mu^- : \fr{M} \to [0, \infty)\). Furthermore, 
    \[
        \mu = \mu^+ - \mu^-, 
        \quad
        |\mu| = \mu^+ + \mu^-
    \]

    and this representation of \(\mu\) is its \tit{Jordan decomposition}. 
\end{definition}

\subsection{Absolute Continuity}

\begin{definition}{Absolute Continuity}*
    Let \((X, \fr{M})\) be a measurable space. Let \(\mu\) be a positive measure, and let \(\lambda\) be an arbitrary measure. Then \(\lambda\) is \tit{absolutely continuous with respect to} \(\mu\) if 
    \[
        \mu(E) = 0 \implies \lambda(E) = 0
        \qquad (E \in \fr{M})
    \]
    which is denoted by \(\lambda \ll \mu\). 
\end{definition}

\begin{definition}{Concentration of Measure}*
    Let \((X, \fr{M}, \lambda)\) be a measure space. If there exists \(A \in \fr{M}\) such that
    \[
        \lambda(E) = \lambda(A \cap E)
        \qquad (E \in \fr{M})
    \]
    then \(\lambda\) is \tit{concentrated} on \(A\). This is equivalent to saying \(\lambda(E) = 0\) whenever \(E \cap A = \varnothing\).
\end{definition}

\begin{proposition}{Basic Concentration Properties}*
    Suppose \(\mu, \lambda, \lambda_1, \lambda_2\) are measures on \((X, \fr{M})\) and \(\mu\) is positive. Then the following hold:
    \begin{enumerate}[(a)]
        \itemsep0em
        \item If \(\lambda\) is concentrated on \(A\), so is \(|\lambda|\). 
        \item If \(\lambda_1 \perp \lambda_2\), then \(|\lambda_1| \perp |\lambda_2|\). 
        \item If \(\lambda_1 \perp \mu\) and \(\lambda_2 \perp \mu\), then \(\lambda_1 + \lambda_2 \perp \mu\).
        \item If \(\lambda_1 \ll \mu\) and \(\lambda_2 \ll \mu\), then \(\lambda_1 + \lambda_2 \ll \mu\).
        \item If \(\lambda \ll \mu\), then \(|\lambda| \ll \mu\).
        \item If \(\lambda_1 \ll \mu\) and \(\lambda_2 \perp \mu\), then \(\lambda_1 \perp \lambda_2\). 
        \item If \(\lambda \ll \mu\) and \(\lambda \perp \mu\), then \(\lambda = 0\). 
    \end{enumerate}
\end{proposition}

\begin{proof}
    \tit{Part (a).} If \(E \cap A = \varnothing\), then
\end{proof}





\newpage
\section{Differentiation}

References: Rudin \tit{Real and Complex Analysis}, Chapter 7. 
\stdvspace

\tit{Remark.} The goal of this section is to motivate a definition of the derivative (of a measure) which will act as an “inverse” to Lebesgue integration.
\stdvspace

\tit{Convention.} The Lebesgue measure will be denoted \(m\), with the appropriate dimension up to specification. 

\begin{theorem}{Motivation for the Derivative}*
    Let \(\mu : \fr{M} \to \bb{C}\) be a complex Borel measure on \(\bb{R}\). Define 
    \[
        f(x) = \mu((-\infty, x)) 
        \qquad (x \in \bb{R})
    \]

    Then the following are equivalent: 
    \begin{enumerate}[(a)]
        \item \(f\) is differentiable at \(x\) with \(f'(x) = w \in \bb{C}\). 
        \item For each \(\varepsilon > 0\) there exists \(\delta > 0\) such that for any open interval \(I \ni x\), 
        \[
            m(I) < \delta 
            \implies
            \left| \frac{\mu(I)}{m(I)} - w \right| < \varepsilon
        \]
    \end{enumerate}
\end{theorem}

This theorem suggests defining the derivative of a measure as the limit of quotients \(\mu(I) / m(I)\). 

\begin{definition}{Symmetric Derivative}*
\end{definition}

\end{document}

