\input{preamble}
\usepackage{titlesec}
\usepackage[many]{tcolorbox}

% Adjust spacing after the chapter title
\titlespacing*{\chapter}{0cm}{-2.0cm}{0.50cm}
\titlespacing*{\section}{0cm}{0.50cm}{0.25cm}

\titleformat*{\section}   {\bfseries\sffamily\Large}
\titleformat*{\subsection}{\bfseries\sffamily\large}
\titleformat*{\subsubsection}{\bfseries\sffamily\normalsize}
\titleformat*{\paragraph} {\bfseries\sffamily\normalsize}
\titleformat*{\subparagraph}{\bfseries\sffamily\normalsize}

% Indent 
\setlength{\parindent}{0pt}
\setlength{\parskip}{1ex}

% --- Theorems, lemma, corollary, postulate, definition ---
\usepackage[dvipsnames]{xcolor}
\usepackage{titlesec}
\usepackage[many]{tcolorbox}
\tcbuselibrary{theorems,skins,breakable}

% Custom colors
\definecolor{MyBlue}{HTML}{37CDDE}
\definecolor{MyGreen}{HTML}{50C878}
\definecolor{MyPurple}{HTML}{7851A9}
\definecolor{MyGold}{HTML}{D4AF37}
\definecolor{MyGray}{HTML}{808080}

\newtcbtheorem[]{theorem}{Theorem}%
  {enhanced,
   colback        = red!20,
   colbacktitle   = red!30,
   coltitle       = black,
   boxrule        = 0pt,
   frame hidden,
   arc            = 2mm,
   before skip    = 3ex,
   after skip     = 3ex,
   before title   = {\vspace{2mm}},
   after title    = {\vspace{2mm}},
   fonttitle      = \bfseries\sffamily,
   breakable
  }{theorem}

\newtcbtheorem[]{proposition}{Proposition}%
  {enhanced,
   colback        = MyGold!20,
   colbacktitle   = MyGold!30,
   coltitle       = black,
   boxrule        = 0pt,
   frame hidden,
   arc            = 2mm,
   before skip    = 3ex,
   after skip     = 3ex,
   before title   = {\vspace{2mm}},
   after title    = {\vspace{2mm}},
   fonttitle      = \bfseries\sffamily,
   breakable
  }{proposition}

\newtcbtheorem[]{lemma}{Lemma}%
  {enhanced,
   colback        = MyBlue!20,
   colbacktitle   = MyBlue!30,
   coltitle       = black,
   boxrule        = 0pt,
   frame hidden,
   arc            = 2mm,
   before skip    = 3ex,
   after skip     = 3ex,
   before title   = {\vspace{2mm}},
   after title    = {\vspace{2mm}},
   fonttitle      = \bfseries\sffamily,
   breakable
  }{lemma}

\newtcbtheorem[]{corollary}{Corollary}%
  {enhanced,
   colback        = MyGreen!20,
   colbacktitle   = MyGreen!30,
   coltitle       = black,
   boxrule        = 0pt,
   frame hidden,
   arc            = 2mm,
   before skip    = 3ex,
   after skip     = 3ex,
   before title   = {\vspace{2mm}},
   after title    = {\vspace{2mm}},
   fonttitle      = \bfseries\sffamily,
   breakable
  }{corollary}

\newtcbtheorem[]{definition}{Definition}%
  {enhanced,
   colback        = MyPurple!20,
   colbacktitle   = MyPurple!30,
   coltitle       = black,
   boxrule        = 0pt,
   frame hidden,
   arc            = 2mm,
   before skip    = 3ex,
   after skip     = 3ex,
   before title   = {\vspace{2mm}},
   after title    = {\vspace{2mm}},
   fonttitle      = \bfseries\sffamily,
   breakable
  }{definition}

\newtcbtheorem[]{remark}{Remark}%
  {enhanced,
   colback        = MyGray!20,
   colbacktitle   = MyGray!30,
   coltitle       = black,
   boxrule        = 0pt,
   frame hidden,
   arc            = 2mm,
   before skip    = 3ex,
   after skip     = 3ex,
   before title   = {\vspace{2mm}},
   after title    = {\vspace{2mm}},
   fonttitle      = \bfseries\sffamily,
   breakable
  }{remark}

\newtcbtheorem[]{example}{Example}%
  {enhanced,
   colback        = Violet!20,
   colbacktitle   = Violet!30,
   coltitle       = black,
   boxrule        = 0pt,
   frame hidden,
   arc            = 2mm,
   before skip    = 3ex,
   after skip     = 3ex,
   before title   = {\vspace{2mm}},
   after title    = {\vspace{2mm}},
   fonttitle      = \bfseries\sffamily,
   breakable
  }{example}

\newtcbtheorem[]{problem}{Problem}%
  {enhanced,
   colback        = MyGold!20,
   colbacktitle   = MyGold!30,
   coltitle       = black,
   boxrule        = 0pt,
   frame hidden,
   arc            = 2mm,
   before skip    = 3ex,
   after skip     = 3ex,
   before title   = {\vspace{2mm}},
   after title    = {\vspace{2mm}},
   fonttitle      = \bfseries\sffamily,
   breakable
  }{problem}


% --- Basic commands ---
%   Euler's constant
\newcommand{\eu}{\mathrm{e}}

%   Imaginary unit
\newcommand{\im}{\mathrm{i}}

%   Sexagesimal degree symbol
\newcommand{\grado}{\,^{\circ}}

% --- Comandos para álgebra lineal ---
% Matrix transpose
\newcommand{\transpose}[1]{{#1}^{\mathsf{T}}}

%%% Comandos para cálculo
%   Definite integral from -\infty to +\infty
\newcommand{\Int}{\int\limits_{-\infty}^{\infty}}

%   Indefinite integral
\newcommand{\rint}[2]{\int{#1}\dd{#2}}

%  Definite integral
\newcommand{\Rint}[4]{\int\limits_{#1}^{#2}{#3}\dd{#4}}

%   Dot product symbol (use the command \bigcdot)
\makeatletter
\newcommand*\bigcdot{\mathpalette\bigcdot@{.5}}
\newcommand*\bigcdot@[2]{\mathbin{\vcenter{\hbox{\scalebox{#2}{$\m@th#1\bullet$}}}}}
\makeatother

%   Hamiltonian
\newcommand{\Ham}{\hat{\mathcal{H}}}

%   Trace
\renewcommand{\Tr}{\mathrm{Tr}}

% Christoffel symbol of the second kind
\newcommand{\christoffelsecond}[4]{\dfrac{1}{2}g^{#3 #4}(\partial_{#1} g_{#2 #4} + \partial_{#2} g_{#1 #4} - \partial_{#4} g_{#1 #2})}

% Riemann curvature tensor
\newcommand{\riemanncurvature}[5]{\partial_{#3} \Gamma_{#4 #2}^{#1} - \partial_{#4} \Gamma_{#3 #2}^{#1} + \Gamma_{#3 #5}^{#1} \Gamma_{#4 #2}^{#5} - \Gamma_{#4 #5}^{#1} \Gamma_{#3 #2}^{#5}}

% Covariant Riemann curvature tensor
\newcommand{\covariantriemanncurvature}[5]{g_{#1 #5} R^{#5}{}_{#2 #3 #4}}

% Ricci tensor
\newcommand{\riccitensor}[5]{g_{#1 #5} R^{#5}{}_{#2 #3 #4}}

% Shortcuts
\newcommand{\cl}{\mathcal}
\newcommand{\bb}{\mathbb}
\newcommand{\mb}{\mathbf}
\newcommand{\fr}{\mathfrak}
\newcommand{\oname}{\operatorname}
\newcommand{\ovl}{\overline}
\newcommand{\st}{\;\biggr\rvert\;}
\newcommand{\mr}{\mathrm}
\newcommand{\ds}{\mathds}

% Standard vertical spacing
\newcommand{\stdvspace}{\vspace{0.5ex}}

% Text formatting
\newcommand{\tit}[1]{\textit{#1}}
\newcommand{\tbf}[1]{\textbf{#1}}

% Horizontal line
\newcommand{\horizontal}{\noindent{\rule{\textwidth}{0.4pt}}}



\begin{document}

\begin{Large}
    \textsf{\textbf{Probability Theory (Princeton)}}
\end{Large}

\vspace{1ex}

\textsf{\textbf{Student:}} Joshua Lin \\
\textsf{\textbf{Lecturer:}} Allan Sly

\vspace{2ex}

Problems are largely derived from \textit{Probability and Measure} by Billingsley (BLN) and \textit{Mathematical Statistics} by Jun Shao (JNS).
\stdvspace

Convention: \textit{independence} will be taken to imply \textit{pairwise-independence} unless otherwise stated. The notation \(\ev{V}_n\) will denote the average of \(V_1, \dots, V_n\), and \(\ev{V} \equiv \lim_n \ev{V}_n\). \(\lambda\) will denote the Lebesgue measure on the appropriate domain.

\section{Probability}


\begin{problem}{Borel-Cantelli Lemmas}{borel_cantelli}
    [BLN Thm 4.3/4.4.] Prove the following:
    \begin{enumerate}[(a)]
        \itemsep0em
        \item \textit{First Borel-Cantelli Lemma:} If \(\sum_n \bb{P}[A_n]\) converges, then \(\bb{P}[\limsup_n A_n] = 0\).
        \item \textit{Second Borel-Cantelli Lemma:} If \(\{A_n\}_n\) are independent and \(\sum_n \bb{P}[A_n]\) diverges, then \(\bb{P}[\limsup_n A_n] = 1\).
    \end{enumerate}
\end{problem}


\begin{proof}
    The first Borel-Cantelli lemma is proved (for general measure spaces, even) in in the \textit{Measures} section of \texttt{real-analysis}. For the second Borel-Cantelli lemma, we can show that \((\limsup_n A_n)^c\) occurs with probability zero. Effectively,
    \[
        \bb{P}\left[\limsup_{n \to \infty} A_n\right] = 1
        \iff \bb{P}\left[ \left(\bigcap_{n=1}^\infty \bigcup_{k=n}^\infty A_k\right)^c \right] = 0
        \iff \bb{P}\left[ \bigcap_{k=n}^\infty A_k^c \right] = 0
        \quad (n \in \bb{N})
    \]
    By independence,
    \[
        \bb{P}\left[ \bigcap_{k =n}^\infty A_k^c \right]
        = \prod_{k=n}^\infty \bb{P}[A_k^c]
        = \prod_{k=n}^\infty (1 - \bb{P}[A_k])
        \leq \prod_{k=n}^\infty \exp(-\bb{P}[A_k])
        = \exp\left( -\sum_{k=n}^\infty \bb{P}[A_k] \right)
    \]
    Since \(\sum_k \bb{P}[A_k] = +\infty\), the final expression vanishes, as desired. Note that there is no measure-theoretic generalization of the second lemma, due to the requirement of independence.
\end{proof}


\begin{problem}{Kolgomorov's Zero-One Law}{kolgomorov_zero_one}
    [BLN Thm 4.5.] If \(\{A_n\}_n\) is an independent sequence of events, and if \(A\) is an event in the tail \(\sigma\)-field defined by
    \[
        \cl{T} = \bigcap_{n=1}^\infty \sigma \left( \{A_k\}_{k=n}^\infty \right)
    \]
    then \(\bb{P}(A)\) is either \(0\) or \(1\).
\end{problem}

\begin{proof}
    It is sufficient to prove that \(\bb{P}[A] = \bb{P}[A]^2\), which would follow from \(A \perp A\), i.e. \(A\) being independent of itself. Note that for each \(n\), \(A \in \sigma(\{A_n, A_{n+1}, \dots\})\) so that \(A_1, \dots, A_{n-1}, A\) are independent, and in particular \(A \perp \sigma(\{A_1, A_2, \dots \})\). However, \(A \in \cl{T} \subseteq \sigma(\{A_1, A_2, \dots, \})\), so \(A\) is independent of itself. The claim follows. 
\end{proof}


\begin{problem}{Monotonicity of \(L^p\)-norm for Simple Random Variables}{lp_monotonicity}
    [BLN Eq 5.33.] Prove that if the \(L^p\) norm of a simple random variable \(X\) is \(\|X\|_p := \sqrt[p]{\bb{E}[|X|^p]}\), then for any \(0 < \alpha \leq \beta\), \(\|X\|_\alpha \leq \|X\|_\beta\).
\end{problem}


\begin{proof}
    It will directly follow from H\"older's inequality: note that since \(\bb{P}(\Omega) = 1\), we have \(\|Y\|_1 \leq \|Y\|_p\) for \(p \geq 1\). Choose \(Y = X^a\) and \(p = b/a\), and the result follows.
\end{proof}


\begin{problem}{Cantelli's Inequality}{cantelli_inequality}
    [BLN P5.5.] Suppose \(X\) is a random variable with mean \(\mu\) and variance \(\sigma^2\).
    \begin{enumerate}[(a)]
        \itemsep0em
        \item Prove \textit{Cantelli's inequality},
        \[
            \bb{P}[X - \mu \geq \alpha] \leq \frac{\sigma^2}{\sigma^2 + \alpha^2}
            \qquad (\alpha \geq 0)
        \]
        \item Prove \(\bb{P}[|X - \mu| \geq \alpha] \leq 2\sigma^2 / (\sigma^2 + \alpha^2)\). When is this tighter than Chebyshev's inequality?
        \item By considering a particular simple random variable, show Cantelli's inequality is sharp.
    \end{enumerate}
\end{problem}

\begin{proof}
    Without loss of generality, let \(\mu = 0\). Then by Markov's inequality,
    \begin{align*}
        \bb{P}[X \geq \alpha]
        &= \bb{P}[X + c \geq \alpha + c]
        \leq \bb{P}[|X + c| \geq \alpha + c]
        \leq \frac{1}{(\alpha + c)^2} \int_{\Omega} (X + c)^2 \, \mr{d}\bb{P}
        = \frac{\sigma^2 + c^2}{(\alpha + c)^2}
    \end{align*}
    Optimizing the bound with respect to \(c\), we find the minimizer to be \(c = \alpha^{-1} \sigma^2\). Substituting yields Cantelli's inequality. From then it follows that
    \[
        \bb{P}[|X - \mu| \geq \alpha]
        = \bb{P}[X - \mu \geq \alpha] + \bb{P}[-(X - \mu) \geq \alpha]
        \leq \frac{2\sigma^2}{\sigma^2 + \alpha^2}
    \]
    Finally, to show Cantelli's inequality is sharp, let \(X\) have the Rademacher distribution. If \(\alpha = 1\),
    \[
        \bb{P}[X \geq \alpha] = \frac{1}{2} = \frac{\sigma^2}{\sigma^2 + \alpha^2}
    \]
    so equality is attained.
\end{proof}


\begin{problem}{The (Simple) Law of Large Numbers}{law_of_large_numbers}
    [BLN Thm 6.1, 6.2.] Prove the following:
    \begin{enumerate}[(a)]
        \itemsep0em
        \item \textit{The Strong Law:} If \(\{X_n\}_n\) are simple i.i.d. with expectation \(\mu\), then
        \[
            \bb{P}\left[\ev{X} = \mu\right]
            \equiv \bb{P}\left[ \lim_{n \to \infty} \frac{1}{n} \sum_{k=1}^n X_k = \mu \right]
            = 1
        \]
        \item \textit{The Weak Law:} Let \((\Omega_n, \fr{M}_n, \bb{P}_n)\) be a sequence of real probability spaces. If \(\{X_{nk}\}_{k=1}^{r_n}\) are simple and independent in \(k\) for each \(n\), and \(\{v_n\}_n\) satisfies \(\sigma_n / v_n \to 0\), then
        \[
            \lim_{n \to \infty} \bb{P}_n \left[ \left| \frac{\ev{X_n} - \mu_n}{v_n} \right| \geq \varepsilon \right] = 0
        \]
        Note that \((\Omega_n, \fr{M}_n, \bb{P}_n) = (\Omega, \fr{M}, \bb{P})\), \(X_{nk} \sim X\), and \(v_n = n\) reduces to the corresponding weak law for the strong law above. 
    \end{enumerate}
\end{problem}

\begin{proof}
    \textit{Part (a).} Let \(S_n = \sum_{k=1}^n X_k\), and without loss of generality let \(\mu = 0\). Observe, 
    \[
        \Omega \setminus \left\{\lim_{n \to \infty} n^{-1} S_n = 0\right\} 
        = \bigcup_{\varepsilon \in \bb{Q}_+} \left\{\limsup_{n \to \infty} |n^{-1} S_n| \geq \varepsilon \right\}
    \]
    It therefore suffices to show that \(\sum_n \bb{P}[|n^{-1} S_n| \geq \varepsilon] = 0\), since then by the first Borel-Cantelli Lemma we show the complement above occurs almost surely. The idea is that we can apply Markov's inequality using the fourth moment: we need an even moment to eliminate the absolute value, and we cannot use the second moment since we need to show the individual probabilities are \(\Theta(n^{-2})\) for the series to converge. Hence,
    \begin{align*}
        \bb{P}[|S_n| \geq n \varepsilon]
        &\leq \frac{1}{n^4 \varepsilon^4} \int_{\Omega} S_n^4 \, \mr{d}\bb{P} 
        = \frac{1}{n^4 \varepsilon^4} \int_{\Omega} \sum_{\alpha=1}^n \sum_{\beta=1}^n \sum_{\gamma=1}^n \sum_{\zeta=1}^n X_\alpha X_\beta X_\gamma X_\zeta \, \mr{d}\bb{P} \\
        &= \frac{1}{n^4 \varepsilon^4} (n \bb{E}[X^4] + 3n(n-1) \bb{E}[X_\alpha^2 X_\beta^2])
        \leq \frac{K}{n^2 \varepsilon^4}
    \end{align*}
    for some constant \(K\) and \(n\) large enough (note that the expectations finitely exist since \(X\) is simple). As a result, \(\bb{P}[|S_n| \geq n \varepsilon]\) converges in sum, completing the proof. 
\end{proof}

\begin{proof}
    \textit{Part (b).} Without loss of generality, let \(\mu_n = 0\) and \(v_n > 0\). By Markov's inequality, 
    \[
        \bb{P}_n[|X_n| \leq v_n \varepsilon]
        \leq \frac{\sigma_n^2}{v_n^2} \frac{1}{\varepsilon^2} \stackrel{n \to \infty}{\longrightarrow} 0
    \]
    thus proving the weak law. 
\end{proof}


\begin{problem}{Poisson's Theorem}{poisson_theorem}
    [BLN P6.5.] If \(\{A_n\}_n\) are independent,
    \[
        \lim_{n \to \infty} \bb{P}\left[ |\ev{\ds{1}_A}_n - \ev{\bb{P}[A]}_n | \geq \varepsilon \right]
        = \lim_{n \to \infty} \bb{P}\left[ \left| \frac{1}{n} \sum_{k=1}^n \ds{1}_{A_k} - \frac{1}{n} \sum_{k=1}^n \bb{P}(A_k) \right| \geq \varepsilon \right] 
        = 0
    \]
    In other words, a sequence of independent but distinct Bernoulli trials behaves, on average, like a Bernoulli of its mean parameter.
\end{problem}

\begin{proof}
    Let \(X_k := \ds{1}_{A_k} - \bb{P}[A_k]\), so that \(\bb{E}[X_k] = 0\). Define \(S_n := \sum_{k=1}^n X_k\). It suffices to show that \(\bb{P}[|S_n| \geq n \varepsilon] \to 0\) as \(n \to \infty\). By Markov's inequality, 
    \begin{align*}
        \bb{P}[|S_n| \geq n \varepsilon] 
        &\leq \frac{1}{n^4 \varepsilon^4} \int_{\Omega} S_n^4 \, \mr{d}\bb{P}
        = \frac{1}{n^4 \varepsilon^4} \int_\Omega \sum_{\alpha,\beta,\gamma,\zeta} X_\alpha X_\beta X_\gamma X_\zeta \, \mr{d}\bb{P}
        \leq \frac{K}{n^2 \varepsilon^4} 
    \end{align*}
    for some constant \(K\) large enough, so \(\bb{P}[|S_n| \geq n \varepsilon] \to 0\) as desired. 
\end{proof}


\begin{problem}{Cantelli's Theorem}{cantelli_theorem}
    [BLN P6.6.] If \(\{X_n\}_n\) are independent with zero mean and uniformly bounded fourth moments, then \(\ev{X} = 0\) almost surely. In particular, the \(X_n\) need not be identically distributed. 
\end{problem}

\begin{proof}
    Observe that 
    \[
        \left\{ \lim_{n \to \infty} \frac{1}{n} \sum_{k=1}^n X_k = 0 \right\}
        = \Omega - \bigcup_{\varepsilon \in \bb{Q}} \left\{ \left| \limsup_{n \to \infty} \frac{1}{n} \sum_{k=1}^n X_k \right| \geq \varepsilon \right\}
        = \Omega - \bigcup_{\varepsilon \in \bb{Q}} \left\{ \left| \frac{1}{n} \sum_{k=1}^n X_k \right| \geq \varepsilon \text{ i.o.} \right\}
    \]
    If \(S_n := \sum_{k=1}^n X_k\), it suffices by the first Borel-Cantelli lemma to show that \(\sum_n \bb{P}[|S_n| \geq n \varepsilon]\) converges. By Markov's inequality, 
    \begin{align*}
        \bb{P}[|S_n| \geq n \varepsilon] 
        &= \frac{1}{n^4 \varepsilon^4} \int_{\Omega} \sum_{\alpha, \beta, \gamma, \zeta} X_\alpha X_\beta X_\gamma X_\zeta \, \mr{d}\bb{P} 
        \leq \frac{1}{n^4 \varepsilon^4} \left(\sum_\alpha \bb{E}[X_\alpha^4] + \sum_{\alpha, \beta} \bb{E}[X_\alpha^2 X_\beta^2] \right) \\
        &\leq \frac{1}{n^4 \varepsilon^4} \left(n \sup_\alpha \bb{E}[X_\alpha^4] + 3n(n-1) \sup_{\alpha, \beta} \sqrt{\bb{E}[|X_\alpha|] \bb{E}[|X_\beta|]} \right) \\
        &\leq \frac{1}{n^4 \varepsilon^4} \left(n \sup_\alpha \bb{E}[X_\alpha^4] + 3n(n-1) \sup_\alpha \bb{E}[|X_\alpha|] \right) \\
        &\leq \frac{1}{n^4 \varepsilon^4} \left(n \sup_\alpha \bb{E}[X_\alpha^4] + 3n(n-1) \sup_\alpha \sqrt[4]{\bb{E}[X_\alpha^4]} \right) 
        \leq \frac{K}{n^2 \varepsilon^4}
    \end{align*}
    where to arrive at the penultimate step, we use monotonicity of the \(L^p\)-norm on random variables. Note that implicitly, we invoke independence to eliminate odd powers in \(\sum X_\alpha X_\beta X_\gamma X_\zeta\). 
\end{proof}


\begin{problem}{The Square Subsequence Lemma}{square_subsequence_lemma}
    [BLN P6.7.] Suppose \(n^{-2} S_{n^2} \to 0\) almost surely and the \(X_n\) are uniformly bounded. Show \(\ev{X} = 0\) almost surely. Notably, the \(X_n\) need not be identically distributed or independent. 
\end{problem}

\begin{proof}
    It suffices to show that \(\limsup_m m^{-1} |S_m| = 0\) almost surely. Fix \(m\), and let \(n\) be the largest number satisfying \(n^2 \leq m < (n+1)^2\). Define \(R_{n,m} := \sum_{k=n^2+1}^m X_k\) so that \(S_m = S_{n^2} + R_{n,m}\). Let \(B\) be the uniform bound on \(\{X_k\}_k\), so that \(|R_{n,m}| \leq (m-n) K \leq (2n+1) K \leq 3nK\). It follows: 
    \[
        \frac{|S_m|}{m} \leq \frac{|S_{n^2}|}{n^2} + \frac{|R_{n,m}|}{n^2} \leq \frac{|S_{n^2}|}{n^2} + \frac{3B}{n}
    \]
    Taking the limit superior of both sides, 
    \[
        \limsup_{m \to \infty} \frac{|S_m|}{m} 
        \leq \limsup_{n \to \infty} \left( \frac{|S_{n^2}|}{n^2} + \frac{3B}{n} \right)
        \leq \limsup_{n \to \infty} \frac{|S_{n^2}|}{n^2} + \limsup_{n \to \infty} \frac{3B}{n}
        = 0
    \]
    almost surely. Hence \(\ev{X} = \lim_m m^{-1} |S_m| = 0\) almost surely. 
\end{proof}



\begin{problem}{Shannon's Theorem}{shannons_theorem}
    [BLN P6.14.] Suppose that \(\{X_n\}_n\) are simple i.i.d. with \(\text{image}(X) = \{1, \dots, r\}\) and \(p_k := \bb{P}[X_n = k]\). For \(\omega \in \Omega\), let \(P_n(\omega) = \bb{P}[(X_1(\omega)] \cdots \bb{P}[X_n(\omega)]\) be the probability that a sequence of \(n\) trials would produce the particular sequence \(X_1(\omega), \dots, X_n(\omega)\). Prove the following: 
    \begin{enumerate}[(a)]
        \itemsep0em
        \item \textit{Entropy:} With probability \(1\), 
        \[
            \lim_{n \to \infty} \left[ - \frac{1}{n} \log P_n(\omega) \right] 
            = - \sum_{i=1}^r p_i \log p_i 
            \equiv H
        \]
        In information theory \(\{1, \dots, r\}\) are the letters of an \textit{alphabet}, the \(\{X_n\}_n\) are successive letters produced by an information \textit{source}, and \(H\) is the \textit{entropy} of the source. 
        \item \textit{Asymptotic Equipartition Property:} For large enough \(n\) there is a probability exceeding \(1 - \varepsilon\) that the probability \(P_n(\omega)\) of the observed \(n\)-long sequence, or \textit{message}, is in the range \(\exp{-n(H \pm \varepsilon)}\).
    \end{enumerate}
\end{problem}

\begin{proof}
    For the entropy definition, let \(Y_k(\omega) = - \log \bb{P}[X_k(\omega)]\), so the \(\{Y_k\}_k\) are i.i.d. Note that \(\bb{E}[Y] = H\), so the result follows from the Strong Law of Large Numbers. TODO: second part. 
\end{proof}


\begin{problem}{The R\'enyi-Lamperti Lemma}{renyi_lamperti_lemma}
    TODO. 
\end{problem}

\newpage
\section{Distributions}

\begin{problem}{Change of Variables for Densities}{change_of_variables_densities}
    [JNS Prop. 1.8.] Let \(X : \Omega \to \bb{R}^n\) have Lebesgue p.d.f. \(f_X\), and suppose \(Y = g \circ X\) for a Borel function \(g : (\bb{R}^n, \cl{B}(\bb{R}^n)) \to (\bb{R}^n, \cl{B}(\bb{R}^n))\). Suppose \(\{U_j\}_{j=1}^m \subseteq \cl{B}(\bb{R}^n)\) is a disjoint open partition of \(\bb{R}^n\) up to measure zero, and that \(g_j := g|_{U_j}\) is a \(\cl{C}^1\) diffeomorphism onto its image. Then \(Y\) has Lebesgue p.d.f.
    \[
        f_Y 
        = \sum_{j=1}^m |\oname{Det}(Dh_j)| (f_X \circ h_j) \ds{1}_{g(U_j)} 
    \]
    where \(h_j : g(U_j) \to U_j\) is the inverse of \(g_j\) from the Inverse Function Theorem.
\end{problem}

\begin{proof}
    Let \((\Omega, \bb{P})\) be a measurable space and \(E \in \cl{B}(\bb{R}^n)\). Then
    \begin{align*}
        \bb{P}(Y \in E)
        &= \bb{P}(g(X) \in E)
        = \bb{P}(X \in g^{-1}(E)) 
        = \bb{P}\left(X \in \bigcup_{j=1}^m (U_j \cap g^{-1}(E))\right) \\
        &= \bb{P}\left(X \in \bigcup_{j=1}^m g_j^{-1}(E)\right) 
        = \bb{P}\left( X \in \bigcup_{j=1}^m h_j(E \cap g(U_j)) \right) = \sum_{j=1}^m \bb{P}(X \in h_j(E \cap g(U_j))) \\
        &= \sum_{j=1}^m \int_{h_j(E \cap g(U_j))} f_X \, \mr{d}\lambda
        = \sum_{j=1}^m \int_{E} |\oname{Det}(Dh_j)| (f_X \circ h_j) \ds{1}_{g(U_j)} \, \mr{d}\lambda \\
        &= \int_{E} \sum_{j=1}^m |\oname{Det}(h_j)| (f_X \circ h_j) \ds{1}_{g(U_j)} \, \mr{d}\lambda 
        \equiv \int_{E} f_Y \, \mr{d}\lambda
    \end{align*}
    By uniqueness of the Radon-Nikodym derivative, \(f_Y\) is the Lebesgue p.d.f. of \(Y\).
\end{proof}

\begin{problem}{Sum and Quotient Rules}{sum_and_quotient_rules}
    [JNS Ex. 1.15.] Let \(X = (X_1, X_2)\) have a joint Lebesgue p.d.f. \(f_X\). Then the Lebesgue p.d.f.s of \(Y = X_1 + X_2\) and \(Z = X_1 / X_2\) are given by
    \begin{align*}
        & f_Y(y) = \int_{\bb{R}} f_X(x, y - x) \, \mr{d}x
        = \int_{\bb{R}} f_{X_1}(x) f_{X_2}(y - x) \, \mr{d}x \\
        & f_Z(z) = \int_{\bb{R}} |y| f_X(zy, y) \, \mr{d}y
        = \int_{\bb{R}} |y| f_{X_1}(zy) f_{X_2}(y) \, \mr{d}y
    \end{align*}
    where the second equalities hold if \(X_2 \neq 0\) a.s.
\end{problem}

\begin{proof}
\end{proof}

\begin{problem}{Cauchy Distribution}{cauchy_distribution}
    [JNS Ex. 1.15.] If \(X\) and \(Y\) are independent standard normal random variables, then \(Z = X / Y\) has Lebesgue p.d.f.
    \[
        f_Z(z) = \frac{1}{\pi(1 + z^2)}
    \]
    which is the standard Cauchy distribution, and \(Z\) has undefined mean and variance.
\end{problem}

\begin{problem}{$t$-distribution and $F$-distribution}{t_and_f_distributions}
    [JNS Ex. 1.16.] If \(X\) and \(Y\) are independent \(\chi^2\) random variables with \(m\) and \(n\) degrees of freedom, respectively, then
    \begin{enumerate}[(a)]
        \itemsep0em
        \item \(T = Z / \sqrt{X/m}\) has the t-distribution with \(m\) degrees of freedom, where \(Z\) is a standard normal random variable independent of \(X\). 
        \item \(F = (X/m) / (Y/n)\) has the \(F\)-distribution with \((m,n)\) degrees of freedom.
    \end{enumerate}
\end{problem}

\newpage
\section{Markov Chains}


\end{document}

